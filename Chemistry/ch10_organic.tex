\documentclass[Chemistry.tex]{subfiles}
\begin{document}
\chapter{Organic Chemistry}
\section{Introductory topics}
Organic chemistry is the study of compounds containing carbon, except its oxides and carbonates.

Carbon is a group IV element and can form up to four bonds. It usually does so in all compounds except when it is a cation (known as a carbocation).
\subsection{Formulae of organic compounds}
A \sldef{molecular formula} shows the actual number of each type of atom in a compound, but it does not show any form of structure or arrangement of atoms.

A \sldef{structural formula} shows the arrangement of atoms within the molecule; it shows the sequence in which atoms are bonded to each other.

A \sldef{displayed formula} or \sldef{full structural formula} shows all bonds in the molecule.

A \sldef{skeletal formula} is a simplified structural formula where all carbon atoms in alkyl chains and the hydrogens attached to them are removed, represented only by a carbon skeleton.
\subsection{Basic nomenclature}
Organic compounds can be classified into aliphatic, alicyclic and aromatic compounds. \sldef{Aliphatic compounds} are compounds without any aromatic ring like benzene; \sldef{alicyclic compounds} are simply cyclic aliphatic compounds like cyclopropane. Finally, \sldef{aromatic compounds} have an aromatic ring e.g. benzene.

\sldef{Functional groups} are specific groups of atoms that give a compound its characteristic chemical properties.

A \sldef{homologous series} is a series of compounds with the same functional group and general formula. Generally, each member has similar chemical properties and different physical properties, because they have the same functional group, but each successive compound differs by a methylene bridge (\slch{-CH2{-}}).

In IUPAC nomenclature, organic compound names are constructed using prefixes, suffixes and infixes. The primary functional group is used as the suffix; the number of carbon atoms in the longest chain containing the primary functional group is used as the stem, and any other secondary functional groups form the prefix. Prefixes are arranged in alphabetical order; the primary functional group is selected based on a priority.

All functional groups are numbered where needed if the name would otherwise be ambiguous. Each number is called a locant. Where there is a choice of locants, the selection is done based on the guidelines that \begin{slinenum}
\item the suffix functional group should have the lowest locants, by the first point of difference rule
\item any double or triple bonds should have the lowest locants, by the first point of difference rule.
\item the prefix functional groups should have the lowest locants, by the first point of difference rule.
\end{slinenum} E.g. 2,4,4-trimethylhexane, not 3,3,5-trimethylhexane.

The \sldef{first point of difference rule} is as follows. List the locants out in numerical order, and compare each term. At the first difference i.e. different locant, the locant set with the smaller locant is the correct one.
\subsection{Isomerism}
\sldef{Isomers} are compounds that have the same molecular formula but a different structural formula. There are two general types of isomerism: structural isomerism and stereoisomerism.

\sldef{Structural isomerism} is further divided into chain, positional and functional group isomerism. \sldef{Chain isomerism} is different arrangements of the carbon chain --- straight or branched. \sldef{Positional isomerism} is having different positions of substituents or functional groups on a chain or ring e.g. 1-chloropropane and 2-chloropropane. \sldef{Functional group} isomerism is having different functional groups due to different arrangement of atoms e.g. ethanoic acid (\slch{CH3COOH}) and methyl methanoate (\slch{HCOOCH3}).

\sldef{Stereoisomerism} is further divided into geometric and optimal isomerism.

\sldef{Geometric isomerism} occurs only when there is restricted rotation (e.g. due to a \slch{C=C} double bond) and there are two or more different substituents attached to the carbons of the double bond. This creates two geometric isomers with different physical properties but similar chemical properties.

Geometric isomers are named using \emph{E}/\emph{Z} or \emph{cis}-/\emph{trans}- nomenclature. If the substituents with higher CIP priority are on the same side, it is the \emph{Z} or \emph{cis} isomer, and if they are on different sides, then it is the \emph{E} or \emph{trans} isomer. E.g. (\emph{E})-but-2-ene and (\emph{Z})-but-2-ene.

The number of geometric isomers of a compound is \(2^n\), where \(n\) is the number of double bonds having geometric isomerism.

A carbon atom with 4 different substituents is a \sldef{chiral carbon}. If the compound it is in has no plane of symmetry, then the compound exhibits optical isomerism. An optical isomer is known as an \sldef{enantiomer}.

Optical isomers are named using \emph{R}/\emph{S} nomenclature. If the carbon atom is viewed with the lowest priority group pointing away, then the three remaining substituents appear to form a trigonal shape. If substituents go from highest priority to lowest priority in the clockwise direction, then it is the \emph{R} isomer; otherwise, it is the \emph{S} isomer.

Enantiomers are said to be optically active as they can rotate linearly polarised light. Enantiomers can also be named in the direction they rotate plane-polarise light; if it rotates clockwise, then it is the (+) enantiomer; otherwise it is the (-) enantiomer. (This is not related to the \emph{R}/\emph{S} system.) Enantiomers otherwise have identical chemical and physical properties.

The number of enantiomers of a compound is given by \(2^m\), where \(m\) is the number of chiral centres.

A \sldef{racemic mixture} is one with equal amounts of both (+) and (-) enantiomers. Such a mixture does not rotate plane-polarised light as there is exact cancellation of all possible rotation of plane polarised light.
\subsubsection{CIP priority}
In the CIP system, priority is assigned by comparing the atoms directly attached to the carbon atom. The group with the atom of highest atomic number has the highest priority, and so on.

If there is a tie, then atoms two bonds away from the carbon must be considered, by making a list of atoms, in descending atomic number order, connected to each atom one bond away from the carbon atom, and the group with the highest atomic number at the first point of difference wins.

This is repeated for more bonds away until the tie is broken. If groups differ only in \emph{E}/\emph{Z} or \emph{R}/\emph{S} isomerism, then the \emph{R} and \emph{Z} isomers are of higher priority. If a \emph{R}/\emph{S} descriptor is decided in such a manner, then the lowercase \emph{r} and \emph{s} are used instead. E.g. (1\emph{R},2\emph{s},3\emph{S})-1,2,3-trichlorocyclopentane.
\subsection{Hybridisation}
\sldef{Hybridisation} is the mixing of atomic orbitals to form hybrid orbitals suitable for the pairing of electrons to form chemical bonds in the valence bond theory.

In general, if a carbon is bonded to 2 atoms, it is \(sp\) hybridised; if it is bonded to 3, then it is \(sp^2\) hybridised, and if it is bonded to 4, then it is \(sp^3\) hybridised.

\(sp\) hybrid orbitals are formed by one \(s\) and one \(p\) orbital to form 2 \(sp\) hybrid orbitals; 2 \(p\) orbitals remain unhybridised. The shape about the carbon is linear.

\(sp^2\) hybrid orbitals are formed by one \(s\) and two \(p\) orbitals to form 3 \(sp^2\) orbitals; 1 \(p\) orbital remains unhybridised. The shape about the carbon is trigonal planar.

\(sp^3\) hybrid orbitals are formed by one \(s\) and three \(p\) orbitals to form 4 \(sp^3\) orbitals. The shape about the carbon is tetrahedral.

Note that benzene is \(sp^2\) hybridised; there is only 1 hydrogen bonded to each carbon, as each carbon-carbon bond is a 1.5-bond.

\(sp^x\) notation indicates the ratio of \(s\) to \(p\) character, which is 1 to \(x\). Thus in \(sp\) orbitals, there is half \(s\) character and half \(p\) character; in \(sp^2\) orbitals, there is one-third s character and two-thirds p character, and so on.
\section{Physical properties}
\subsection{Polarity}
The \slch{C-H} bond is considered to be nonpolar as the electronegativities of carbon and hydrogen are extremely close. This means that all hydrocarbons regardless of shape are nonpolar.

Organic compounds with \slch{O} like in \slch{-OH} or \slch{=O} groups, halogen substituents, or \slch{N} like in \slch{-NH2} can be polar.
\subsection{Melting and boiling points}
Hydrocarbons generally have low melting and boiling points, due to their simple molecular structure, where there are only weak intermolecular van der Waals' forces that require a small amount of energy to overcome.

Branched hydrocarbons generally have a lower boiling point than straight-chain isomers as branching gives molecules a more spherical shape, decreasing the surface area of contact, decreasing the extent of distortion of electron clouds, leading to less extensive van der Waals' forces that require less energy to overcome.

In general, \emph{cis} isomers tend to have a higher boiling point than \emph{trans} isomers as \emph{cis} isomers are more likely to be polar. However, the reverse is true for melting point as \emph{trans} isomers tend to pack better than \emph{cis} isomers, so more energy is required to overcome the intermolecular forces in a crystal of a \emph{trans} isomer.

Halogenoalkanes and carbonyl compounds, being polar, will generally have higher melting and boiling points than similar hydrocarbons, as the permanent dipole-permanent dipole forces between molecules of the former two are stronger and require more energy to overcome than the induced dipole-induced dipole forces between hydrocarbon molecules. A carbonyl compound will have a higher boiling point than a similar halogenoalkane.

Halogenoalkanes with larger halogens (in terms of electron cloud size) will have higher boiling points.

Alcohols and amines have higher boiling points than similar hydrocarbons, halogenoalkanes and carbonyl compounds as the hydrogen bonds between alcohol molecules are stronger and require more energy to overcome. Secondary or tertiary alcohols tend to have lower boiling points than primary alcohols as the \slch{-OH} group experiences more steric hindrance in the former two, leading to weaker intermolecular hydrogen bonds. Amines have lower boiling points than similar alcohols as \slch{N-H} is less polar than \slch{O-H}.

Carboxylic acids have the highest boiling points if compared to other organic compounds with similar mass --- except amino acids, as the electron-withdrawing carbonyl group results in an even more polar hydroxyl group, leading to stronger hydrogen bonding; more hydrogen bonds are also formed, so there is more extensive intermolecular hydrogen bonding that requires more energy to overcome.

Amino acids have the highest melting and boiling points compared to other organic compounds, as they exist as zwitterions. A high amount of energy is required to overcome the strong electrostatic attraction between zwitterions. Because of this, amino acids exist as solids at r.t.p. (their melting point is above \SI{200}{\celsius}).
\subsection{Solubility}
Organic compounds that cannot form hydrogen bonds i.e. hydrocarbons and halogenoalkanes are insoluble in water as the hydrogen bonding between water molecules are incompatible with the weak van der Waals' attraction between these organic molecules. These compounds are soluble in organic solvents e.g. \slch{CCl4} as the weak van der Waals' attraction between these organic molecules is compatible with the weak van der Waals' attraction between organic solvent molecules.

Alcohols, carbonyl compounds, carboxylic acids, esters and amides of smaller molecular size are soluble in water and insoluble in organic solvents for the above reasons. However, as the alkyl group becomes bulkier, the solubility in water decreases as the bulkier alkyl group has a greater hydrophobic nature.

Amino acids are soluble in water as the ion-dipole interactions formed result in the release of energy that causes the detachment of zwitterions from the crystal lattice for solvation. \section{Acidity and basicity}
A weak acid is a substance that dissociates partially in solution to give protons.

The strength of an acid depends on its tendency to dissociate, which depends on the stability of the conjugate base. If the negative charge on the anion is dispersed e.g. via intramolecular hydrogen bonding, lone pair delocalisation, or the presence of an electron-withdrawing group, the anion is stabilised, and vice versa.

Electron-withdrawing groups cause greater conjugate base stabilisation if there are more of them, they contain more electronegative atoms, or they are nearer to the acidic group.

Alcohols can behave as very weak acids --- weaker than water. They are so weak because the electron-donating alkyl group intensifies the negative charge on the conjugate base (alkoxide), destabilising the conjugate base.

Ethane-1,2-diol is more acidic than ethanol (and water) as intramolecular hydrogen bonds can be formed between the oxygen in the deprotonated alcohol group and the other alcohol group in the conjugate base. This only applies when the hydroxyl groups are on adjacent carbon atoms.

Phenol is more acidic than water as the phenoxide ion is stabilised by charge delocalisation where the lone pair of electrons on the oxygen atom of phenoxide is delocalised into the benzene ring, reducing the intensity of the negative charge on the oxygen atom of phenoxide, stabilising phenoxide, the conjugate base of phenol.

Carboxylic acids are more acidic than alcohols or phenols as the carboxylate anion is resonance stabilised by the delocalisation of the negative charge over the \slch{C} atom and both \slch{O} atoms. The carboxylate anion is thus more stable than the phenoxide or alkoxide anion, so carboxylic acids ionise more, thus being more acidic.

A base is a substance that can accept a proton. Amines are weak bases in water, as the lone pair on \slch{N} can accept a proton.

The strength on a base depends on the availability of its lone pair, which depends on the electron density of the \slch{N} atom's lone pair. Electron-donating groups will increase the availability of the lone pair on \slch{N}, increasing basicity; electron-withdrawing groups (or a benzene ring, which allows the lone pair to be delocalised into the ring) will decrease the availability of the lone pair on \slch{N}, decreasing basicity. As with acids, the number, distance and strength of substituents have similar effects.
\section{Organic reactions}
All organic compounds can undergo free radical substitution. Organic compounds with \slch{C=C} can undergo electrophilic addition; those with a benzene ring can undergo electrophilic substitution. Organic compounds with electron-withdrawing groups like \slch{-X} or \slch{-OH} can undergo nucleophilic substitution; organic compounds with \slch{=O} (ketones and aldehydes only!) can undergo nucleophilic addition.
\subsection{Electrophilic addition}
The rate of reaction of the electrophilic addition of \slch{X2} and \slch{HX} depends on the bond energies of \slch{H-X} and \slch{C-X}.

Adding \slch{X2} involves breaking \slch{X-X} and the pi bond of \slch{C=C}, and the formation of \slch{C-X}. Down the halogens, the decrease in energy released forming \slch{C-X} is more than the decrease in energy absorbed breaking \slch{X-X}, so reactivity decreases down the group.

Adding \slch{HX} involves breaking \slch{H-X} and the pi bond of \slch{C=C}, and the formation of \slch{C-X} and \slch{C-H}. Down the halogens, the decrease in energy absorbed breaking \slch{H-X} is more than the decrease in energy released forming \slch{C-X}, so reactivity increases down the group.

In an addition where there are two possible products, the major product is the one where the nucleophile formed as an intermediate in the electrophilic addition attaches to the carbon that would form a more stable carbocation. In other words, (\sldef{Markovnikov's rule}) the electrophile attaches to the carbon with more \slch{H} atoms.
\subsection{Elimination}
In general, if more than one alkene can be obtained as a result of elimination, the major product is the alkene which is more substituted i.e. the alkene with the greater number of \slch{R} groups. This is \sldef{Saytzeff's rule}.
\subsection{Electrophilic substitution}
Since the reactivity of the benzene ring is due to the pi electron cloud, substituents affecting the pi electron density will also affect the reactivity of the ring.
Groups with electronegative atoms that cannot delocalise their electrons into the ring are deactivating. They decrease the pi electron density of the ring, making it less reactive and decreasing the rate of reaction of electrophilic substitution. Common deactivating groups include \slch{-NO2}, \slch{-CN}, \slch{-COR}, \slch{-COOR}, \slch{-COOH}, \slch{-CHO}, and the halogens \slch{I}, \slch{Br}, \slch{Cl}, and \slch{F}. This is in ascending order of ring reactivity i.e. descending order of deactivating strength.

Electron-donating groups or groups that have lone pairs that can be delocalised are activating. They increase the pi electron density of the ring, making it more reactive and increasing the rate of reaction. Common activating groups include \slch{-NHCOR}, \slch{-R}, \slch{-OR}, \slch{-OH}, \slch{-NH2}, in similar order as above.

Substituents also have an effect on the position at which further substitutions take place. In general, saturated groups are 2-,4- (ortho, para) directing, while unsaturated groups are 3- (meta) directing.
\subsection{Nucleophilic substitution}
There are two types of nucleophilic substitution --- S\textsubscript{N}2 i.e. bimolecular nucleophilic substitution, and S\textsubscript{N}1 i.e. unimolecular nucleophilic substitution. The former is favoured when there is less steric hindrance around the electron-deficient \slch{C}; the latter is favoured when the electron-deficient \slch{C} forms a stable carbocation.

The S\textsubscript{N}2 mechanism follows second order kinetics, and it is a one-step mechanism. Due to the nucleophile attacking from the backside of the electron-deficient \slch{C}, products formed by this mechanism always have their optical configuration inverted compared to the original reactant. S\textsubscript{N}2 is most favourable and has the highest rate for primary halogenoalkanes or alcohols, and becomes less favourable the more carbon neighbours the electron-deficient \slch{C} has, due to steric hindrance.

The S\textsubscript{N}1 mechanism follows first order kinetics, and it is a two-step mechanism. The rate equation involves only the compound being attacked. In the first step of the mechanism, a carbocation intermediate is formed, which is trigonal planar. In the second step, the nucleophile can attack from either the top or the bottom of the carbocation, and there is equal probability of either. Thus, a racemic mixture is formed when the attacked carbon is chiral. S\textsubscript{N}1 is most favourable and has the highest rate for tertiary compounds, and becomes less favourable the less carbon neighbours the electron-deficient \slch{C} has, as the carbocation intermediate formed is less stable.

The rate of reaction of nucleophilic substitution on halogenoalkanes depends on the ease of breaking the \slch{C-X} bond. It follows that the rate of reaction for \(\slch{R-I} > \slch{R-Br} > \slch{R-Cl} > \slch{R-F}\).

A halogenoarene or phenol generally does not undergo nucleophilic substitution as one lone electron pair on the halogen or oxygen atom is delocalised into the benzene ring, strengthening the carbon-halogen bond due to partial double bond character, preventing nucleophilic substitution under normal conditions.

Similarly, a halogenoalkene (\slch{RCH=CXR'}) will not undergo nucleophilic substitution under normal conditions as one lone electron pair of the halogen atom is delocalised with the adjacent \slch{C=C}, strengthening the \slch{C-X} bond due to partial double bond character, preventing nucleophilic substitution under normal conditions.

In an acyl chloride, the carbonyl \slch{C} atom has 2 very electronegative atoms (\slch{O} and \slch{Cl}) bonded to it, so it has a large partial positive charge. It is thus highly electron deficient and will undergo nucleophilic substitution readily; this causes acyl chlorides to be extremely reactive.
\subsection{Nucleophilic addition}
In the nucleophilic addition of \slch{HCN} to a carbonyl compound, \slch{HCN} acts as an acid in the second step to donate a proton to form the final product. \slch{HCN} also generates the \slch{CN-} nucleophile.

The slow first step involves the \slch{CN-} nucleophile, but \slch{HCN} is a weak acid and so is a poor source of \slch{CN-}. Trace \slch{NaOH} or \slch{NaCN} is added to remedy this. \slch{NaOH} will neutralise \slch{H+} in the reaction mixture, decreasing \conc{H+}, and by Le Chatelier's principle, the equilibrium position in the dissociation of \slch{HCN} shifts right to increase \conc{H+}, increasing \conc{CN-} and the rate of reaction. \slch{NaCN} will fully dissociate to produce the initial \slch{CN-} for the nucleophilic attack on the electron-deficient \slch{C} atom, acting as a homogenous catalyst as \slch{CN-} is regenerated in the second step.

The temperature of the reaction is kept to \SI{10}{\celsius} to \SI{20}{\celsius} to prevent poisonous \slch{HCN} gas from escaping to the environment. It is not brought lower than this to ensure a reasonable rate of reaction.

If the addition creates a chiral \slch{C}, then a racemic mixture will be formed as there is equal probability of \slch{CN-} attacking the trigonal planar electron-deficient carbonyl \slch{C} atom from either the top or bottom of the plane.

Ketones are usually less reactive than aldehydes as ketones have two alkyl groups attached to the carbonyl \slch{C} while aldehydes only have one, resulting in steric hindrance; it is more difficult for the nucleophile to attack the electron-deficient carbonyl \slch{C} atom in a ketone than in an aldehyde. The additional electron-donating R-group in ketones causes the partial positive charge on ketones' carbonyl \slch{C} to be less than that on aldehydes, so ketones' carbonyl \slch{C} is less susceptible to nucleophilic attack.
\section{Uses and environmental concerns}
Alkanes are generally used as fuels: petrol is a mixture of \slch{C5} to \slch{C10} alkanes. In car engines, petrol vapours are ignited in air causing an explosive reaction, driving the pistons of the engine.

When the combustion of fuel is not smooth, knocking occurs. Knocking reduces engine power, leads to wear and tear in the engine, and also wastes petrol. To reduce knocking, an antiknock agent, tetraethyllead(IV) i.e. \slch{Pb(C2H5)4} is added to petrol, which becomes leaded petrol. The weak \slch{Pb-C} bonds are easily broken, resulting in ethyl radicals that initiate smooth burning. However, this forms \slch{PbO} which coats car cylinders. To prevent this, \slch{CH2BrCH2Br} is added to remove the lead content as volatile \slch{PbBr2} which passes out as exhaust.

The use of petrol as fuels produces many pollutants. The incomplete combustion of fuel releases \slch{CO} as well as unburnt hydrocarbons. \slch{CO} causes carbon monoxide poisoning as it combines with haemoglobin in the blood, forming carboxyhaemoglobin, which prevents the transportation of \slch{O2} to all parts of the body. Unburnt hydrocarbons become part of photochemical smog in strong sunlight, causing lung damage. The use of TEL releases \slch{PbBr2} vapour, which can lead to lead poisoning, causing brain damage. Oxides of nitrogen are produced due to the reaction of \slch{N2} with \slch{O2} at high temperatures and pressures in the car engine. They form acid rain that corrodes buildings and destroys marine life, as well as causing respiratory problems in humans and interfering with nitrogen metabolism in plants. \slch{SO2} produced due to trace sulfur compounds in fuels creates acidic gases that also form acid rain.

Catalytic converters, which are wire meshes coated with rhodium, platinum and palladium, reduce \slch{NO_x} to \slch{N2} and oxidise \slch{CO} to \slch{CO2}. \begin{equation}\ch{2 NO~(g) + 2 CO~(g) ->[Rh] 2 CO2~(g) + N2~(g)}\end{equation} Unburnt hydrocarbons and \slch{CO} are oxidised to \slch{CO2} and \slch{H2O}. \begin{align}\ch{2 CO~(g) &+ O2~(g) ->[ 'Pt,~Pd' ] 2 CO2~(g)}\\\begin{split}\ch{C_xH_y~(g) &+ (x + y/4) O2~(g) \\&->[ 'Pt,~Pd' ] \(x\) CO2~(g) + \((y/2)\) H2O~(g)}\end{split}\end{align}

Fluoroalkanes and fluorohalogenoalkanes are stable and unreactive due to the strong \slch{C-F} bond. They are good solvents with low boiling points and are non-flammable, non-toxic and odourless. Due to their lack of reactivity, they are useful as inert materials in fire extinguishers, refrigerants, and aerosol propellants. Their boiling points just below room temperature make them easy to liquefy by a slight increase in pressure, and conversely easy to vaporise by a slight decrease in pressure. This makes them useful as liquid refrigerants, and also aerosol propellants.

Chlorofluorocarbons, also known as CFCs, have been widely used as aerosols and refrigerants. However, due to their inertness, they tend to drift up into the stratosphere, where they undergo homolytic fission to form chlorine radicals that destroy the ozone layer. \begin{equation}\begin{split}\ch{CF2Cl2 &->[hv] ^.CF2Cl + Cl^.}\\\ch{Cl^. + O3 &->[hv] O2 + ClO^.}\\\ch{ClO^. + O^. &-> O2 + Cl^.}\end{split}\end{equation} Note that \slch{Cl^.} is regenerated. In general, only fluoroalkanes are safe for use as the \slch{C-F} bond will not break as easily as the other halogens'.
\section{Amino acids and proteins}
\subsection{Amino acid properties}
Amino acids exist as electrically neutral dipolar ions called zwitterions, formed when the carboxylic acid group donates its proton to the amino group.

The pH at which an amino acid exists as a zwitterion is known as its isoelectric point. Below this pH, amino acids exist as cations; above this pH, amino acids exist as anions.

Electrophoresis can be used to separate a mixture of amino acids by using an electric field. Depending on the pH of the medium, different amino acids will move towards the cathode or anode; the distance of movement depends on the \(q/m\) ratio.

Amino acids have different types depending on the number of \slch{C} atoms between \slch{-NH2} and \slch{-COOH}. If there is only one \slch{C} atom, it is an \mupalpha-amino acid; if there are two, it is a \mupbeta-amino acid; and so on.

\mupalpha-amino acids have general formula \slch{RCH(NH2)COOH}.

The properties of amino acids, other than those caused by the \mupalpha-\slch{NH2} and \mupalpha-\slch{COOH}, depend on the R-groups of each amino acid. Amino acids can have R-groups that are one of nonpolar i.e. hydrophobic, uncharged and hydrophilic, negatively charged, or positively charged.
\subsection{Proteins and basic properties}
Proteins are polymers consisting of long peptide chains of amino acid residues. Peptide linkages are formed by the condensation of amino acids by removing a \slch{H2O} molecue between the \slch{-COOH} and \slch{-NH2} groups of separate amino acid molecules.

Proteins have important functions in living systems. They act as \begin{slinenum}
\item structural proteins, defining shapes and sizes of cells
\item muscle fibres, providing mechanical force
\item transport proteins, moving metabolites
\item hormones, controlling cell activity
\item enzymes, catalysing metabolic processes.\end{slinenum}

Since peptide bonds are simply amide linkages, they can be hydrolysed similarly. They can also be broken in the presence of suitable enzymes.
\subsection{Protein structures}
\subsubsection{Primary and secondary structure}
The primary structure of proteins refers to the sequence of amino acids in polypeptide chains. By the nature of interactions between different amino acid residues, the primary structure determines the folding, shape, nature and function of the protein.

By convention, amino acid sequences are written with the \slch{N}-terminus first (on the left).

The secondary structure of proteins refers to regular arrangements of the polypeptide chain stabilised by hydrogen bonds between an \slch{O} of an amide group and a \slch{H} of another on the same chain.

The \mupalpha-helix is a regular coiled configuration of the polypeptide chain, held in place by intra-chain hydrogen bonds. The \slch{O} atom in the amide group of each amino acid is hydrogen-bonded to the \slch{H} atom in the amide group of the fourth amino acid further down the chain, and there are \num{3.6} amino acids per helical turn. The R-groups on the \mupalpha-carbon point outside of the helix and are perpendicular to the main axis of the helix. The formation of many hydrogen bonds within the helix gives a strong total binding effect, causing the helix to be flexible and elastic.

The \mupbeta-pleated sheet consists of sections of polypeptide aligned side-by-side linked by intra-chain hydrogen bonding, with R-groups projected above or below the sheet. Hydrogen bonding between sections of the polypeptide results in a very stable structure. The \mupbeta-pleated sheet is flexible but inelastic, and can either run parallel or antiparallel.
\subsubsection{Tertiary structure}
The tertiary structure of proteins is the overall three-dimensional shape of a protein that is formed when a polypeptide chain folds extensively to form a complex rigid three-dimensional structure. The tertiary structure is held together by R-group interactions, depending on the nature of the group.

Non-polar or hydrophobic groups interact via van der Waals' attraction with other non-polar groups only.

Polar or hydrophilic groups interact via permanent dipole-permanent dipole attraction with other polar groups, and hydrogen bonding with groups containing \slch{O-H}, \slch{C=O} or \slch{N-H}.

Groups containing the thiol group (\slch{S-H} can form disulfide bonds with other such groups i.e. cysteine residues), forming a cystine. \begin{multline}\ch{2 SHCH2CH(NH2)COOH + [O] <=> \\NH2CH(COOH)CH2-S-S-CH2CH(COOH)NH2 \\+ H2O}\end{multline}

Ionic groups can form ionic bonds with oppositely charged groups, and hydrogen bonds with groups containing \slch{O-H}, \slch{C=O} or \slch{N-H}.

Globular proteins have distinctive tertiary structures where almost all hydrophilic R-groups are on the outer surface, half of more of the hydrophobic R-groups are internally located, and the folding is compact with little or no room for water molecules in the interior. This allows them to be soluble in water.
\subsubsection{Quaternary structure}
The quaternary structure is the combination of two or more individually folded polypeptide chains interacting to form a complex structure. Individual polypeptide chains exist in their own tertiary structures and are called subunits. Subunits in the quaternary structure are held together by van der Waals' interactions, ionic bonds and hydrogen bonds. Proteins where two or more subunits are identical polypeptide chains are said to be oligomeric.

Haemoglobin is a protein with a quaternary structure consisting of 4 subunits --- 2 \mupalpha-subunits and 2 \mupbeta-subunits, each subunit being noncovalently linked to a haem group. Each haem group contains a \slch{Fe^{2+}} ion that binds reversibly to \slch{O2}. \ch{Hb + 4 O2 <=> Hb(O2)4}. Both the \mupalpha- and \mupbeta-subunits have 70\% \mupalpha-helical regions. There is little contact between two \mupalpha-subunits and two \mupbeta-subunits, but there is a lot of contact between \mupalpha-subunits and their neighbouring \mupbeta-subunits.
\subsection{Denaturation}
Denaturation refers to the breaking of weak bonds holding the secondary, tertiary and quaternary structures (but not the covalent bonds within the primary structures). The loss of shape of the protein causes the loss of protein function. Proteins can be denatured by heating, pH change, mechanical agitation, and addition of metal ions.

Heating and mechanical agitation disrupts the weaker bonds like van der Waals' forces that stabilise the tertiary and quaternary structures, and the hydrogen bonds that stabilise the secondary structure.

pH changes protonate or deprotonate acidic and basic R-groups. At low pH, excess \slch{H+} protonates basic R-groups like \slch{-COO^-} and \slch{-NH2}, forming \slch{-COOH} and \slch{-NH3+} respectively, disrupting ionic bonds and hydrogen bonds respectively. At high pH, excess \slch{OH^-} deprotonates acidic R-groups like \slch{-COOH} and \slch{-NH3+}, forming \slch{-COO^-} and \slch{-NH2} respectively, disrupting hydrogen bonds and ionic bonds respectively. Changes in pH thus disrupt the ionic bonds and hydrogen bonds between basic and acidic R-groups, which hold the secondary, tertiary and quaternary structure of proteins together, leading to changes in protein shape and loss of protein function.

Metal ions like \slch{Cu^{2+}}, \slch{Ag+} and \slch{Hg+} disrupt the ionic interactions between charged ionic R-groups by forming ionic bonds with these groups. Metal ions can interact with anionic R-groups to form insoluble salts. They can also disrupt the formation of disulfide bonds as they have high affinity for sulfur in the thiol group, leading to the formation of precipitates. \begin{gather}\ch{RCOO^- + Ag+ <=> RCOO^-Ag+}\\\ch{RSH + Ag+ <=> RS^-Ag+ + H+}\end{gather} These effects disrupt the tertiary and quaternary structures of proteins.
\end{document}
