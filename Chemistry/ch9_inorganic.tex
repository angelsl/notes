\documentclass[Chemistry.tex]{subfiles}
\begin{document}
\chapter{Inorganic Chemistry}
\section{Chemical Periodicity}
The \sldef{periodic table} is an arrangement of elements in order of increasing atomic number. Generally, the properties of elements are a periodic function of the atomic number.

A \sldef{group} refers to a vertical column in the table. The traditional group numbers (i.e. group I, II, etc.) show the number of valence electrons an atom of each element has. As a result, elements within the same group generally have similar chemical properties.

A \sldef{period} refers to a row in the table. The period number refers to the number of electron shells an atom of each element possesses.
\subsection{Trends of period 3 elements}
Atomic radius tends to decrease across periods, as \begin{slinenum}
\item across a single period, electrons are added to the same valence shell
\item the increase in nuclear charge outweighs the negligible increase in shielding effect, so effective nuclear charge increases
\item thus there is stronger electrostatic attraction between the nucleus and valence electrons, so valence electrons are closer to the nucleus.
\end{slinenum}

Atomic and ionic radius tends to increase down the group, as \begin{slinenum}
\item down the group, there is an increasing number of quantum electron shells
\item valence electrons are further away from the nucleus and experience more shielding by inner shells
\item thus there is weaker electrostatic attraction between the nucleus and valence electrons, so valence electrons are further from the nucleus.
\end{slinenum}

For an element that forms a cation, its ionic radius is always smaller than its atomic radius, as \begin{slinenum}
\item the cation has one less electron shell occupied than the neutral atom
\item the nuclear charge of the nucleus stays the same, as the number of protons does not change
\item thus there is stronger electrostatic attraction between the nucleus and the cation's valence electrons, so the cation's valence electrons are closer to the nucleus.
\end{slinenum}

For an element that forms an anion, its ionic radius is always larger than its atomic radius, as \begin{slinenum}
\item the anion has more electrons in its valence shell than in the neutral atom, so the extent of repulsion between valence electrons is greater
\item the nuclear charge of the anion stays the same, as the number of protons does not change
\item thus there is weaker electrostatic attraction between the nucleus and the anion's valence electrons, so the anion's valence electrons are further from the nucleus.
\end{slinenum}

In an isoelectronic series of ions, the ionic radius decreases as the atomic number increases, as \begin{slinenum}
\item the cations\slash anions are isoelectronic but
\item the nuclear charge increases while screening effect is approximately constant as they have the same number of quantum electron shells
\item so the same number of electrons thus experiences stronger electrostatic attraction to the more positively charged nucleus, thus the valence electrons are closer to the nucleus.
\end{slinenum} There is generally a sharp increase in ionic radius from group IV elements to group V elements as the anions have one more quantum electron shell occupied.

Electronegativity tends to increase across periods as \begin{slinenum}
\item the increase in nuclear charge outweighs the negligible change in screening effect, so effective nuclear charge increases across the period
\item there is stronger electrostatic attraction between the nucleus and valence electrons, thus electron-attracting power of the atom increases.
\end{slinenum}

Electronegativity tends to decrease down groups as \begin{slinenum}
\item there is an increasing number of quantum electron shells, so the valence electrons are further away from the nucleus and better shielded by inner quantum electron shells
\item there is weaker electrostatic attraction between the nucleus and valence electrons, so electron-attracting power of the atom decreases.
\end{slinenum}

Melting point generally increases from group I to III, peaks at group IV, and then decreases, as \begin{slinenum}
\item elements in group I to III have giant metallic structures; a large amount of energy is required to overcome the strong electrostatic attraction between the cations and the sea of delocalised electrons, leading to a high melting point
\item the element in group IV has a giant molecular structure; a very large amount of energy is required to overcome the strong and extensive covalent bonding between atoms in a giant three-dimensional structure, leading to a very high melting point
\item elements in group V to 0 have simple molecular structures; a small amount of energy is required to overcome the weak van der Waals' attraction between molecules (or atoms, for group 0), leading to a low melting point.
\end{slinenum}

Melting point generally increases from group I to III as \begin{slinenum}
\item the number of valence electrons contributed per atom increases
\item the ionic radius decreases while the ionic charge increases, so ionic charge density increases
\item so there is stronger electrostatic attraction between the metal cations and the sea of delocalised electrons requiring more energy to overcome, leading to a higher melting point.
\end{slinenum}

Melting point generally decreases from group V to 0 in period 3 (\slch{P} to \slch{Ar}) as \begin{slinenum}
\item the strength of van der Waals' attraction is proportional to the size of the electron cloud
\item from \slch{P} to \slch{Ar}, the size of the element's electron cloud decreases (\slch{S8} to \slch{P4} to \slch{Cl2} to \slch{Ar}), so the extent of intermolecular van der Waals' attraction decreases
\item less energy is required to overcome the less extensive intermolecular van der Waals' attraction, so melting point decreases.
\end{slinenum}

Electrical conductivity generally increases from group I to III and then decreases from group IV onwards as \begin{slinenum}
\item from group I to III, the elements are metals, which are good conductors of electricity due to the sea of delocalised electrons present in their giant metallic lattice structure acting as charge carriers. Electrical conductivity increases as more valence electrons are contributed per atom from group I to III
\item at group IV, the conductivity drops as metalloids are poor conductors of electricity
\item group V onward elements do not conduct electricity as there are no free mobile ions or delocalised electrons in their simple molecular structures to act as charge carriers.
\end{slinenum}

Period 3 elements generally form oxides with oxidation states equal to their respective group numbers. However, \slch{P}, \slch{S}, and \slch{Cl} can form oxides with oxidation number +3 (\slch{P4O6}), +4 (\slch{SO2}), and +1 (\slch{Cl2O}) respectively.

Elements up to group V in period 3 generally form chlorides with oxidation states equal to their respective group numbers. However, \slch{P} can form a chloride with oxidation state +3 (\slch{PCl3}), and \slch{S} forms oxides with oxidation states +1 (\slch{S2Cl2}) and +2 (\slch{SCl2}). Chlorine forms a rather obvious chloride.

Elements in period 3 form typical hydrides (for \slch{Na} to \slch{Al}) or hydrogen compounds (for the rest).

Generally, elements can attain up to oxidation states with the same numerical value as their group number.

Some elements have similar physical and chemical properties with those diagonally next to it. When this occurs, this is known as a \sldef{diagonal relationship}. Diagonal relationships occur due to similar charge densities.

Examples of diagonal relationships include \begin{slinenum}
\item \slch{LiNO3} decomposing in the same way as \slch{Mg(NO3)2}
\item \slch{Be} and \slch{Al} both dissolving in alkali; \slch{BeCl2} and \slch{AlCl3} both being covalent
\item \slch{Be(OH)2} and \slch{Al(OH)3}, and \slch{BeO} and \slch{Al2O3}, all being amphoteric
\item \slch{Be} and \slch{Al} both reacting with alkalis, liberating hydrogen.
\end{slinenum} \begin{align}\begin{split}\ch{2 Al\solid{} &+ 6 H2O\lqd{} + 2 OH^-\aq{} \\&-> 2 Al(OH)4^-\aq{} + 3 H2\gas{}}\end{split}\\\begin{split}\ch{Be\solid{} &+ 2 H2O\lqd{} + 2 OH^-\aq{} \\&-> Be(OH)4^-\aq{} + H2\gas{}}\end{split}\end{align}
\subsection{Trends of period 3 oxides}
The melting point of period 3 oxides increases from \slch{Na2O} to \slch{MgO} and decreases thereafter, as \begin{slinenum}
\item \slch{Na2O}, \slch{MgO} and \slch{Al2O3} have giant ionic lattice structures; a large amount of energy is required to overcome strong electrostatic forces of attraction between the oppositely charged ions, leading to a high melting point
\item \slch{SiO2} has a giant molecular structure; a large amount of energy is required to overcome the strong and extensive covalent bonding between atoms in a giant 3-dimensional molecular structure, leading to a high melting point
\item \slch{P4O10}, \slch{SO2}, \slch{SO3} and \slch{Cl2O7} have simple molecular structures; a small amount of energy is required to overcome the weak van der Waals' attraction between simple molecules, leading to a low melting point.
\end{slinenum}

The melting point of \slch{Na2O} is lower than that of \slch{MgO} because the lattice energy of \slch{MgO} is higher.

The melting point of \slch{MgO} is higher than that of \slch{Al2O3} because \slch{Al^{3+}} has a very high charge density, so it is able to polarise the anion \slch{O^{2-}}'s electron cloud to a large extent, inducing partial covalent character, so \slch{Al2O3} has slightly weaker ionic bonds than \slch{MgO} leading to \slch{MgO} having a higher melting point.
\subsection{Trends of period 3 chlorides}
The melting point of period 3 chlorides decreases from \slch{NaCl} to \slch{SiCl4}, increases to \slch{PCl5}, and decreases to \slch{S2Cl2}, as \begin{slinenum}
\item \slch{NaCl} and \slch{MgCl2} both have giant ionic lattice structures; they need a large amount of energy to overcome the strong electrostatic attraction between oppositely charged ions, thus a high melting point
\item \slch{Al2Cl6}, \slch{SiCl4}, \slch{PCl5}, \slch{S2Cl2}, and \slch{Cl2} all have simple molecular structures; they require only a small amount of energy to overcome weak van der Waals' attraction and so have low melting points. (At r.t.p., \slch{PCl5} exists as an ionic solid of \slch{PCl4^{+}PCl6^{-}}, leading to its higher than expected melting point.)
\end{slinenum}

\slch{MgCl2} has a lower melting point than expected it because of covalent character due to \slch{Mg^{2+}}'s high charge density.
\section{Group II}
Group II elements are the alkali earth metals. They are silvery-white, soft, and easily oxidised in air.

As with all other groups, down the group, their atomic and ionic radii increase, and their first ionisation energy, electronegativity and melting point decrease.

Group II metals are very good reducing agents, and their reducing power, and thus reactivity, increases down the group, as down the group, \begin{slinenum}
\item the ionisation energy decreases thus valence electrons are more easily lost
\item they have a greater tendency to be oxidised, thus they are stronger reducing agents.
\end{slinenum}

All group II metals burn with a bright flame to form basic oxides (except the amphoteric \slch{BeO}). When burnt in oxygen, they produce flames of colours as in table~\ref{tb:a9.gii}.

Group II metals generally react with water to produce hydroxides and \slch{H2} gas, as in table~\ref{tb:a9.gii}.

The hydroxides of \slch{Ca} and \slch{Sr} are sparingly soluble in water, while that of \slch{Ba} is soluble in water. In general, the solubility of the hydroxides, and thus base strength, of group II metals increases down the group.

Group II oxides are basic except for \slch{BeO} which is amphoteric. \slch{BeO} behaves like \slch{Al2O3} when reacting with bases, and forms the \slch{Be(OH)4^{2-}} ion. The rest of the oxides are basic and react as per normal, as in table~\ref{tb:a9.gii}.

Group II carbonates, nitrates, and hydroxides are thermally unstable and decompose on heat to form oxides. \begin{align}\ch{MCO3\solid{} &-> MO\solid{} + CO2\gas{}}\\\begin{split}\ch{2 M(NO3)2\solid{} &-> 2 MO\solid{} \\&+ 4 NO2\gas{} + O2\gas{}}\end{split}\\\ch{M(OH)2\solid{} &-> MO\solid{} + H2O\lqd{}}\end{align}

The thermal stability of group II nitrates, carbonates and hydroxides increases down the group, as \begin{slinenum}
\item down the group, ionic size of cation increases
\item so charge density and polarising power of cation decreases
\item the ability of the cation to distort the anion charge cloud and break the relevant bond decreases
\item thus thermal stability increases and decomposition temperature increases.\end{slinenum}

Group II metals have various uses. \begin{itemize}
\item \slch{MgO} is used as a refractory for furnace linings, and as an antacid.
\item \slch{CaO}, i.e. quicklime, is used as a drying agent and to reduce the acidity of soil.
\item \slch{CaCO3}, i.e. limestone, is used as building materials and in the manufacture of \slch{CaO}.
\item \slch{BaSO4} is given to patients suffering from digestive disorders as it can be traced through the body by X-ray.
\end{itemize}
\section{Group VII}
Group VII elements are the halogens. All halogens exist as diatomic molecules \ch{X2}, are nonpolar, and have simple molecular structures.

Volatility of the halogens decreases down the group, as down the group, \begin{slinenum}
\item size of electron cloud increases
\item there is a greater extent of distortion of the electron cloud
\item more energy is required to overcome the more extensive intermolecular van der Waals' forces
\item boiling and melting points increase
\item thus volatility decreases.
\end{slinenum}

Intensity of the halogens' colour increases down the group.

Halogens are soluble in nonpolar solvents like \slch{CCl4} as \begin{slinenum}
\item both halogens and nonpolar solvents have simple molecular structures
\item the weak van der Waals' forces between halogen molecules are compatible with the weak van der Waals' forces between nonpolar solvent molecules.
\end{slinenum}

Halogens are only slightly soluble in water as the weak van der Waals' forces between halogen molecules are not compatible with the stronger hydrogen bonds between water molecules.

\slch{I2}'s solubility is increased in a solution of \slch{I^-} in water as the brown \ch{I3^-} complex is formed: \begin{align}\ch{I2\solid{} &<=> I2\aq{}}\\\ch{I2\aq{} + I^-\aq{} &<=> I3^-\aq{}}\end{align}

\slch{Cl2}, \slch{Br2} and \slch{I2} exhibit variable oxidation states as they all have empty and energetically accessible \(d\) orbitals enabling them to expand their octet configuration. \slch{F2} does not as it has no energetically accessible and vacant \(d\) orbital to expand its octet configuration.

Oxidising power and reactivity of the halogens decreases down the group as down the group, \begin{slinenum}
\item the \slEo{} values decrease
\item there is a decrease in tendency to accept electrons and be reduced to \ch{X^-}
\item so oxidising power decreases.
\end{slinenum}

It follows that halogens higher up in the group can displace the halide ions of halogens lower in the group out of their aqueous solutions e.g. \[\ch{Cl2 + 2 Br^- -> Br2 + 2 Cl^-}\]

This can be used to detect the presence of \slch{Br^-} and \slch{I^-} ions in solution, by shaking an unknown with \slch{Cl2\aq{}} then adding an organic solvent like \slch{CCl4}. Chlorine will displace the halide forming its element, and the addition of the organic solvent will cause most of the displaced halogen to dissolve in it, with a bit left in the water. Since water and organic solvents are immiscible, two layers of different colours will be formed, and the identity of the displaced halogen can be deduced.

The reactivity of halogens with hydrogen decreases down the group as down the group, \begin{slinenum}
\item the total bond energy released in forming \slch{H-X} decreases more significantly than the total bond energy absorbed in breaking \slch{X-X} and \slch{H-H}
\item thus \(\Delta H_\text{rxn}\) becomes less exothermic and reactivity decreases.
\end{slinenum}

All halogens oxidise \slch{S2O3^{2-}} but to varying degrees; \slch{Cl2} and \slch{Br2} oxidise it to \slch{SO4^{2-}} while \slch{I2} only oxidises it to \slch{S4O6^{2-}}. \slch{Cl2} and \slch{Br2}, being stronger oxidising agents, can oxidise \slch{S2O3^{2-}} all the way to \slch{SO4^{2-}}, increasing the oxidation state of \slch{S} from +2 to +6 respectively. However, \slch{I2}, being a weaker oxidising agent, can only oxidise \slch{S2O3^{2-}} to \slch{S4O6^{2-}}, increasing the oxidation state of \slch{S} from +2 to +2.5 only.
\subsection{Trends of halides}
\slch{HCl}, \slch{HBr} and \slch{HI} are colourless gases with simple molecular structures.

The melting and boiling points of hydrogen halides increases down the group due to the increase in size of electron cloud \ldots{}.

The thermal stability of hydrogen halides decreases down the group as down the group, \begin{slinenum}
\item covalent bond length of \slch{H-X} increases
\item covalent bond strength decreases
\item bond dissociation energy decreases.
\end{slinenum}

The acid strength of hydrogen halides increases down the group as down the group, \begin{slinenum}
\item covalent bond strength of \slch{H-X} decreases
\item bond dissociation energy decreases
\item ease of breaking \slch{H-X} increases
\item so \slch{H3O^+} and \slch{X^-} can be formed more easily.
\end{slinenum}

The reducing power of halides increases down the group as their \slEo{} decreases down the group.
\subsubsection{Solubility of silver halides}
An unknown halide ion can (other than being shaken with \slch{Cl2}) be identified by examining the solubility of the silver halide precipitate in dilute and concentrated \slch{NH3}.

\slch{AgF} is very soluble in pure water, while the other silver halides form precipitates.

\slch{AgCl} is soluble in excess dilute ammonia, because \begin{equation}\label{eq:9.g7.agx}\ch{Ag^+\aq{} + X^-\aq{} <=> AgX\solid{}}\end{equation}\begin{slinenum}
\item its \(K_\text{sp}\) is relatively higher
\item so when \slch{NH3\aq{}} is added, it combines with \slch{Ag^+\aq{}} to form the colourless and soluble \slch{[Ag(NH3)2]^+} complex
\item by Le Chatelier's principle, the position of equilibrium~\eqref{eq:9.g7.agx} shifts left
\item ionic product of \slch{AgCl} decreases to a value lower than its \(K_\text{sp}\) since the latter is relatively high
\item thus \slch{AgCl} precipitate dissolves.
\end{slinenum}

\slch{AgBr} and \slch{AgI} are insoluble in excess dilute ammonia as their \(K_\text{sp}\) are relatively low and their ionic products easily exceed their \(K_\text{sp}\) values.

When a stronger ligand like cyanide is added, \slch{AgBr} will dissolve, as \begin{slinenum}
\item \slch{CN^-} is a stronger ligand than \slch{NH3} and will form a more stable complex \slch{[Ag(CN)2]^-}
\item by Le Chatelier's principle, the equilibrium position~\eqref{eq:9.g7.agx} shifts right
\item the ionic product of \slch{AgBr} decreases below its \(K_\text{sp}\)
\item thus \slch{AgBr} dissolves.
\end{slinenum}

\slch{AgCl} and \slch{AgBr} are soluble in excess concentrated ammonia while \slch{AgI} is still insoluble.
\section{Transition Elements}
A \sldef{transition element} is a d-block element that forms at least one stable ion with partially filled \(d\) orbitals.

Based on the definition, \slch{Sc} and \slch{Zn} are not transition elements as the former forms only \slch{Sc^{3+}} which has no \(d\) electron, while the latter forms only \slch{Zn^{2+}} which has a completely filled \(d\) subshell.
\subsection{Physical properties}
Across the transition elements, atomic radius decreases slightly, as across the period, \begin{slinenum}
\item nuclear charge increases but electrons are added to inner \(3d\) orbitals and thus provide more effective shielding of the \(4s\) electrons
\item so the increase in nuclear charge is almost cancelled by the increase in shielding effect
\item effective nuclear charge increases slightly
\item thus atomic and ionic radii decrease slightly.
\end{slinenum}

This is contrasted to the period 2 and 3 elements, where across the period, \begin{slinenum}
\item electrons are added to the same outermost quantum shell
\item the increase in nuclear charge outweighs the negligible increase in shielding effect
\item effective nuclear charge increases significantly
\item thus atomic and ionic radii decrease significantly
\end{slinenum}

Transition metals are harder and denser than non-transition metals, as transition metals \begin{slinenum}
\item have a relatively smaller atomic radius, and thus a more closely-packed structure
\item they also have higher relative atomic mass
\item so atoms are attracted closer together, leading to higher mass per unit volume i.e higher density.
\end{slinenum}

Across the transition metals, \slnIE{1} and \slnIE{2} increase slightly, as \begin{slinenum}
\item \slnIE{1} and \slnIE{2} involve the removal of \(4s\) electrons
\item across the period, the added inner 3d electrons provide effective shielding of outer \(4s\) electrons
\item effective nuclear charge increases slightly
\item thus there are only small increases in \slnIE{1} and \slnIE{2}.
\end{slinenum}

Across the transition metals, \slnIE{3} and \slnIE{4} increase rapidly, as \begin{slinenum}
\item \slnIE{3} and \slnIE{4} involve the removal of \(3d\) electrons
\item the remaining \(d\) electrons provide a relatively poor shielding effect
\item across the period, there is a significant increase in effective nuclear charge
\item thus there is an rapid increase in \slnIE{3} and \slnIE{4}.
\end{slinenum}

The \slnIE{1}s of \slch{Cr} and \slch{Cu} are slightly lower than expected, as the removal of an outer electron results in the attainment of a stable half-filed \(3d^5\) or fully-filled \(3d^{10}\) configuration, which is a favourable process, so less energy is required.

The \slnIE{2}s of \slch{Cr} and \slch{Cu} are slightly higher than expected, as the removal of an outer electron results in the disruption of the stable half-filed \(3d^5\) or fully-filled \(3d^{10}\) configuration, which is unfavourable, so more energy is required.

The \slnIE{3} of \slch{Fe} is lower than expected, as inter-electron repulsion is predominant between paired \(d\) electrons in the doubly-filled d orbital, resulting in less energy needed to remove a valence electron from \slch{Fe^{2+}}.

Transition metals have higher melting points and boiling points than s-block metals, as \begin{slinenum}
\item they have giant metallic structures with stronger metallic bonds as both \(3d\) and \(4s\) electrons are involved in delocalisation
\item melting and boiling involves overcoming the stronger electrostatic forces of attraction between the cations and the sea of delocalised electrons
\item more energy is thus required
\item so they have higher melting and boiling points than typical metals.
\end{slinenum}

For s-block metals, only \(s\) electrons contribute to delocalisation in metallic bonding, resulting in weaker metallic bonds.

Transition elements possess good mechanical qualities like high tensile strength, malleability and ductility, as the layers of closely-packed atoms easily slip over one another without disrupting the electrostatic attraction between cations and delocalised electrons.

Transition metals are better thermal and electrical conductors than main group metals, as both \(3d\) and \(4s\) electrons are available for delocalisation, so there are more of mobile electrons to act as charge carriers.
\subsection{Chemical properties}
Transition elements possess variable oxidation states as the \(3d\) and \(4s\) orbitals are close in energy, so a variable number of \(3d\) and \(4s\) electrons can be removed, forming ions of similar stability.

The highest oxidation state a transition metal can take is equal to the number of \(4s\) electrons plus the number of unpaired \(3d\) electrons.

Negative \slEo{} values for \slch{Ti}, \slch{V} and \slch{Cr} indicate that reduction is less feasible, so \slch{M^{3+}\aq{}} is more stable than \slch{M^{2+}\aq{}}. \slch{M^{2+}\aq{}} would be easily oxidised, so \slch{M^{2+}} is a good reducing agent.

Positive \slEo{} values for \slch{Mn} to \slch{Cu} indicate that reduction is more feasible, so \slch{M^{2+}\aq{}} is more stable than \slch{M^{3+}\aq{}}. \slch{M^{3+}\aq{}} would be easily reduced, so \slch{M^{3+}\aq{}} is a good oxidising agent.

\slch{Mn^{2+}} is more stable than \slch{Mn^{3+}} as the removal of a third electron from \slch{Mn} is more difficult due to disruption of the stable half-filled \(3d^5\) configuration, so oxidation is less likely to occur and \slEo{} is more positive than expected.

Another anomaly is \slch{Fe^{3+}} being more stable than \slch{Fe^{2+}}, which occurs as the removal of third electron is easier due to inter-electron repulsion between \(d\) electrons in the doubly-filled \(d\) orbitals in \slch{Fe^{2+}}, so oxidation is more likely to occur and \slEo{} is less positive than expected.
\subsection{Complex formation}
A \sldef{complex ion} is an ion containing a central metal atom or ion closely surrounded by ions or molecules, called ligands, bonded through dative bonds.

A \sldef{ligand} is a molecule or anion which contains at least one lone pair of electrons available for forming dative bonds with a central metal atom or ion.

The \sldef{coordination number} is the number of dative bonds that each central metal atom or ion can form with its ligands.

A monodentate ligand forms only one dative bond per ligand, a bidentate ligand forms two, a hexadentate ligand forms six, and so on.

Complexes with coordination number 2 are linear; those with 4 are tetrahedral or sometimes square planar, and those with 6 are octahedral.
\subsection{Colour of complexes}
Transition metal complexes are usually coloured.
\subsubsection{Explanation}
A transition metal complex has partially filled \(d\) orbitals. When ligands approach the central metal atom\slash ion for dative bonding, repulsion causes the \(d\) orbitals to be split into two groups; this effect is known as \(d\) orbital splitting.

A lower energy \(d\) electron can undergo \(d\)-\(d\) transition and be promoted to an available higher energy \(d\) orbital. During the transition, the \(d\) electron absorbs a certain wavelength of light from the visible region of the electromagnetic spectrum. The remaining wavelengths not absorbed appears as the colour of the complex.

\(3d^0\) and \(3d^{10}\) complexes give colourless solutions or white solids as there is either no lower energy \(d\) electron to be promoted, or there is no available higher energy \(d\) orbital to accommodate \(d\) electron promotion.
\subsubsection{Factors affecting colour}
The colour of a complex or compound depends on the splitting energy \(\Delta E\), which in turn depends on several factors. \begin{itemize}
\item As the oxidation state of the metal increases, the splitting energy also increases.
\item The shape of the complex results in different groupings of orbitals, also affecting the splitting energy.
\item Different ligands split the energy levels to different extents. Weak-field ligands result in small splitting energies while strong-field ligands result in large splitting energies.
\end{itemize}

Differences in colour between complexes can be explained by comparing their splitting energy, the wavelength absorbed, and the wavelengths that are not absorbed.
\subsection{Ligand exchange and stability}
A stronger ligand can replace a weaker ligand from a cation complex through \sldef{ligand exchange}.

For example, if ligand \slch{Y} is a stronger ligand than \slch{L}, then \slch{ML5Y} is more stable than \slch{ML6}, and the following reaction occurs. \[\ch{ML6 + Y -> ML5Y + L}\]

For example, when concentrated \slch{HCl} is added to a solution containing \slch{Cu^{2+}\aq{}}, the solution turns green. \[\ch{!(blue)( [Cu(H2O)6]^{2+} ) + 4 Cl^- <=> !(yellow)( [CuCl4]^{2-} ) + 6 H2O}\] The stronger \slch{Cl^-} ligands replace \slch{H2O} ligands in blue \slch{[Cu(H2O)6]^{2+}} to form yellow \slch{[CuCl4]^{2-}}. The presence of both complexes result in a green colour.

Another example is haemoglobin. \[\ch{Hb(H2O)4 + 4 O2 <=> Hb(O2)4 + 4 H2O}\] In a haemoglobin molecule, each of the four \slch{Fe^{2+}} is octahedrally bonded to five \slch{N} atoms and to an \slch{O} atom from a water molecule. The \slch{H2O} ligand may be replaced by an \slch{O2} ligand to form oxyhaemoglobin in a reversible reaction. \slch{O2} is thus taken up by blood and distributed to cells.

\slch{CN-} and \slch{CO} are toxic because the \slch{H2O} ligand may be replaced by stronger \slch{CN-} and \slch{CO} ligands in an irreversible reaction, preventing the formation of oxyhaemoglobin, eventually depriving cells of \slch{O2}.
\subsection{Dissolution of precipitates by complexation}
When dilute ammonia is gradually added to a solution containing \slch{Cu^{2+}\aq{}} ions, a pale blue precipitate is formed, which dissolves on adding more dilute ammonia.

When dilute ammonia is added gradually, \begin{equation}\label{eq:9.te.lx}\ch{[Cu(H2O)6]^{2+} + 2 OH- <=> Cu(OH)2 + 6 H2O}\end{equation} the ionic product of \slch{Cu(OH)2} increases past its \(K_\text{sp}\), so a pale blue precipitate of \slch{Cu(OH)2} is formed.

In excess ammonia, both \slch{NH3} and \slch{OH^-} compete to combine with \slch{[Cu(H2O)6]^{2+}}. \slch{NH3} ligands replace \slch{H2O} ligands to form a deep blue complex
\slch{[Cu(NH3)4(H2O)2]^{2+}}. \[\begin{split}\ch{[Cu(H2O)6]^{2+} &+ 4 NH3 \\&<=> {[}Cu(NH3)4(H2O)2{]}^{2+} + 4 H2O}\end{split}\] As concentration of \slch{[Cu(H2O)6]^{2+}} decreases, position of equilibrium~\eqref{eq:9.te.lx} shifts left, and ionic product of \slch{Cu(OH)2} decreases until it is less than its \(K_\text{sp}\), and the pale-blue precipitate dissolves.
\subsection{Catalysis}
A \sldef{catalyst} is a substance that increases the rate of reaction without itself undergoing any permanent chemical change, and provides an alternative pathway of lower activation energy for a reaction to occur.
\subsubsection{Heterogeneous catalysis}
A \sldef{heterogeneous catalyst} operates in a different physical state compared to the reactant. Transition metals and their compounds are good heterogeneous catalyst because of the availability of \(3d\) and \(4s\) electrons for temporary bond formation with reactants.

Heterogeneous catalysis function through adsorption, activation and desorption.

Adsorption occurs when temporary bonds are formed with reactant molecules when they adsorb on the catalyst surface (i.e. on the active sites).

Activation occurs when adsorption increases the surface concentration of reactants and weakens the covalent bonds within reactant molecules, lowering the activation energy. Reactant molecules are thus brought closer together and reaction can take place between the reactants molecules more easily.

Desorption occurs when resulting product molecules diffuse away from the catalyst's surface, leaving the active sites free for more reactant molecules to adsorb.
\subsubsection{Homogeneous catalysis}
\sldef{Homogeneous catalysts} operate in the same physical phase as the reactant. Transition metals and their compounds are good homogeneous catalysts because of their ability to exist in various oxidation states, facilitating the formation of reaction intermediates via alternative pathways of lower activation energy.

A substance acting as a homogeneous catalyst will react with one reactant to form an intermediate that then reacts with the other reactant. In most cases, the two separate reactions are more feasible than the two reactants reacting together.

An example is \slch{Fe^{3+}\aq{}} used to catalyse the reaction of \slch{S2O8^{2-}\aq{}} and \slch{I^-\aq{}}. The two reactants have the same charge, so they repel and the reaction is unfavourable. However, the reaction of both reactants with the catalyst is favourable as \slch{Fe^{3+}} is oppositely charged, and they attract.
\end{document}
