\documentclass[Chemistry.tex]{subfiles}
\begin{document}
\chapter{The Gaseous State}
When dealing with gases, two sets of conditions are commonly used. \sldef{Standard temperature and pressure} (s.t.p.) refers to temperature and pressure of \SI{273}{\kelvin} and \SI{101}{\kilo\pascal} respectively, while \sldef{room temperature and pressure} (r.t.p.) \SI{298}{\kelvin} and \SI{101}{\kilo\pascal} respectively.
\section{Gas Laws}
\sldef{Avogadro's law} states that equal volumes of all gases under the same conditions of temperature and pressure contain the same number of particles, i.e. \begin{equation}V \propto n\end{equation} for constant \(T\) and \(p\). It also means the volume ratio of reacting gases is equal to the mole ratio.

\sldef{Boyle's law} states that the volume of a fixed mass of gas is inversely proportional to its pressure at constant temperature, i.e. \begin{equation}p_1V_1=p_2V_2\end{equation} for constant \(T\) and \(n\).

\sldef{Charles's law} states that the volume of a fixed mass of gas is directly proportional to its thermodynamic absolute temperature at constant pressure, i.e. \begin{equation}\frac{V_1}{T_1} = \frac{V_2}{T_2}\end{equation} for constant \(n\) and \(p\).

The \sldef{combined gas law} is a combination of Boyle's law and Charles's law, which gives the relationship \begin{equation}\frac{p_1V_1}{T_1} = \frac{p_2V_2}{T_2}\end{equation}

The \sldef{ideal gas law} can be seen as a combination of the combined gas law and Avogadro's law, resulting in \begin{equation}pV = nRT\end{equation} where \(R\) is the molar gas constant, approximately \SI{8.31}{\joule\per\kelvin\per\mole}.

From the ideal gas law, we have molar mass and density \begin{align}M &= \frac{mRT}{pV}\\\rho &= \frac{pM}{RT}\end{align}

\sldef{Dalton's law} states that in a mixture of gases that do not react chemically, the total pressure of the mixture is equal to the sum of the partial pressures exerted by the constituent gases.

The \sldef{partial pressure} of a gas is the pressure that the gas would exert if it alone occupied the volume it currently occupies.
\section{Ideal Gas Behaviour}
An \sldef{ideal gas} is a gas that obeys the ideal gas law exactly. Ideal gases do not exist in reality.

A \sldef{real gas} is a gas that does not obey the ideal gas law. Under conditions of high temperature and low pressure, real gases approximate the behaviour of an ideal gas.

The ideal gas law describes gases that satisfy assumptions that \begin{slinenum}
\item intermolecular forces of attraction between gas particles are negligible
\item the total volume of gas particles is negligible compared to the volume of the container
\item gas particles are in continuous random linear motion
\item all collisions between gas particles and between gas particles and the walls of their container are perfectly elastic, such that no kinetic energy is lost in a collision, so the total kinetic energy of gas particles at a constant temperature is constant.
\end{slinenum}

At high temperatures, real gases tend toward ideal behaviour as \begin{slinenum}
\item gas particles have high kinetic energy and do not attract each other as much during collisions
\item so the assumption that intermolecular forces of attraction between gas particles are negligible is valid.
\end{slinenum}

At low temperatures, real gases deviate from ideal behaviour as \begin{slinenum}
\item gas particles possess less kinetic energy so particles tend to attract each other more during collisions
\item the intermolecular forces of attraction between gas particles become significant
\item the impact of a given particle on the walls of the container during a collision is reduced
\item this causes \(p\) and thus \(pV\) to be smaller than if the gas was ideal.
\end{slinenum}

At low pressures, real gases tend toward ideal behaviour as \begin{slinenum}
\item gas particles are far apart and there are effectively no forces of attraction between gas particles
\item the assumption that the total volume of gas particles is negligible compared to the volume of the container is valid.
\end{slinenum}

At high pressures, real gases deviate from ideal behaviour as \begin{slinenum}
\item the volume of the container is decreased and gas particles come closer together
\item the volume occupied by the gas particles becomes significant compared to the container's volume
\item this causes \(V\) and thus \(pV\) to be larger than if the gas was ideal.
\end{slinenum}

The extent of deviation of a gas from ideal behaviour depends on 3 factors. The lower the temperature or the higher the pressure, the greater the deviation, and the stronger the intermolecular forces of attraction between gas particles, the greater the deviation.

On a plot of \(pV/RT\) against \(p\), the negative deviation is due to intermolecular forces between gas particles, while the positive deviation is due to the volume of the gas particles becoming significant.
\end{document}