\documentclass[Chemistry.tex]{subfiles}
\begin{document}
\chapter{Atomic Structure}
\section{Structure and Particles}
Atoms are made up of protons, electrons and neutrons. Protons and neutrons make up the nucleus while electrons orbit the nucleus.

\sldef{Protons} are subatomic particles with the symbol \slch{p^+} with a positive electric charge of \(+1e\) and a mass of approximately \SI{1.673E-27}{\kilogram}.

\sldef{Neutrons} are subatomic particles with the symbol \slch{n} with no electric charge and a mass of approximately \SI{1.675E-27}{\kilogram}.

\sldef{Electrons} are subatomic particles with the symbol \slch{e^-} with an electric charge of \(-1e\) and a mass of approximately \SI{9.109E-31}{\kilogram}, or about \(\frac{1}{1840}\) that of a proton.

An electric field shows the direction a positive particle would be deflected in if it were placed in the field; it goes from the positive terminal to the negative terminal. Protons in an electric field will be deflected towards the negative terminal, and electrons vice versa.

In an electric field, the force exerted on a particle is given by Coulomb's inverse square law. Simplifying things, the extent of deflection of a particle is directly proportional to its charge over mass (\(q/m\)) ratio.

The \sldef{mass number} of an element is the total number of protons and neutrons in an atom of the element.

The \sldef{atomic number} is the number of protons in an atom.

\sldef{Nuclides} are any species with a specified mass and atomic number. They can be represented as \slnuclide{\(Z\)}{\(A\)}{X}, where \(A\) represents the mass number, \(Z\) the atomic number, and \slch{X} the element's symbol.

\sldef{Isotopes} are atoms of the same element containing the same number of protons but different number of neutrons.
\section{Electronic Structure}
Each electron in an atom occupies an orbital. An \sldef{orbital} can be defined as a region of space around the nucleus where there is a \SI{90}{\percent} probability of locating an electron. The \sldef{electronic structure} of an atom is the arrangement of its electrons in their orbitals.

Electron orbitals are grouped into \sldef{subshells} \(l\), which are then grouped into \sldef{principal quantum shells} \(n\).

Electrons have a principal quantum number \(n\), \(n \in \symbb{Z}^+\). Generally, the highest \(n\) found in a ground-state atom of an element is the period the element is in. The larger the \(n\) of an electron is, the further it is from the nucleus, the less strongly the electron is bound to the nucleus, and the higher the energy level of the electron.

Electrons also have an azimuthal quantum number \(l\), \(l \in \symbb{Z}^+_0\). Electrons are grouped into subshells by their \(l\), where 0 is the \(s\) subshell, 1 is \(p\), 2 is \(d\), 3 is \(f\), 4 is \(g\), and so on (except \(j\)). The highest \(l\) possible for any \(n\) is one less than the value of \(n\). Thus the first shell (\(n=1\)) only has a single \(1s\) subshell, while the second has a \(2s\) and \(2p\) subshell, and so on.

Subshells are identified by their \(n\) and \(l\), e.g. \(l=1\) in the \(n=2\) shell is referred to as the \(2p\) subshell.

Electrons also have a magnetic quantum number \(m_l\), \(m_l \in \symbb{Z}\). For a given \(l\), \(-l \leq m_l \leq l\).

Finally, electrons have a spin quantum number \(s\), \(s = \pm\frac{1}{2}\) (spin up or spin down).

These four quantum numbers \(n\), \(l\), \(m_l\) and \(s\) quantum mechanically fully describe an electron's quantum state in an atom. By the \sldef{Pauli exclusion principle}, no two electrons in an atom can occupy the same quantum state, and so must have a unique combination of quantum numbers.

It follows from the above rules that the \(s\) subshell can have 2 electrons, the \(p\) subshell 6, the \(d\) subshell 10, and so on. It then follows that the first shell can contain 2 electrons, the 2nd 8, the 3rd 18, and so on.

In each quantum shell, there is one \(s\) orbital, which is spherical in shape, as well as three dumbbell-shaped p orbitals (\(p_x\), \(p_y\) and \(p_z\)). The larger the \(n\), the larger the orbital and the further the electrons in the orbital are from the nucleus. Orbitals in a given subshell are degenerate \slIE{} they are at the same energy level.

Orbitals with larger \(n\) are generally at a higher energy level than orbitals with a smaller \(n\). Within a quantum shell, orbitals with larger \(l\) are at a higher energy level than orbitals with a smaller \(l\). Orbitals with the same \(n\) and \(l\) in the same atom are at the same energy level.

This depends on each element, however. Elements up to calcium have \(4s\) at a lower energy level than \(3d\), which is why \(4s\) fills first for potassium and calcium.

\emph{\textbf{In the `A' level syllabus}}, the \(4s\) subshell has a lower energy level than the \(3d\) subshell when empty, but (magically somehow) jumps to a higher energy level when filled.

An orbital with two electrons (completely filled) is said to be paired; one with only one is unpaired.

The \sldef{Aufbau principle} states that electrons occupy the lowest energy orbital possible first before occupying higher energy orbitals. The \sldef{Hund principle} states that orbitals in a subshell are occupied singly with the same \(s\) before pairing occurs.

These rules together generally predict the ground state electronic configuration of an atom. When any electron is promoted to a higher energy level, the species is said to be excited.

Exceptions include chromium, which has a ground state electronic configuration of \(\conc{Ar} 3d^5 4s^1\). Note the extra electron in 3d when compared to the configuration predicted by Aufbau. Another exception is copper, which has a ground state electronic configuration of \(\conc{Ar} 3d^{10} 4s^1\). These exceptions occur because the elements are at a lower energy, or `gain more stability' when they have such a configuration.

Atoms with similar electronic configurations or the same number of electrons are said to be \sldef{isoelectronic}.

For main group elements, the number of valence electrons is equal to the group number.
\section{Ionisation Energies}
The \sldef{first ionisation energy} (\slnIE{1}) is the energy required to remove one mole of electrons from one mole of gaseous atoms of an element to form 1 mole of singly charged positive gaseous ions: \begin{equation}\ch{X\gas{} -> X^+\gas{} + e^-}\end{equation}

The \(n\)th \slIE{} is the energy required to remove one mole of electrons from one mole of \((n-1)\)ly positively charged gaseous ions to form one mole of \(n\)ly positively charged gaseous ions. The higher the ionisation energy of an element, the more difficult it is to remove an electron.

The \sldef{effective nuclear charge} (ENC) is the combined effect of the nucleus's charge and the screening effect, and is approximately equal to the nuclear charge minus the number of inner core electrons. The \sldef{nuclear charge} is simply the charge of the protons in the nucleus. The \sldef{screening effect} is the phenomenon where valence electrons seem `shielded' from the electrostatic attraction from the positively charged nucleus by inner core electrons, \slIE{} electrons in inner quantum shells.

Generally, effective nuclear charge increases across the period.
\subsection{Trends in Ionisation Energy}
The \(n\)th \slIE{} is generally greater than the \((n-1)\)th \slIE{} as more energy is required to remove an electron from a more positive ion due to greater net electrostatic attraction between the nucleus and valence electrons.

Down a group, the \(n\)th \slIE{} generally decreases as \begin{slinenum}
\item the atomic radius increases due to a greater number of quantum shells
\item valence electrons are further away from the nucleus and experience greater shielding from the greater number of inner core electrons
\item thus there is weaker electrostatic attraction between the nucleus and the valence electrons
\item so less energy is required to remove a valence electron.\end{slinenum}

Across a period, the \(n\)th \slIE{} generally increases as \begin{slinenum}
\item the nuclear charge increases but the shielding effect remains relatively constant as the inner quantum shell of electrons remains the same
\item effective nuclear charge increases
\item there is stronger electrostatic attraction between the nucleus and valence electrons
\item so more energy is required to remove a valence electron.\end{slinenum}

An \slIE{} involving the removal of an \(ns^2\) electron will be higher than one involving that of an \(np^1\) electron as \begin{slinenum}
\item the \(np\) electron is further away from the nucleus than the \(ns\) electron
\item so there is weaker electrostatic attraction between the nucleus and the \(np\) electron
\item thus less energy is required to remove the \(np\) electron compared to the \(ns\) electron.\end{slinenum} E.g. \slnIE{1} of \slch{B} is less than \slnIE{1} of \slch{Be}.

An \slIE{} involving the removal of an \(np^3\) electron will be higher than one involving that of an \(np^4\) electron as \begin{slinenum}
\item there is inter-electron repulsion between electrons in the doubly filled \(np\) orbital
\item so less energy is required to remove the \(np\) electron in the element with 4 electrons in the \(np\) subshell.\end{slinenum} E.g. \slnIE{1} of \slch{O} is less than \slnIE{1} of \slch{N}.

An \slIE{} involving the removal of an electron in the \(n\)th quantum shell will be higher than one involving that of an electron in the \((n+1)\)th quantum shell, as \begin{slinenum}
\item the valence \((n+1)s\) electron is further away from the nucleus and more shielded compared to the \(ns/p/d\) electron
\item so there is weaker net electrostatic attraction between the nucleus and valence electron in the \((n+1)\)th shell
\item thus less energy is required to remove the valence electron in the \((n+1)\)th shell.\end{slinenum} E.g. \slnIE{1} of \slch{Ne} \(<\) \slnIE{1} of \slch{Na}.

When investigating successive \slIE{}s of an element, \begin{slinenum}
\item a steady increase shows electrons in the same subshell
\item a minor jump shows electrons in a subshell with lower \(l\) being removed
\item a sharp increase shows electrons in the next inner quantum shell being removed.\end{slinenum}
\end{document}