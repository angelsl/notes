\documentclass[Chemistry.tex]{subfiles}
\begin{document}
\chapter{Atoms, Molecules and Stoichiometry}
An \sldef{atom} is the smallest particle an element can be divided into without losing its identity.

\sldef{Isotopes} are atoms of the same element with the same number of protons in the nucleus but different number of neutrons.

A \sldef{molecule} is made up of a group of atoms held together by covalent bonds.

The \sldef{relative atomic mass} (\(A_r\)) of an element is the ratio of the average mass of one atom of that element to one-twelfth the mass of an atom of carbon-12.

The \sldef{relative isotopic mass} of an isotope of an element is the ratio of the mass of one atom of that isotope to one-twelfth the mass of an atom of carbon-12.

The \sldef{relative molecular mass} (\(M_r\)) of a molecule is the ratio of the average mass of one molecule to one-twelfth the mass of an atom of carbon-12.

The \sldef{relative formula mass} (\(M_r\)) of an ionic compound is the ratio of the average mass of one formula unit of that compound to one-twelfth the mass of an atom of carbon-12.

One \sldef{mole} is the amount of substance (\(n\)) that contains the same number of particles as there are atoms in exactly \SI{12.0}{\gram} of pure carbon-12. The number of particles (\(N\)) in one mole of any substance is a constant known as \sldef{Avogadro's constant} (\(L\)), which is approximately equal to \SI{6.02E23}{\per\mole}.

The \sldef{molar mass} (\(M\)) of a substance refers to the mass of one mole of that substance; it has units \si{\gram\per\mole}.
\section{Stoichiometry Involving Gases}
The \sldef{molar volume} of a gas is the volume occupied by one mole of a gas at a specified temperature and pressure. The molar volume at room temperature and pressure \SI{298}{\kelvin} and \SI{1}{\atmosphere} is \SI{24.0}{\cubic\deci\metre}, while the molar volume at standard temperature and pressure \SI{273}{\kelvin} and \SI{1}{\atmosphere} is \SI{22.4}{\cubic\deci\metre}.
\section{Stoichiometric Calculations}
\sldef{Theoretical yield} is the maximum amount of product that can be obtained from a given amount of reactants. Likewise, \sldef{actual yield} is the amount of product actually obtained from a reaction. \sldef{Percentage yield} is the ratio of actual yield to theoretical yield, expressed as a percentage.

In a reaction, the \sldef{limiting reagent} is the reactant that is completely used up at the end of the reaction. The amount of product formed is determined by the amount of the limiting reagent(s) at the start.

The \sldef{concentration} of a solution is the amount of solute, in grams or moles, per unit volume of solution. A \sldef{standard solution} is a solution with a known concentration.

\sldef{Dilution} is the process of adding more solvent to a known volume of solution to lower the concentration of the solution, with the number of moles of solute remaining the same.

The \sldef{empirical formulae} of a compound is formula showing the simplest ratio of the number of atoms of each element in a compound.

The \sldef{molecular formula} is the exact formula showing the actual number of atoms of each element present in a compound.

A \sldef{part per million} (\si{\ppm}) refers to a fraction out of a million. It is similar to a percentage.
\section{Volumetric Analysis}
Titration is a process involving the gradual addition of one solution to a fixed volume of another solution until stoichiometric amounts of the two solutions have reacted.

When direct titration is not possible, back titration is used. Titration is not possible when \begin{slinenumor}
\item one of the reactants is an insoluble solid
\item there is no suitable indicator for the titration
\item the sample may contain impurities, which may interfere with direct titration.\end{slinenumor}

Back titration involves a known excess of one reagent reacting with an unknown amount of another reagent, followed by a direct titration to find out the amount of excess reagent.

Table \ref{tb:1.indicators} details common indicators.

The \sldef{volume strength} of \slch{H2O2} is the volume of \slch{O2} (at s.t.p.) that can be evolved by the decomposition of one volume of \slch{H2O2}. It is a ratio of volume of \slch{O2} per volume of \slch{H2O2}.
\begin{table*}\centering\begin{tabular}[c]{lllll}
\toprule
\textbf{Indicator} & \textbf{pH range} & \textbf{Acid} & \textbf{Endpoint} & \textbf{Alkali}\tabularnewline
\midrule
Methyl orange & \numrange{3}{5} & Red & Orange & Yellow\tabularnewline
(screened) & \numrange{3}{5} & Purple & Grey & Green\tabularnewline
Phenolphthalein & \numrange{8}{10} & Colourless & Pale pink & Red\tabularnewline
Bromothymol blue & \numrange{6}{7.6} & Yellow & Bluish-green & Blue\tabularnewline
\bottomrule
\end{tabular}\caption{Common indicators}\label{tb:1.indicators}\end{table*}
\end{document}