\clearpage
\chapter{Annex: IUPAC organic nomenclature}
\restartlist{enumerate}
\section{List of multiplicative prefixes}

1: mono-

2: di-

3: tri-

4: tetra-

5: penta-

6: hexa-

7: hepta-

8: octa-

9: nona-

10: deca-

11: undeca-

12: dodeca-

13: trideca-

14: tetradeca-

15: pentadeca-

16: hexadeca-

17: heptadeca-

18: octadeca-

19: nonadeca-

20: icosa-/eicosa-

21: henicosa-/heneicosa-

22: docosa-

23: tricosa-

30: triaconta-

31: hentriaconta-

32: dotriaconta-

40: tetraconta-

50: pentaconta-

60: hexaconta-

70: heptaconta-

80: octaconta-

90: nonaconta-

100: hecta-

200: dicta-

300: tricta-

400: tetracta-

500: pentacta-

600: hexacta-

700: heptacta-

800: octacta-

900: nonacta-

1000: kilia-

2000: dilia-

3000: trilia-

4000: tetralia-

5000: pentalia-

6000: hexalia-

7000: heptalia-

8000: octalia-

9000: nonalia-

These prefixes are applied in little endian order (least significant
first). E.g. C\textsubscript{9267}H\textsubscript{18536} is hepta- (7) +
hexaconta- (60) + dicta- (200) + nonalia- (9000) + -ane =
heptahexacontadictanonaliane. When naming the carbon skeleton, for
chains with 1 to 4 carbons, meth-, eth-, prop- and but- are used instead
of 1 to 4 above.

\section{Functional group priority}

\begin{longtable}[c]{@{}lllll@{}}
\toprule
\textbf{P} & \textbf{Functional group} & \textbf{Formula} &
\textbf{Prefix} & \textbf{Suffix}\tabularnewline
\midrule
\endhead
\textbf{1} & \textbf{Cations} & & -onio- & -onium\tabularnewline
& e.g. Ammonium & NH\textsubscript{4}\textsuperscript{+} & ammonio- &
-ammonium\tabularnewline
\textbf{2} & \textbf{Carboxylic acids} & --COOH & carboxy- & -oic
acid*\tabularnewline
& Carbothioic S-acids & --COSH & sulfanylcarbonyl- & -thioic
\emph{S}-acid*\tabularnewline
& Carboselenoic Se-acids & --COSeH & selanylcarbonyl- & -selenoic
\emph{Se}-acid*\tabularnewline
& Sulfonic acids & --SO\textsubscript{3}H & sulfo- & -sulfonic
acid\tabularnewline
& Sulfinic acids & --SO\textsubscript{2}H & sulfino- & -sulfinic
acid\tabularnewline
\textbf{3} & \textbf{Carboxylic acid derivatives} & & &\tabularnewline
& Esters & --COOR & R-oxycarbonyl- & -R-oate\tabularnewline
& Acyl halides & --COX & halocarbonyl- & -oyl halide*\tabularnewline
& Amides & --CONH\textsubscript{2} & carbamoyl- & -amide*\tabularnewline
& Imides & --CON=C\textless{} & -imido- & -imide*\tabularnewline
& Amidines & --C(=NH)NH\textsubscript{2} & amidino- &
-amidine*\tabularnewline
\textbf{4} & \textbf{Nitriles} & --CN & cyano- &
-nitrile*\tabularnewline
& Isocyanides & --NC & isocyano- & isocyanide\tabularnewline
\textbf{5} & \textbf{Aldehydes} & --CHO & formyl- & -al*\tabularnewline
& Thioaldehydes & --CHS & thioformyl- & -thial*\tabularnewline
\textbf{6} & \textbf{Ketones} & =O & oxo- & -one\tabularnewline
& Thiones & =S & sulfanylidene- & -thione\tabularnewline
& Selones & =Se & selanylidene- & -selone\tabularnewline
& Tellones & =Te & tellanylidene- & -tellone\tabularnewline
\textbf{7} & \textbf{Alcohols} & --OH & hydroxy- & -ol\tabularnewline
& Thiols & --SH & sulfanyl- & -thiol\tabularnewline
& Selenols & --SeH & selanyl- & -selenol\tabularnewline
& Tellurols & --TeH & tellanyl- & -tellurol\tabularnewline
\textbf{8} & \textbf{Hydroperoxides} & & &\tabularnewline
& Peroxols & -OOH & hydroperoxy- & -peroxol\tabularnewline
& Thioperoxols (Sulfenic acid) & -SOH & hydroxysulfanyl- &
-\emph{SO}-thioperoxol\tabularnewline
& Dithioperoxols & -SSH & disulfanyl- & -dithioperoxol\tabularnewline
\textbf{9} & \textbf{Amines} & --NH\textsubscript{2} & amino- &
-amine\tabularnewline
& Imines & =NH & imino- & -imine\tabularnewline
& Hydrazines & --NHNH\textsubscript{2} & hydrazino- &
-hydrazine\tabularnewline
\bottomrule
\end{longtable}

* The carbon in these suffixes are included in the parent chain, and are
most commonly used.

