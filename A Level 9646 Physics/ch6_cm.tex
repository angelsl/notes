\documentclass[Physics.tex]{subfiles}
\begin{document}
\chapter{Motion in a Circle}
\sldef{Angular displacement} is the angle in radians through which a body is rotated about an axis. \begin{equation}\mathbf{\theta} = \frac{s}{r}\end{equation} One \sldef{radian} is the angle subtended at the centre of a circle by an arc that is equal in length to the radius of the circle.

\sldef{Angular velocity} is the rate of change of angular displacement w.r.t. time. \begin{equation}\mathbf{\omega} = \sldd{\mathbf{\theta}}{t}\end{equation}

The \sldef{period} of an object in circular motion is the time taken for the object to make one complete revolution, and the frequency of an object in circular motion is the number of complete revolutions made by the object per unit time. \begin{align}T &= f^{-1} = \frac{2\pi}{\omega}\\f &= T^{-1} = \frac{\omega}{2\pi}\end{align}

The \sldef{linear velocity} of an object moving in a non-linear path is tangential to the object's path. \begin{equation}\mathbf{v} = \mathbf{r} \times \mathbf{\omega}\end{equation}

An object in \sldef{uniform circular motion} has constant angular velocity and a constant linear speed.

The \sldef{centripetal acceleration} (acceleration in the direction of the centre of motion) is \begin{equation}\mathbf{a}_c = \frac{\mathbf{v}^2}{\mathbf{r}} = \mathbf{r\omega}^2\end{equation} For the centripetal acceleration to exist, the net force must be in that direction. A net force in the direction of the centre of motion is known as a centripetal force. \begin{equation}\mathbf{F}_c = m\frac{\mathbf{v}^2}{\mathbf{r}} = m\mathbf{r\omega}^2\end{equation} In some cases, friction can be the force providing for the centripetal force.
\end{document}