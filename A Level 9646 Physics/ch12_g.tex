\documentclass[Physics.tex]{subfiles}
\begin{document}
\chapter{Gravitational Field}
Newton's \sldef{law of universal gravitation} states that every particle in the universe attracts every other particle with a force directly proportional to the product of their masses and inversely proportional to the square of the distance between them i.e. \begin{equation}\left|\mathbf{F}\right| = G\frac{m_{1}m_{2}}{r^{2}}\end{equation} where \(G\) is the gravitational constant.

The law of gravitation is an inverse-square law i.e. the magnitude of force is inversely proportional to the square of something -- in this case the square of the separation between the particles.

This equation holds only for point masses, but large masses at a large distance can be assumed, without losing too much accuracy, to be a point mass, so this equation is still reasonably valid.

The region in space where an object exerts a gravitational force on another object is the \sldef{gravitational field} of the first object. A gravitational field can be represented by field lines representing the resultant direction of gravitational force acting on a mass placed at any point in the field. Where the field lines are closer, the field is stronger, and vice versa.

The \sldef{gravitational field strength} \textbf{g} at a point in a gravitational field is the gravitational force per unit mass acting on a body placed at that point. \begin{equation}\left|\mathbf{g}\right| = G\frac{Mm}{mr^{2}} = G\frac{M}{r^{2}}\end{equation}

Near the surface of the Earth, \(\mathbf{g}\) is approximately constant, as \(r\) does not vary very appreciably. However, \(\mathbf{g}\) varies over different points on Earth's surface, as the Earth is not a perfect sphere -- it bulges at the equator, the density of the Earth is not uniform, and the Earth is rotating about an axis through its poles -- so for an object not at the poles, gravity also has to provide for centripetal acceleration.

An object is only truly weightless when it experiences no gravitational force at all. Otherwise, it experiences apparent weightlessness -- it simply does not feel its weight as it experiences no normal force.

The \sldef{gravitational potential energy} \(U\) of a mass at a point in a gravitational field is the work done by an external force in bringing the mass from infinity to that point without acceleration. \begin{equation}U = -G\frac{Mm}{r}\end{equation} At infinity, \(U = 0\). The negative sign arises from the fact that the gravitational force is attractive in nature.

The \sldef{gravitational potential} \(\Phi\) at a point in a gravitational field is the work done per unit mass by an external force in bringing a test mass from infinity to that point without acceleration.\begin{equation}\Phi = -G\frac{M}{r}\end{equation}

The \sldef{escape speed} is the minimum speed with which a mass should be launched from a planet's surface to escape the planet's gravitational field. It is simply the speed that gives the mass kinetic energy that makes the mass's total energy exactly equal what its total energy would be if it were stationary at infinity.

An \sldef{equipotential} line or surface is formed by all the points that have the same potential. Objects moving along an equipotential do not lose or gain any energy.

\sldef{Kepler's 3rd law} states that the square of an object's orbit period is proportional to the cube of its orbit radius.

A \sldef{geostationary satellite} of a planet is a satellite that orbits the planet such that it will always be above the same point on the planet. This means that the satellite must be orbiting the planet's equator in the direction of the planet's axial rotation, and the period of the orbit must be equal to the period of the rotation.

Geostationary satellites are often used for communication. They are useful because they always stay over the same point, so there is no need to adjust satellite dishes; they are also high above the Earth and so can see large areas of the Earth. However, also due to their high altitude, images taken tend to be of low spatial resolution, and because they must be over the equator, they are of limited use for latitudes more than about \SI{70}{\degree} N or S.
\end{document}