\documentclass[Physics.tex]{subfiles}
\begin{document}
\chapter{Work, Energy and Power}
The \sldef{work} done by a force on a body is the product of the force and the displacement of the body in the direction of the force. \begin{equation}W = \mathbf{F}\cdot\mathbf{s} = \left|\mathbf{F}_\parallel\right|\left|\mathbf{s}\right|\end{equation} The SI unit of work done is the \sldef{joule} (\si{\joule}), which is the work done by a force of \SI{1}{\newton} in moving an object by \SI{1}{\metre}.

When a gas is expanding, it does work on external pressure of
\begin{equation}W = p\mathbf{A}\cdot\mathbf{s} = p\Delta V = p(V_f - V_i)\end{equation} When a gas is compressed, work is done on it by the surroundings, and work done by the gas is negative.

\sldef{Energy} is the stored ability to do work. Energy can be transferred through thermal transfer (heating), or mechanical transfer (by doing work).

There are many forms of energy, including kinetic energy, gravitational potential energy, elastic potential energy, chemical potential energy, electrical potential energy and light energy.

\sldef{Kinetic energy} is the energy possessed by a body due to its motion. \begin{equation}T = \frac{1}{2}m\mathbf{v}^2\end{equation}

\begin{slderiv}Consider a constant net force \(\mathbf{F}\) acting on an object with mass \(m\) with some initial velocity \(\mathbf{u}\). \begin{equation*}\begin{split}\mathbf{v}^2 &= \mathbf{u}^2 + 2\mathbf{a}\mathbf{s}\\\implies W &= \mathbf{F}\cdot\mathbf{s} = m\mathbf{as} \mathrel(\mathrel\because\mathbf{F} = m\mathbf{a}) = \frac{1}{2}m(\mathbf{v}^2 - \mathbf{u}^2)\end{split}\end{equation*} The work done by \(\mathbf{F}\) increases only the kinetic energy of the block, so \[W = \Delta T \mathrel\therefore T_f - T_i = \frac{1}{2}m(\mathbf{v}^2 - \mathbf{u}^2)\] If \(\mathbf{u} = 0\), \[T_i = 0 \mathrel\therefore T_f = \frac{1}{2}m\mathbf{v}^2 \mathrel\therefore T = \frac{1}{2}m\mathbf{v}^2\qedhere\]\end{slderiv}

\sldef{Gravitational potential energy} is the energy a body possesses due to its position relative to something else. \begin{equation}U = mgh\end{equation}

\begin{slderiv}Consider a body of mass \(m\) at height \(h_i\) above the surface of the Earth, with acceleration of free fall \(\mathbf{g}\). A force \(\mathbf{F}\) raises the body vertically at a constant velocity to height \(h_f\). \[\begin{split}\mathbf{a} &= 0 \mathrel\therefore \mathbf{F}_\text{net} = 0 \mathrel\therefore \left|\mathbf{F}\right| - mg = 0 \mathrel\therefore \left|\mathbf{F}\right| = mg\\\mathrel\therefore W &= \left|\mathbf{F}\right|\left|\mathbf{s}\right| = mg(h_f - h_i)\end{split}\] The work done by \(\mathbf{F}\) increases only the gravitational potential energy of the block, so \[W = \Delta U \mathrel\therefore U_{f} - U_{i} = mg(h_f - h_i)\] If we take the gravitational potential energy at \(h_i\) as the reference level i.e. \(U\) at \(h_i = 0\), then an object of mass \(m\) raised by height \(h\) vertically above the reference level has gravitational potential energy of \(U = mgh\).\qedhere\end{slderiv}

\sldef{Elastic potential energy} is the energy possessed by an elastic body when it is subjected to deformation. \begin{equation}E = \frac{1}{2}k(\Delta\mathbf{x})^2\end{equation}

The principle of \sldef{conservation of energy} states that the total energy is constant i.e. energy cannot be created or destroyed, but can be converted from one form to another.

A \sldef{dissipative force} is a resistive force that reduces the energy of a system, like friction.

A non-isolated system is one where the system can interact and exchange energy with its environment. Energy of a non-isolated system is not conserved, but the energy of the system and its environment is always conserved.

\sldef{Power} is the work done per unit time, or the rate at which work is done. \begin{equation}P = \sldd{W}{t}\end{equation}

For a constant force \(\mathbf{F}\) acting on an object such that the object moves with a constant velocity \(\mathbf{v}\), \begin{equation}P = \sldd{W}{t} = \sldd{\mathbf{Fs}}{t} = \mathbf{F}\sldd{s}{t} + \mathbf{s}\sldd{F}{t} = \mathbf{F}\sldd{s}{t} = \mathbf{F}\cdot\mathbf{v}\end{equation}

\sldef{Efficiency} is the ratio between the useful output of an energy conversion and the input.
\end{document}