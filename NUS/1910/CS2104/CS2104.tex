\documentclass[fontsize=10pt]{slnotes}
\usepackage{listings}
\lstset{language=Haskell,
basicstyle=\ttfamily,
breaklines=true,aboveskip=0pt,postbreak=\mbox{\(\hookrightarrow\)}}
\begin{document}
\chapter{CS2104: Haskell}
\begin{lstlisting}
map :: (a -> b) -> [a] -> [b]
map _ []     = []
map f (x:xs) = f x : map f xs

foldr :: Foldable t => (a -> b -> b) -> b -> t a -> b

foldl :: Foldable t => (b -> a -> b) -> b -> t a -> b

scanl :: (b -> a -> b) -> b -> [a] -> [b]
scanl = scanlGo
  where
    scanlGo :: (b -> a -> b) -> b -> [a] -> [b]
    scanlGo f q ls = q :
        (case ls of
           []   -> []
           x:xs -> scanlGo f (f q x) xs)

scanr :: (a -> b -> b) -> b -> [a] -> [b]
scanr _ q0 []     = [q0]
scanr f q0 (x:xs) = f x q : qs
  where qs@(q:_) = scanr f q0 xs

filter :: (a -> Bool) -> [a] -> [a]
filter _pred [] = []
filter pred (x:xs)
  | pred x    = x : filter pred xs
  | otherwise = filter pred xs

reverse :: [a] -> [a]
reverse = foldl (flip (:)) [] -- or
reverse l = rev l []
  where
    rev []     a = a
    rev (x:xs) a = rev xs (x:a)

take :: Int -> [a] -> [a]
take n _ | n <= 0 = []
take _ []         = []
take n (x:xs)     = x : take (n-1) xs

drop :: Int -> [a] -> [a]
drop n xs | n <= 0 = xs
drop _ []          = []
drop n (_:xs)      = drop (n-1) xs

(++) :: [a] -> [a] -> [a] -- infixr 5
(++) []     ys = ys
(++) (x:xs) ys = x : xs ++ ys

(:) :: a -> [a] -> [a] -- infixr 5

($) :: forall r a (b :: TYPE r). (a -> b) -> a -> b -- infixr 0
f $ x =  f x

data T = AAA | BBB Int | CCC [String] | DDD {name::String, age::Int}
     deriving Show
-- access functions are name, age
v1 = AAA
v2 = CCC ["hello","cs2104"]
v3 = DDD {name="Ali", age=12}
-- update syntax sugar
v4 = v3 {name="Mohd"}
-- desugared
setName (DDD _ age_) name = DDD name age_

sieve (x:xs) = x:(sieve [y | y <-xs, (mod y x) /= 0])
primes = sieve [2..]
\end{lstlisting}
\section{Monads}
\begin{lstlisting}
x1 = return 42 :: Maybe Int
x2 = return 42 :: [Int]
x3 = return 42 :: IO Int

perform_twice m =
  do
    x1 <- m
    x2 <- m
    return (x1,x2)

perform_twice' :: Monad m => m t -> m (t, t)
perform_twice' m = m >>= \ t1 ->
                   m >>= \ t2 ->
                   return (t1,t2)

const :: a -> b -> a
const x _ =  x

mult_table n m = [(a,b,a*b)| a <- [1..n],b <-[1..m]]
\end{lstlisting}

The Functor class is used for types that can be mapped over. Instances of Functor should satisfy the following laws:
\begin{lstlisting}
fmap id == id
fmap (f . g) == fmap f . fmap g
class Functor f where
    fmap :: (a -> b) -> f a -> f b
\end{lstlisting}
Replace all locations in the input with the same value.
\begin{lstlisting}
    (<$) :: a -> f b -> f a
    (<$) =  fmap . const

class Functor f => Applicative f where
    {-# MINIMAL pure, ((<*>) | liftA2) #-}
    pure :: a -> f a
    (<*>) :: f (a -> b) -> f a -> f b
    (<*>) = liftA2 id
\end{lstlisting}
Sequence actions, discarding the value of the first argument.
\begin{lstlisting}
    (*>) :: f a -> f b -> f b
    a1 *> a2 = (id <$ a1) <*> a2
\end{lstlisting}
Sequence actions, discarding the value of the second argument.
\begin{lstlisting}
    (<*) :: f a -> f b -> f a
    (<*) = liftA2 const
\end{lstlisting}

Instances of Monad should satisfy the following laws:
\begin{lstlisting}
'return' a '>>=' k = k a
m >>= return = m
m >>= (\x -> k x >>= h) = (m >>= k) >>= h
\end{lstlisting}

Furthermore, the Monad and Applicative operations should relate as follows:
\begin{lstlisting}
pure = return
<*> = ap
\end{lstlisting}

The above laws imply:
\begin{lstlisting}
fmap f xs = xs >>= return . f
(>>) = (*>)

class Applicative m => Monad m where
\end{lstlisting}
Sequentially compose two actions, passing any value produced by the first as an argument to the second.
\begin{lstlisting}
    (>>=) :: forall a b. m a -> (a -> m b) -> m b
\end{lstlisting}
Sequentially compose two actions, discarding any value produced
by the first, like sequencing operators (such as the semicolon)
in imperative languages.
\begin{lstlisting}
    (>>) :: forall a b. m a -> m b -> m b
    m >> k = m >>= \_ -> k
\end{lstlisting}
Inject a value into the monadic type.
\begin{lstlisting}
    return :: a -> m a
    return = pure
\end{lstlisting}
Fail with a message.  This operation is not part of the
mathematical definition of a monad, but is invoked on pattern-match
failure in a do expression.
\begin{lstlisting}
    fail :: String -> m a
    fail s = errorWithoutStackTrace s
\end{lstlisting}

\clearpage\chapter{Prolog}
\section{Tutorial 6}
\begin{lstlisting}[escapechar=\~,escapebegin=\rmfamily,language=Prolog]
father(john, mary).
father(john, tom).
father(kevin, john).
mother(eva, mary).
mother(eva, tom).
mother(cristina, john).
mother(cristina, kelly).
mother(kelly, alloy).
male(john).
male(kevin).
male(tom).
% ~works for known X~
female(X):-atom(X),not(male(X)).
parent(X,Y) :- father(X,Y).
parent(X,Y) :- mother(X,Y).
grandparent(X,Y) :- parent(X,Z),parent(Z,Y).
daughter(X,Y) :- female(X),parent(Y,X).
sibling(X,Y) :- parent(Z,X),parent(Z,Y),X\==Y.
true_sibling(X,Y) :- father(Z,X),father(Z,Y),X\==Y,mother(M,X),mother(M,Y).
ancestor(X,Y) :- parent(X,Y).
ancestor(X,Y) :- parent(X,Z),ancestor(Z,Y).

/* Q1 */
/* ~Q1. (Answer and explanation)~
grandparent(G,mary).
% parent(kevin, john), parent(john, mary)
G = kevin ;
% parent(cristina, john), parent(john, mary)
G = cristina ;
false.

ancestor(A,mary).
% parent(john, mary)
A = john ;
% parent(eva, mary)
A = eva ;
% parent(kevin, john), ancestor(john, mary)
A = kevin ;
% parent(cristina, john), ancestor(john, mary)
A = cristina ;
false.

sibling(S,mary).
% parent(john, tom), parent(john, mary)
S = tom ;
% parent(eva, tom), parent(eva, mary)
S = tom ; */

/* Q2
   aunt(X,Y)~ means ~X~ is aunt of ~Y
   cousin(X,Y)~ means ~X~ is cousin of ~Y
   nephew(X,Y)~ means ~X~ is nephew of ~Y
~Many possible answers since conjunction is in general commutative. Below is one of the possible answers.~
*/
aunt(X,Y) :- sibling(X,Z), parent(Z,Y), female(X).
cousin(X,Y) :- parent(Z,X), sibling(Z,W), parent(W,Y).
nephew(X,Y) :- male(X), parent(Z,X), sibling(Z,Y).

sel(X,[X|_]).
sel(X,[_|T]) :- sel(X,T).
remDupl([],[]).
remDupl([H|T],R) :- sel(H,T), remDupl(T,R).
remDupl([H|T],[H|R]) :- remDupl(T,R).

append([],Y,Y).
append([X|Xs],Y,[X|Rs]):-append(Xs,Y,Rs).
rev([],[]).
rev([X|Xs],Y) :- rev(Xs,Y2),append(Y2,[X],Y).
rev3([],[]).
rev3([X|Xs],Y) :- append(Y2,[X],Y),rev3(Xs,Y2).
rev2([],[]).
rev2(.(X,Xs),Y) :- rev2(Xs,Y2),append(Y2,[X],Y).
fact(0,1).
fact(N,R) :- N>0, M is N-1, fact(M,R1), R is N*R1.
select2(X,[X]).
select2(X,[_|T]):-select2(X,T).

/* Q3
 last(Xs,X) :- X~ is the last element of ~Xs
 len(Xs,N) - N~ is the length of ~Xs
 nth(Xs,I,X) - X~ is the ~I~-th element of ~Xs~, starting from 0.~
 occurs(Ys,X,N) - X~ occurs ~N~-times in ~Ys~.~

   last([X]) ==> X
   last([X|T]) ==> last(T)
*/

last([X], X):-!.
last([_|Y], X) :- last(Y, X).

len([], 0).
len([_|Y], N) :- len(Y, M), N is M+1.

nth([X|_], 0, X) :- !.
nth([_|Y], N, Z) :- N > 0, M is N-1, nth(Y, M, Z).

occurs([], _, 0) :- !.
occurs([X|Y], X, N) :- occurs(Y, X, M), N is M+1, !.
occurs([X|Y], Z, N) :- X \= Z, occurs(Y, Z, N).

/* Q5
   Hint: you may use mod operation
   ?- divisors(30,X).
	X = [1,2,3,5,6,10,15,30]
*/

divides(X, Y) :- Y > 0, 0 is X mod Y.
divisors1(_, 0, []) :- !.
divisors1(X, N, [N|L]):- N>0, P is N-1, divides(X,N), divisors1(X,P,L), !.
divisors1(X, N, L):- N>0, P is N-1, not(divides(X,N)), divisors1(X,P,L).

divisors(X, L) :- X > 0, divisors1(X, X, L).
\end{lstlisting}
Here are the definitions of similar functions using constraint logic programming finite domain. Notice that these definitions are more powerful than the definitions given above. e.g. \texttt{nth\-(\-[\-1\-,\-4\-,\-10\-,\-5\-,\-4\-,\-8\-,\-10\-,\-7\-,\-8\-,\-10\-]\-,\-X\-,\-8\-)} does not provide us the positions of 8 in the list, however, \texttt{cnth\-(\-[\-1\-,\-4\-,\-10\-,\-5\-,\-4\-,\-8\-,\-10\-,\-7\-,\-8\-,\-10\-]\-,\-X\-,\-8\-)} provides us the positions of 8, 5th and 8th. You can try similar example to see noccurs is more powerful than occurs.
\begin{lstlisting}[escapechar=\~,escapebegin=\rmfamily,language=Prolog]
:- use_module(library(clpfd)).

clen([], 0).
clen([_|Y], N) :- clen(Y, M), N #= M+1.

cnth([X|_], 0, X).
cnth([_|Y], N, Z) :- N #> 0, N #= M+1, cnth(Y, M, Z).

coccurs([], _, 0).
coccurs([X|Y], X, N) :- N #= M+1, coccurs(Y, X, M).
coccurs([X|Y], Z, N) :- X #\= Z, coccurs(Y, Z, N).

:- use_module(library(clpfd)).

appendall([],[]).
appendall([L|Ls],R) :- appendall(Ls,R1),append(L,R1,R).
\end{lstlisting}

\begin{enumerate}
\item The Sagittarius owes more money than Landyn.
\item The person who ordered soup of the day as an appetizer owes less money than the Aries.
\item The patient with the \$550 hospital bill didn't order soup of the day.
\item The 5 people were the patient with the \$760 hospital bill, the person who ordered soup of the day as an appetizer, Skylar, the Pisces, and the patient with the \$960 hospital bill.
\item Either the Sagittarius or the Virgo ordered fried pickles.
\item The Pisces didn't order buffalo wings.
\item The person who ordered fried pickles as an appetizer is Kayla.
\item The patient with the \$960 hospital bill is the Virgo.
\item Kayla owes less money than Skylar.
\item Of Perla and the person who ordered potato skins as an appetizer, one has the \$1610 hospital bill and the other has the \$960 hospital bill.
\end{enumerate}

\begin{lstlisting}[language=Prolog]
main(Answer) :- Answer =
[['Jorge'=Jorge, 'Kayla'=Kayla,
'Landyn'=Landyn, 'Perla'=Perla,
'Skylar'=Skylar],
['BuffaloWings'=Wings, 'FriedMozarella'=Mozarella,
'FriedPickles'=Pickles, 'PotatoSkins'=PSkins,
'SoupofDay'=Soup],
['Aries'=Aries, 'Leo'=Leo,
'Pisces'=Pisces, 'Sagittarius'=Sagittarius,
'Virgo'=Virgo]],
Names = [Jorge,Kayla,Landyn,Perla,Skylar],
Appetizers = [Wings,Mozarella,Pickles,PSkins,Soup],
Signs = [Aries,Leo,Pisces,Sagittarius,Virgo],
Bills=[B550,B760,B960,B1610,B2500],
Bills=[1,2,3,4,5],
/*B550=1,B760=2,B960=3,B1610=4,B2500=5,*/
L=[Names,Appetizers,Signs],
appendall(L,All),
All ins 1..5,
all_different(Names), all_different(Appetizers),
all_different(Signs),
Sagittarius #> Landyn,
Soup #< Aries,
B550 #\= Soup,
all_different([B760,Soup,Skylar,Pisces,B960]),
(Sagittarius #= Pickles; Virgo #= Pickles),
Pisces #\= Wings,
Kayla #= Pickles,
Virgo #= B960,
Kayla #< Skylar,
tuples_in([[Perla, PSkins]], [[B1610, B960], [B960, B1610]]),
label(All).

/*
Answer =
[['Jorge'=3, 'Kayla'=2, 'Landyn'=1, 'Perla'=4, 'Skylar'=5],
['BuffaloWings'=5, 'FriedMozarella'=1, 'FriedPickles'=2, 'PotatoSkins'=3, 'SoupofDay'=4],
['Aries'=5, 'Leo'=4, 'Pisces'=1, 'Sagittarius'=2, 'Virgo'=3]]
*/
\end{lstlisting}

\clearpage\chapter{OCaml}
\begin{lstlisting}[escapechar=\~,escapebegin=\rmfamily,language={[Objective]Caml}]
~Q1: Consider a tail-recursive method. Re-implement this method using an iterative loop together with two imperative ref type values that are internal to the method.~
let fib n =
  let rec aux n a b =
    if n<=0 then (a,b)
    else aux (n-1) (a+b) a
 in fst(aux n 1 0);;

let fib_imp n =
  let a = ref 1 in
  let b = ref 0 in
  for i = n downto 1 do
    let aplusb = !a + !b in
    let () = b := !a in
    let () = a := aplusb in
    ()
  done; !a ;;

let fib_imp2 n =
  let a = ref 1 in
  let b = ref 0 in
  let rec aux n =
    if n<=0 then ()
    else
      let aplusb = !a + !b in
      let () = b := !a in
      let () = a := aplusb in
      aux (n-1) in
  let () = aux n in
  !a;;

~Q2: The for-loop is implemented as follows. What is the polymoprhic type of this method?~

Ans: for_loop:: ini -> int -> (int -> unit) -> unit

let for_loop init final stmt =
  let rec aux i =
    if i<=final then (stmt i; aux (i+1))
    else ()
  in aux init

~Q3: Write two higher-order methods that would implement a for-down-to loop iterator, and a while loop method~

(* let while_loop t1 body = *)
(*   failwith "TBI" *)
(* ;; *)

let rec while_loop (t1:unit -> bool) (body:unit -> unit) : unit =
  if t1 () then (body(); while_loop t1 body)
  else ()
;;

(* let for_loop init final stmt = *)
(*   failwith "TBI" *)
(* ;; *)

let for_loop_down init final stmt =
  let rec aux i =
    if i>=final then (stmt i; aux (i-1))
    else ()
  in aux init

~Q4: Implement fib function with memoization using a hash table. How does this compares with your implementation in Q1?~

~Ans: memoization is a more general memoization (which applies to other functions, but require some extra codes to maintain table and its retrieval. Memoization also leads to extra space costs.~

(* let fib8 n = *)
(*   let h = Hashtbl.create 10 in *)
(*   let rec aux n = *)
(*     if n<=1 then 1 *)
(*     else  *)
(*       aux (n-1) + aux (n-2) *)
(*   in aux n;; *)

let fib8 =
  let h = Hashtbl.create 10 in
  fun n ->
  let rec aux n =
    if n<=1 then 1
    else
      try
       Hashtbl.find h n
      with _ ->
       let r = aux (n-1) + aux (n-2) in
       let () = print_endline ("hash add "^(string_of_int n)) in
       let () = Hashtbl.add h n r in
       r
  in aux n;;

~Q1. Consider a generic memoizing function~
memo_fn : ('a -> 'b) -> 'a -> 'a
~Show how this may be used to memoize fibonacci function~

let memo_fn (f:'a ->'b) : 'a -> 'b =
  let () = print_endline ("hash create ") in
  let h = Hashtbl.create 10 in
   (fun n ->
      try
       Hashtbl.find h n
      with _ ->
       let r = f n in
       let () = print_endline ("hash add ") in
       let () = Hashtbl.add h n r in
         r)
   ;;

~this version is generic + fast but need lazy evaluation~
let rec fib n =
        let f = Lazy.force aux in
	if n<=1 then 1
        else (f (n-1)) + (f (n-2))
and aux = lazy (memo_fn fib)
 ;;

~this version does not memoize properly~
let rec fib3 n =
        let f = (memo_fn fib3) in
	if n<=1 then 1
        else (f (n-1)) + (f (n-2))
;;

(* let fib n = memo_fn fib1;; *)

(* let rec fib1 n = *)
(*   if n<=1 then 1 *)
(*   else *)
(*      let fib1 = memo_fn fib in  *)
(*      fib1 (n-1) + fib1 (n-2) *)
(* ;; *)

(* lazy :: 'a -> 'a Lazy.t *)
(* force :: 'a Lazy.t -> a *)

let rec fib1 n =
	if n<=1 then 1
        else (fib1 (n-1)) + (fib1 (n-2))
;;

(* let rec bin (n,m) = *)
(*    let f = Lazy.force bin' in *)
(*    if base then r *)
(*    else f (n-1,m-1)+ f(n-1,m) *)
(* and bin' = lazy (memo_fn bin) *)
(* ;; *)

(* below does not work! *)
let rec fib2 n =
	let aux = lazy (memo_fn fib2) in
        let f = Lazy.force aux in
	if n<=1 then 1
        else f (n-1) + f (n-2)
;;

(* print_endline(string_of_int (fib(40))) *)
\end{lstlisting}
Comment on the Objects and Modules for OCaml below.

You have been given a imperative stack of integers,
called stack\_int, as an OO type class.

Study the stack carefully, and describe which are
pure vs imperative methods. Take note of mutable
variables inside objects

Ans: push, pop are imperative since it modifies memory
\texttt{top}, \texttt{is\_empty}, \texttt{size} and \texttt{show} are pure methods

You are tasked to make a new polymorphic stack by
    declaring instead:
    \texttt{class ['a] stack}
    Create such a polymorphic stack.
    Write test harness that would then creates two stacks:
      (i) a stack of integer and
      (ii) another stack of strings.

   Ans: see s short fragment starting with
       \texttt{class ['a] stack =}

The current stack is an imperative one with updates.
    Can you design a polymorphic functional stack, call
    it \texttt{stack\_pure}, without any updates

\begin{lstlisting}[escapechar=\~,escapebegin=\rmfamily,language={[Objective]Caml}]
class stack_pure_int =
object (self)
  val stk = []
  method push (i:int) = i:stk
  (* ~make sure modified stack is always returned~ *)
\end{lstlisting}

You are given a module, called Mstack\_int, which can be
    used to build an imperative stack of integers; by using
     the command \texttt{Mstack\_int.make}.

    Can you re-design a more general polymorphic module,
    calling it \texttt{Mstack} module?
\begin{lstlisting}[escapechar=\~,escapebegin=\rmfamily,language={[Objective]Caml}]
module Mstack_int =
struct
  type t = ('a list) ref
  let make () : t = ref []
  let push (stk:t) (i:'a) : unit =
    stk := i::!stk
\end{lstlisting}

    Contrast the pros and cons of OOP versus modules.

Modules are meant for separate compilation and name space conservation
OOP is meant to support class inheritance and subtyping
both language feature are to support greater code reuse safely

Can you also re-design a functional version of the
    stack module that works for arbitrary types?

\begin{lstlisting}[escapechar=\~,escapebegin=\rmfamily,language={[Objective]Caml}]
module Mstack =
struct
  type t = ('a list)
  let make () : t = []
  let push (stk:t) (i:'a) : t = i::stk
\end{lstlisting}

For polymorphic stack, one of the problem we encounter
    is obtaining the string representation of an element
    The show method requires an extra \texttt{to\_string} function-type
    parameter.

    We can avoid this problem by using a functor, called \texttt{Sstack},
    to structure our code. Complete this functor, so that we
    can use it to build stacks which have the show method that
    is imported for another module.
\begin{lstlisting}[escapechar=\~,escapebegin=\rmfamily,language={[Objective]Caml}]
module type SHOW_TYPE =
sig
  type t
  val show : t -> string
end;;

module INT_Elem : SHOW_TYPE =
struct
  type t = int
  let show (i:t) = string_of_int i
end;;

(* functor that takes a module of the show type *)
module FStack =
    functor (Elt: SHOW_TYPE) ->
struct
  type elem = Elt.t
  type t = elem list ref
  let show (stk:t) : string =
    "["^(String.concat "," (List.map Elt.show !stk))^"]"
end;;

class ['a] stack =
object (self)
  val mutable stk = []
  method push (i:'a) =
    begin
      stk <- i::stk
    end
  method pop  =
    begin
      match stk with
        | [] -> failwith "~empty stack~"
        | i::is -> stk <- is
    end
  method show (p:('a -> string))  =
    "["^(String.concat "," (List.map p stk))^"]"
end

class stack_int =
object (self)
  val mutable stk = []
  method push (i:int) =
    begin
      stk <- i::stk
    end
  method pop =
    begin
      match stk with
        | [] -> failwith "~empty stack~"
        | i::is -> stk <- is
    end
  method top =
    begin
      match stk with
        | [] -> failwith "~empty stack~"
        | i::is -> i
    end
  method is_empty  =
    stk == []
  method size  = (List.length stk)
  method show  =
    "["^(String.concat "," (List.map string_of_int stk))^"]"
end

let test_prog () =
  let test_stk = new stack_int in
  test_stk#push 5;
  test_stk#push 2;
  test_stk#push 6;
  print_endline (test_stk#show);
  test_stk#pop;
  print_endline (test_stk#show);
  test_stk#push 1;
  test_stk#push 8;
  let stk1 = new stack in
  stk1#push 2;
  let stk2 = new stack in
  stk2#push "aa";
  print_endline (test_stk#show);
  print_endline ("size of stack is "^(string_of_int (test_stk#size)));;

test_prog();;

module Mstack_int =
struct
  type t = (int list) ref
  let make () : t = ref []
  let push (stk:t) (i:int) : unit =
    stk := i::!stk
  let pop (stk:t) : unit =
    begin
      match !stk with
        | [] -> failwith "stack is empty"
        | _::s -> stk := s
    end
  let top (stk:t) : int =
    match !stk with
      | [] -> failwith "stack is empty"
      | i::s -> i
  let is_empty (stk:t): bool =
    !stk==[]
  let size (stk:t) : int  = (List.length !stk)
  let show (stk:t) : string =
    "["^(String.concat "," (List.map string_of_int !stk))^"]"
end

module type Mstack_int_ADT =
sig
  type t
  val make : unit -> t
  val push : t -> int -> unit
  val pop : t -> unit
end;;

module M2 = (Mstack_int : Mstack_int_ADT) ;;

module M = Mstack_int;;

let test_prog2 () =
  let test_stk =  M.make () in
  M.push test_stk 5;
  M.push test_stk 2;
  M.push test_stk 6;
  print_endline (M.show test_stk);
  M.pop test_stk;
  print_endline (M.show test_stk);
  M.push test_stk 1;
  M.push test_stk 8;
  print_endline (M.show test_stk);
  print_endline ("size of stack is "^(string_of_int (M.size test_stk)));;

test_prog2();;

module type SHOW_TYPE =
sig
  type t
  val show : t -> string
end;;

module INT_Elem : SHOW_TYPE =
struct
  type t = int
  let show (i:t) = string_of_int i
end;;

(* ~functor that takes a module of the show type~ *)
module FStack =
    functor (Elt: SHOW_TYPE) ->
struct
  type elem = Elt.t
  type t = elem list ref
  let show (stk:t) : string =
    "["^(String.concat "," (List.map Elt.show !stk))^"]"
end;;
\end{lstlisting}
\end{document}
