\documentclass{slnotes}
\begin{document}
\chapter{Design of studies}
Studies are broadly categorised into experimental or observational. In experimental studies, the researcher typically has control over some variables.

When the rate of A changes when B changes, then we say there is an association between A and B.

If X has an association between both parameters (exposure and response) being studied in a study, then X is a confounder. Slicing is commonly used to deal with confounders.

If there are variables affecting either the exposure or response variable in a study (but not both), these are extraneous variables that must be controlled (i.e. held constant between e.g. the control and treatment groups); otherwise, bias is introduced.

Randomised assignment is usually used in experimental studies to separate the experimental subjects into treatment and control groups. When the number of subjects is large, this creates two groups that are generally similar in all aspects, reducing bias. Blinding of the subjects or researchers or both can be done to further reduce bias.

Simpson's paradox occurs when a trend that appears in aggregate data reverses or disappears when the data is stratified into subgroups (i.e. slicing).

Another way to control for confounders is by adjusting the exposure variable for them (see UCB example).
\chapter{Association}
A linear association between two variables can be quantified using a correlation coefficient.

The ecological fallacy is committed when one deduces an association about individuals using aggregated data. The atomistic fallacy is the reverse.

When the range of data is restricted, the correlation tends to become smaller due to the attenuation effect.
\chapter{Measurement}
In general, there are three types of variables: categorial nominal, categorial ordinal and numerical.

The accuracy of a measurement can be described in terms of validity and reliability. A valid measurement is one that actually measures what it claims to measure. A reliable measurement is one that is consistent with repeated measurements.

The true score theory states that a measurement is the sum of the true value, random error, and systematic error.
\chapter{Sampling}
Sampling plans are generally divided into probability and non-probability sampling plans.

Probability sampling plans include simple random sampling, systematic sampling, stratified sampling, multistage sampling, and cluster sampling.

Non-probability sampling plans include convenience sampling, judgement sampling, quota sampling, and volunteer sampling. These could potentially cause large biases in the sample.

An estimate of a parameter usually invludes random error and bias. Sources of bias include response bias, non-response bias, and selection bias. Random error is generally reduced with a larger sample size.
\chapter{Risks and odds}
The risk of X in a population is defined as the rate of X in a population. The risk ratio of X between two populations is the ratio of the risks of X in the two populations.

Probability samples allow the accurate estimation of population risks and risk ratios, provided the sample is representative of the population (i.e. cohort but not case-control studies).

The odds of X in a population is defined as the rate of X in a population over the rate of not-X. The odds ratio is defined similarly to the risk ratio.

The odds ratio (but not purely odds!) can be estimated from a case-control study.
\end{document}
