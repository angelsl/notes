\documentclass{slnotes}
\newcommand{\scn}[1]{\textit{#1}}
\begin{document}
\chapter{Assorted definitions}
\sldef{Abiotic} components of the environment are the non-living components including the latitude, altitude, climate, soil, and disturbance.

\sldef{Biotic} components of the environment are its living components.

\sldef{Habitat} refers to the locality, site, and particular type of local environment occupied by an organism.

\sldef{Keystone species} are those that have a much greater effect on a biological community in proportion to their individual biomass.
\chapter{Singapore}
Singapore's human population is about 5.9 million with a density of 8274 per square kilometre. In 1819, Singapore was covered by natural habitats, but now, 95\% of our forests, 98\% of our mangroves, and 60\% of our coral reefs have been lost, mainly due to urbanisation causing habitat modification and loss.

But even then, new species are still being discovered, and some habitats have escaped human attention. E.g. a new mangrove tree species \scn{Bruguiera hainesii} was discovered, and \scn{Bruguiera sexangula} was rediscovered, in 2005. The blackwater mud snake/spotted water snake \scn{Phytolopsis punctata} and smooth slug sname \scn{Asthenodiapsas laevis} was discovered from Nee Soon Swamp Forest in 2014.

\sldef{Tanjong Chek Jawa}, on the south-eastern tip of Pulau Ubin, was discovered by accident in the early 2000s during a nature outing, just as the government was about to reclaim the area. The government agreed to postpone the reclamation of the area.

\sldef{Keppel Hill Reservoir}, off Telok Blangah Road, was re-discovered in 2014.

Singapore has 2145 native vascular plant species, 25 bat species, 392 bird species, 122 dragonfly species, more than 800 spider species, 35 true mangrove plant species, 12 seagrass species, 255 hard coral species, 50 sea anemone species, 200 sponge species, and more.

Singapore has exotic species that were purposefully brought in, e.g. guppies to control malaria, tilapias to supplement the protein supply, and ornamental plants to decorate roadsides, and also some that were accidentally introduced e.g. the Giant African land snail and American and German cockroaches.

Exotic species have an impact on indigenous species. E.g. the changeable lizard \scn{Calotes versicolor} displaced the indigenous species \scn{Bronchocela cristella}. The red ear slider \scn{Trachemys scripta elegans} which was released into the wile competed with native species for food and shelter. It carries parasites.

The \sldef{Lee Kong Chian Natural History Museum} is a museum of natural history at the Kent Ridge Campus of the National University of Singapore. Officially opened on 18 April 2015, it houses the Raffles Natural History Collection.
\chapter{Intertidal habitats}
The intertidal habitat is the part of the shore that is submerged at high tide and exposed at low tide. It is also known as the littoral zone. The periodic change in tide causes great variation in the environmental conditions of the habitat.

The habitat is harsh and stressful on organisms for this reason, and also due to plastic wastes, and climate change and associated factors like ocean acidification, rising sea levels, and rising ocean temperatures.

Despite this, it is rich in biodiversity because there is an abundance of nutrients from the land and sea, effective mixing of nutrients, usually strong solar energy, a high concentration of dissolved oxygen, closer contract between producers and consumers, and a diversity of micro-habitats and habitat complexity.

\sldef{Types of intertidal habitats} include mudflats, seagrass habitats, sandflats, mangroves, reef flats, rocky shores, and algal beds.

Intertidal plants generally consist of seaweeds and seagrasses.

\sldef{Seaweeds} (i.e. macroalgae) include species like Mermaid's fan seaweed \scn{Padina} spp., coin seaweeds \scn{Halimeda} spp., \scn{Sargassum} spp., \sldef{coralline algae}, sea grapes \scn{Caulerpa} spp., and sea lettuce \scn{Ulva} spp.

\sldef{Seagrasses} are the only group of flowering plants that are fully submerged in water. Unlike algae, they have roots, stems (that also go underground), leaves that project above the seabed, and flowers and fruits with seeds. They usually develop in shallow coastal waters like sandflats, and best develop in warm, tropical waters, forming extensive meadows.

There are about 50 species of seagrass worldwide, 20 in SEA, and 12 in Singapore.

\sldef{Seagrass beds} support a high diversity of species. Their leaf beds provide an extensive surface area that promotes the growth of epiphytes, and shelter and food for animals. Thir roots also help to stabilise the substrate (seabed). Some endangered species that are supported by seagrass systems include the green sea turtle \scn{Chelonia mydas}, olive ridley sea turtle \scn{Lepidochelys olivacea}, loggerhead sea turtle \scn{Caretta caretta}, flatback sea turtles \scn{Chelonia depressa}, hawksbill sea turtle \scn{Eretmochelys imbricata}, wart snake \scn{Acrochordus granulatus} and the dugong \scn{Dugong dugon}, which is fully dependent on seagrass.

Animals that survive in an intertidal habitat must prevent themselves from being swept away by the incoming and outgoing tide, and protect themselves from drying out and against solar radiation.

Meiofauna, which are benthic animals that are larger than microfauna but smaller than macrofauna, are generally found within the sediment. They include species like sand anemones (mainly in sandy habitats), barnacles, oysters (which attach to rocks), and giant clams (which bore into rock for protection).

Burrowing animals play an important role as burrows provided effective shelter and protection. The mud lobster \scn{Thalassina anomala}'s nocturnal burrowing is important for the recycling of nutrients in mangrove ecosystems, and it is a keystone species of the mangrove habitat.

The mudskipper is a semi-terrestrial fish that can live above water for an extended period provided they are kept moist; they have modified pectoral fins that help them move over ground. They are common in mangroves and mudflats.

The sand crab \scn{Scopimera} spp. has membranes on each leg for the exchange of air.

Other species include the sand tubeworm, sea cockroach \scn{Ligia} spp., hermit crabs, marine snails, top shells \scn{Trochus} spp., black-lipped conch \scn{Strombus urceus}, cowries, spider conch \scn{Lambis lambis}, xanthid crabs, pilumnid crab \scn{leelumnus radium}, leucosid crabs, red egg crab \scn{atergatis integerrimus}, velcro crab \scn{camposcia retusa} (so named because of hooked hairs all over its body, which attaches living sponges, seaweeds, etc) (i.e. decorator crab), fiddler crab \scn{Uca} spp. (sexual dimorphism: males have a large major claw), brittle stars, knobby sea star \scn{protoreaster nodosus}, sea cucumbers, sea urchins, frogfishes (a type of anglerfish).
\chapter{Subtidal habitats}
Benthic habitats are formed by the seafloor/seabed; they can be muddy, sandy or rocky, with the former two called \sldef{soft bottom habitats}. Specialised equipment is needed to study seabed habitats, like grabs, dredges or sledges for shallow waters, and submarines etc. for deep sea exploration.

Common benthic organisms include sea fans, sea pens \scn{Pteroeides} spp., noble volute \scn{Cymbiola nobilis}, sponges, baler shell \scn{melo melo} and basket star \scn{gorgonocephalus caputmedusae}.

Benthic habitats are important as benthic animals help in \sldef{bioturbation} (the reworking of soils and sediments by animals or plants), making use of accumulated nutrients and contributing to the food chain (by being eaten).

Singapore waters are generally shallow, and average 20 metres in depth.

Common species include \scn{Sargassum} spp., diatoms (single-celled algae), dinoflagellates (plankton) (blooms can cause red tide; they contain poisons that can be carried by shellfish e.g. saxitoxin), copepods (small crustaceans), rays, sharks, bony fishes, marine turtles like the hawksbill sea turtle \scn{Eretmochelys imbricata}, loggerhead sea turtle \scn{Caretta caretta}, green sea turtle \scn{Chelonia mydas}, leatherback turtle \scn{Dermochelys coriacea}, sea snakes \scn{Laticauda colubrina}, Indo-Pacific humpbacked dolphins \scn{Sousa chinensis}, bottlenose dolphins \scn{Tursiops} spp., Whalebone whale (i.e. baleen whales), false killer whale \scn{pseudorca crassidens}, sperm whale \scn{physeter macrocephalus}, dugong \scn{Dugong dugon}, jellyfishes, flower crabs \scn{portunus pelagicus}, mosaic reef crab \scn{lophozozymus pictor}, stone fish \scn{Synanceia horrida}, razor fish, cone shell \scn{Conus} spp.
\section{Coral reef habitats}
Of the 255 coral species known from Singapore reefs, two \scn{Seriatopora hystrix} and \scn{Stylophora pistillata} (appeared in 2006 and disappeared after 2010 bleaching, now unknown) are believed to be locally extinct.

Common coral species include \sldef{branching corals}, \sldef{foliose corals}, \sldef{boulder corals} and \sldef{encrusting corals}.

Intertidal reef flats consist only of species that can tolerate periodic exposure to air. Common species include \scn{Platygyra} spp. and \scn{Favites} spp.

Common species at reef slopes include \scn{Diploastrea} spp. and \scn{Turbinaria} spp.

Plants associated to reefs include \scn{Padina} spp., \scn{Halimeda} spp., \scn{Ulva} spp. and \scn{Sargassum} spp.

Animals associated to reefs include tunicates, sea urchins, nudibranches, giant clams, soft corals, hydroids, sponges, molluscs, sea fans, crustaceans, giant carpet anemones, clownfishes (protandrous hermaphrodites).

Sisters Islands, the first marine park in Singapore, opened in 2015. It is about 40 hectares and includes corals from Pulau Semakau. It is part of the Singapore Blue Plan.
\section{Threats to subtidal habitats}
Subtidal habitats are threatened by habitat loss, modification and degradation due to oil rigs, artificial barriers (e.g. sea walls), risks from toxic chemical and oil spills, threats from invasive species, loss of species, and climate change.
\chapter{Primary vegetation}
A \sldef{primary forest} is a forest in the state of primary vegetation. They are now rare in Singapore (only 279 ha, from the whole island of 539 square km to now about 0.52\% of that), in small, isolated patches.

\sldef{Vegetation} refers to all the plant life in a particular area.

A \sldef{forest} refers to a large area covered by trees.

A \sldef{tree} is a large woody plant with a single main stem or trunk.

\sldef{Synusiae} refer to plant life forms with similar ecological requirements.

\sldef{Autotrophs} are (green) plants that can photosynthesise.

\sldef{Heterotrophs} are (non-green) plants that cannot photosynthesise. They consist of \sldef{saprophytes} that obtain nutrients from rotting mater and \sldef{parasites} that obtain all nutrients from other plants.

\sldef{Mechanically independent} or erect plants are plants that stand on their own; they consist of woody, single-trunked (trees/treelets) or multiply-trunked (shrubs) plants, and non-woody plants i.e. herbs.

\sldef{Mechanically dependent} plants are plants that cannot stand on their own. They consist of \sldef{hydrophytes} that grow in water, \sldef{climbers} that lean or climb on another plant/support, \sldef{creepers} that grow on the ground, \sldef{epiphytes} that grow on another plant's stem or branches, \sldef{hemi-epiphytes} that grow as epiphytes then send roots to the ground, \sldef{epiphylls} that grow on another plant's leaves, and \sldef{hemi-parasites} (half-parasite as it still photosynthesises) that grow into the stem of another plant to extract water and nutrients from that plant.

\sldef{Stranglers} are hemi-epiphytes that strangle the host's trunk, killing its host to take over its space in the community.

Examples of epiphylls include mosses, liverworts, and algae. Examples of hemi-parasites include the mistletoe. Examples of hemi-epiphytes include strangling figs \scn{Ficus} spp. Examples of saprophytes (i.e. mycoheterotroph) include \scn{Thismia aseroe} (probably extinct in SG). Examples of parasites include \scn{Rafflesia arnoldii}, the largest flower in the world. It is not present in Singapore.

The \sldef{canopy} refers to the above-ground part of a forest. Canopies consist of \sldef{strata} (singular \sldef{stratum}) which are distinct layers.

Primary dryland forests have the most strata, at 5, due to high diversity, complexity of structure and great height. Temperate forests have fewer strata. Secondary forests have 1 to 2 strata.

The 5 strata of a primary forest are named the emergent, main canopy, subcanopy, treelet or shrub layer, and forest floor.

\sldef{Primary dryland forest} is forest that grows on ground that is not wet most of the year. It is usually what is meant by a "tropical rainforest". It originally covered 82\% of Singapore's land, but now covers only 192 ha, 20\% of which is in BTNR and 80\% in CCNR.

\sldef{Bukit Timah Nature Reserve} has a total area of 192.6442 ha, of which 38.3 ha is primary forest. It is home to about 2000 plant species, several thousand animal species, but no large mammals. The tiger was formerly the top predator, but it is now insect-dominated.

\sldef{Central Catchment Nature Reserve} has a total area of 3043.1 ha (including the man-made reservoir areas), of which 153.6 is primary forest. Its biodiversity is similar to that of BTNR. One animal that stays in CCNR is the sambar deer.

Primary dryland forests in Singapore are dominated by \scn{Meranti} spp., which are dipterocarps. Thus the forests are called dipterocarp forests. BTNR is a hill dipterocap forest, while CCNR is a lowland dipterocarp forest.

\sldef{Dipterocarps} (Dipterocarpaceae) are so-named because of their two-winged fruit. The seraya \scn{Shorea curtisii} is a dipterocarp.

Other than those, other common tree families include the bean family (Fabaceae), the Chinese olive family (Burseraceae), Chiku family (Sapotaceae), mango family (Anacardiaceae), etc.
\section{Swamp forest}
\sldef{Freshwater swamp forest} is forest that grows on ground that is temporarily to semi-permanently inundated by acidic, mineral-rich freshwater with water level fluctuations through periodic drying of the soil. In Singapore, this occurs mostly in Nee Soon Swamp Forest (87 ha). Due to the waterlogged and unstable soil, trees develop adaptations similar to those of mangrove plants, like prop and stilt roots, and kneed breathing roots.

Swamp forests have the highest diversity of native freshwater organisms found nowhere else, e.g. the entire world's population of swamp forest crab \scn{Parathelphusa reticulata}, which is endemic to Nee Soon Swamp Forest.

Common animals in primary forests include the reticulated python (now the top predator), sambar deer, banded leaf monkey, cream-coloured giant squirrel, Malayan colugo (i.e. flying lemur), long-tailed macaque (carries Herpes B), and asian/common palm civet.
\section{Conservation and value}
Pacific yew \scn{Taxus brevifolia} was the origin of paclitaxel, now used in chemotherapy for lung, ovarian and breast cancers.

The BKE separated BTNR from CCNR, which isolated the two habitats, but a bridge has been constructed to allow animals to cross the BKE to go between the two areas.

Some measures to conserve primary forests include studying organisms and ecosystems, managing areas to support pollinators and dispersers, propagating forest plants, re-introducing dispersers, protecting keystone species, surveilling for and punishing poachers and vandals, minimising the impact of construction works, avoiding the introduction of exotic species, protecting the area from lightning, and forbidding smoking in the area.
\chapter{Secondary vegetation}
\sldef{Secondary vegetation} is that which has regrown after the destruction of the original, primary, vegetation, by natural or human causes.

When primary vegetation is disturbed, it becomes secondary vegetation. With time, it can recover back to primary vegetation. However, if secondary vegetation is removed, then it becomes wasteland.

\sldef{Ecological succession} describes the process of continuous, unidirectional change in vegetation. In Singapore, if undisturbed for at least several decades, priamry forest which is cleared becomes Trema belukar (if soil is undegraded) or Adinandra belukar (if soil is degraded by agriculture). The belukar become tall secondary forest, if seed sources are available, and finally primary forest, again if seed sources are available.

Secondary forests are the most common type of forest in Singapore and in BTNR and CCNR, covering about 4\% of its area in large continuous patches.

Some examples of crops planted in Singapore include gambier, rubber, pineapple, pepper and tapioca.

\sldef{Trema belukar} is secondary forest on undegraded soil, dominated by the lesser trema \scn{Trema cannabina} and rough trema \scn{Trema tomentosa}. It occurs when large gaps form in the canopy due to natural or human causes.

\sldef{Adinandra belukar} is species-poor secondary forest on degraded soil, dominated by tiup tiup \scn{Adinandra dumosa}. Its soil is low in nutrients (nitrogen and phosphorus) and acidic (pH 3.3 to 3.9).

Species in Adinandra belukar include the tiup tiup (cream-white flower), simpoh air (yellow flower; fruits split open into star-shaped seeds under red flesh), tembusu (cream-white to yellow flower; green turning orange-red when ripe fruit), tropical pitcher plants, common acacia (yellow, long pendulous flower; fruits are green pods that split open when mature), albizia, silverback (silver-backed leaf), ant plant \scn{Macaranga gigantea} (leaves are large), sendudok and fig.

The tiup tiup and simpoh air are pollinated by carpenter bees. The tiup tiup's fruit is spread by the lesser dog-faced fruit bat. The simpoh air, silverback and common acacia's fruit are spread by the yellow-vented bulbul.

Acacia have flattened and widened stalks called phyllodes.

Other species include the leafy liverwort, and mosses.
\section{Tall secondary forest}
\sldef{Tall secondary forest} is forest that succeeds Adinandra or Trema belukar. It is mostly found in CCNR, BTNR, Labrador Nature Reserve (~10 ha), and Botanic Gardens' Jungle (~3 ha). They consist of fewer species than primary forest but much more than both belukar. Trees are medium- to tall-sized.

Common species in tall secondary forest include myrtle family (silverback), mangosteen family (bintangor, wild kandis), custard apple family (Cyathocalyx, Xylopia), laurel family (Litsea, shiny laurel), coffee family (silver timon, wild randa), rubber family (mahang), etc.

Abandoned rubber plantations in Ubin, Tekong and the western reservoirs are still not at the stage of tall secondary forest as they are still dominated by rubber trees.
\section{Spontaneous secondary vegetation}
A secondary freshwater swamp forest exists in CCNR. It is species-poor compared to primary freshwater swamp forest.

Secondary vegetation exists in reservoirs (which are man-made), ponds, canals and drains. Species include the hydrilla, lotus, and water lily.
\section{Animals in secondary forest}
Some animals in secondary forests include the lesser dog-faced fruit bat, Malayan pangolin, plaintain squirrel, long-tailed macaque, reticulated python, and golden orb spider (sexual dimorphism: male is tiny compared to female).
\chapter{Wasteland and reclaimed land}
\section{Reclaimed land}
Reclaimed land is that which has been reclaimed from the sea by dumping subsoil from inland areas, or marine clay or marine sand from seabeds. It is rather uncommon in the world.

About 25\% of Singapore's current land area is reclaimed land. Reclamation started in 1820, with major reclamations from 1961 on. Some areas that have been reclaimed include Pulau Tekong, Southern Islands, NSRCC Golf Course at Changi, and Gardens by the Bay.

Subsoil, from inland excavations and has been exausted, is clay-like, compacted and retains water, leading to waterlogging. Marine sand, from Singapore seabeds or other (SEA) countries is sandy, loose (i.e. well-aerated) and drains water (never floods). Marine clay is intermediate.

Environmental conditions are similar to coastal habitats, since reclamation happens at the coastline. There is high light intensity, wind speeds and maximum temperatures, but low relative humidity. The habitat experiences salt sprays.

\sldef{Xerophytes} are plants that grow in dry conditions. \sldef{Xeromorphs} have characteristics of a xerophyte.

Adaptations to dry conditions of the coast which help to reduce water loss include tiny/modified leaves, thick-walled epidermal cells, thick cuticles, and good stomatal control. Some plants also have water storage cells.

Reclaimed land vegetation is a form of secondary vegetation. They may grow spontaneously or be managed.

Spontaneous plants that grow on (marine) sand-filled areas are coastal forest species adapted to growing on sand. Such species that are native include the casuarina, coconut, sea lettuce, seashore morning glory, saga \scn{Adenanthera pavonina}, sea apple/jambu air laut \scn{Syzygium grande}, screw pine \scn{Pandanus odorifer} and paku laut \scn{Acrostichum aureum}. There are also exotic species like the albizia \scn{Falcataria moluccana}, common acacia, leucaena, etc.

Spontaneous plants that grow on subsoil-filled areas are adapted for growing on poorly-drained (flooded) soil. They are usually exotic species like albizia, common acacia, giant sensitive plant \scn{Mimosa pigra}, and leucaena.

Animals in reclaimed land areas are usually those found in urban areas, including birds like the white-headed munia, brown shrike, Javan myna and house crow, frogs and toads like the crab-eating frog, banded bullfrog and Asian toad, land hermit crabs, and lizards like the garden supple skink and changeable lizard.
\section{Wasteland}
\sldef{Wasteland} is vacant urban land awaiting development, including construction sites for expanding new towns.

Wasteland plants are mostly exotic weeds from South America, like the African tulip, albizia, common acacia, Malayan cherry, and mile-a-minute (fruit looks like shuttlecock).

Animals that in wastelands are mostly similar to those in reclaimed land, with the addition of the green crested lizard, spitting cobra \scn{Naja sumatrana}, red jungle fowl \scn{Gallus gallus} and king cobra \scn{Ophiophagus hannah}.
\chapter{Coastal vegetation}
\sldef{Coastal vegetation} is vegetation that grows on land at the edge of the sea, including mangrove forest, sandy beach vegetation, and rocky shore or cliff vegetation.

Coasts in primeval Singapore were mostly fringed by mangroves, covering about 13\% of Singapore. Today, there is only 659 ha of mangrove, the majority being at Sungei Buloh, Sungei Besar-Puaka-Jelutong on Ubin, and the islands Tekong, Semakau, Pawai and Senang.
\section{Mangrove forest}
Mangrove forests exist between the high-tide to mid-tide levels, where there are fine sediment deposits. Salinity varies with the tides (seawater is hyperosmotic). The substrate is anaerobic and unstable. Other environmental conditions are similar to other tropical forest types.

Adaptations for high salinity include salt secretion via glands (e.g. api api, sea holly), salt ultrafiltration (e.g. bakau, bruguiera, lumnitzera, perepat).

Adaptations for unstable substrate include prop or stilt roots (bakau), and plank roots (nyireh, dungun).

Some plants are vivipary (bakau) or cryptovivipary (api api, kacang-kacang, nipah palm). Vivipary means the embryo grows and breaks out of the seed coat and fruit wall while attached to the parent plant. Cryptovivipary means the same, except it remains within the fruit wall.

Adaptations for anaerobic substrate include breathing roots like pencil/cone roots (api api, perepat), kneed roots (Bruguiera, tengar), plank roots (nyireh, dungun), and prop or stilt roots (bakau).
\section{Sandy beaches}
There are few present-day natural sandy beaches in Singapore, as most were reclaimed. The remaining ones are at islands Ubin and Tekong, Changi Beach, etc. Present-day artificial beaches include ECP, Pasir Ris Park, Pulau Seletar, Pulau Serangoon, etc.

Beach vegetation succession happens in two stages: Pes-caprae association, formed mostly by non-woody plants, followed by Barringtonia association, formed by woody plants.

Plants on beaches include the seashore morning glory \scn{Ipomoea pes-caprae} (goat's foot-shaped leaf), morning glory \scn{Ipomoea cairica}, sea putat (bat-pollinated, water-dispersed).
\section{Rocky shore and cliffs}
Cliffs and rocky shores were rare in primeval Singapore, and even rarer today. The remaining ones today include those below Fort Siloso on Sentosa, and on islands Biola, St. John's, Lazarus, Tekukor and Salu.

Plants that grow on cliffs and rocky shores must be able to tolerate the harsh conditions. They include mentigi, superb fig, sea purselane, pelir musang and sea teak.
\section{Animals in coastal vegetation}
Animals in coastal vegetation must be able to cope with the harsh conditions. All three kinds of coastal vegetation have similar animals.

Molluscs in such vegetation include the telescope shell, red berry snail, mangrove slug, mangrove periwinkle, and common nerite.

Crabs in such vegetation include the face-banded sesarmine, tree-climbing crab, and fiddler crab.

The mud lobster is also present, and is a keystone species in the mangrove forest. Ants, mound crabs, mud shrimp, and file snake live in mud lobster mounds.

Fishes in mangroves include the archer fish and mudskipper.

Reptiles in coastal vegetation include the water monitor, dog-faced water snake, and estuarine crocodile.
\end{document}
