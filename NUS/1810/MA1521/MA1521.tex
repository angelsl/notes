\documentclass{slnotes}
\DeclarePairedDelimiter\abs{\lvert}{\rvert}%

\begin{document}
\chapter{Differentiation, Integration}
\section{Limits}
If \(L = \lim f(x) \land M = \lim g(x)\) then \begin{itemize}
\item \(\lim kf(x) = kL\); \(\lim (f(x) \pm g(x)) = L \pm M\)
\item \(\lim f(x)g(x) = LM\)
\item \(\lim f(x)/g(x) = L/M\) if \(M \neq 0\)
\item \(\lim_{x\to0} (\sin x)/x = 1\)
\item \(\sum^\infty_{n=0} 1/n! = e\); \(\lim_{n\to\infty} (1+1/n)^n = e\)
\end{itemize}

\sldef{L'Hôpital's rule}. For functions \(f, g\) differentiable on an open interval \(I\) except possibly at a point \(c \in I\), if \(\lim f(x) = \lim g(x) = 0\) or \(\pm\infty\), and \(g'(x) \neq 0\) for all \(x \in I\), \(x \neq c\), and \(\lim f'(x)/g'(x)\) exists, then \(\lim f(x)/g(x) = \lim f'(x)/g'(x)\).

\section{Derivatives}
\begin{itemize}
\item \((kf(x))' = kf'(x)\); \((f(x) \pm g(x))' = f'(x) \pm g'(x)\)
\item \((f(x)g(x))' = f'(x)g(x) + f(x)g'(x)\)
\item \((f(x)/g(x))' = (f'(x)g(x)-f(x)g'(x))/((g(x))^2)\)
\item \(\sldd{y}{x} = \sldd{y}{u} \cdot \sldd{u}{x} = \sldd{y}{t}/\sldd{x}{t}\)
\item \((e^x)' = e^x\); \((\ln x)' = 1/x\)
\item \((\sin x)' = \cos x\); \((\sin^{-1} x)' = 1/\sqrt{1-x^2}\)
\item \((\cos x)' = -\sin x\); \((\cos^{-1} x)' = -1/\sqrt{1-x^2}\)
\item \((\tan x)' = \sec^2 x\); \((\tan^{-1} x)' = 1/(1+x^2)\)
\item \((\csc x)' = -\csc x \cot x\); \((\csc^{-1} x)' = -1/\abs{x}\sqrt{x^2-1}\)
\item \((\sec x)' = \sec x \tan x\); \((\sec^{-1} x)' = 1/\abs{x}\sqrt{x^2-1}\)
\item \((\cot x)' = -\csc^2 x\); \((\cot^{-1} x)' = -1/(1+x^2)\)
\item \((x\ln x - x)' = \ln x\); \((\ln\sec x)' = \tan x\)
\item \((x\sin^{-1}x + \sqrt{1-x^2})' = \sin^{-1}x\)
\item \((x\cos^{-1}x - \sqrt{1-x^2})' = \cos^{-1}x\)
\item \((x\tan^{-1}x-\frac{1}{2}\ln(1+x^2))' = \tan^{-1}x\)
\item \((\ln(\sec x + \tan x))' = \sec x\)
\item \((-\ln(\csc x + \cot x))' = \csc x\)
\item \((\ln\sin x)' = \cot x\)
\item \(f_x = \frac{\partial}{\partial x} f(x, \hdots)\); \(f_{xx} = \frac{\partial^2}{\partial x^2} f(x, \hdots)\); and so on
\item For \(z(t) = f(x(t), y(t))\), \(\sldd{}{t} z(t) = f_x \sldd{}{t} x(t) + f_y \sldd{}{t} y(t)\)
\item For \(z(s, t) = f(x(s, t), y(s, t))\), \(z_s = f_xx_s + f_yy_s\) and \(z_t = f_xx_t + f_yy_t\)
\item \(\nabla f = f_x\mathbf{i} + f_y\mathbf{j} + f_z\mathbf{k}\); \(D_{\mathbf{u}}f(a, b) = \nabla f(a, b)\cdot\mathbf{u}\) (\(\lvert\mathbf{u}\rvert = 1\))
\item In MA1521, \(f_{xy} = f_{yx}\)
\end{itemize}

\section{Integrals}
\begin{itemize}
\item \(\frac{\sld}{\sld x}\int^x_a f(t)\,\sld t = f(x)\); \(\int^b_a f(x)\,\sld x = F(b) - F(a)\) where \(F\) is an antiderivative of \(f\) on \([a, b]\)
\item \(\int^b_a f(g(x))g'(x)\,\sld x = \int^{g(b)}_{g(a)}f(u)\,\sld u\), provided \(g' \ge 0\) or \(g' \le 0\) in \([a, b]\).
\item \(\int u\,\sld v = uv - \int v\,\sld u\); choose \(u\) as the harder-to-integrate function: \(\log\), \(\sin^{-1}\) et al., algebraic, \(\sin\) et al., and exponential, in decreasing order.
\item Volume about \(x\)-axis \(= \pi\int^b_a[f(x)]^2\,\sld x\)
\item Volume about \(y\)-axis \(= 2\pi\int^b_a x\lvert f(x) \rvert\,\sld x\)
\end{itemize}

\sldef{Extreme values} occur at interior points where \(f'(x) = 0\) or does not exist, and domain endpoints. \sldef{Critical points} are interior points where \(f'(x) = 0\) or does not exist, and where \(f_x(a, b) = f_y(a, b) = 0\), or either \(f_x(a, b)\) or \(f_y(a, b)\) does not exist.

A graph is \sldef{concave down} on an interval if its shape looks like the graph of \(y = -x^2\) i.e. \(y'' < 0\), and \sldef{concave up} if it looks like \(y = x^2\) i.e. \(y'' > 0\). \sldef{Points of inflection} are points where \(f\) is continuous and its concavity changes.

\sldef{First derivative test}. If \(f'(x) > 0\) for \(x \in (a, c)\) and \(f'(x) < 0\) for \(x \in (c, b)\) then \(f(c)\) is a local maximum. If \(f'(x) < 0\) for \(x \in (a, c)\) and \(f'(x) > 0\) for \(x \in (c, b)\) then \(f(c)\) is a local minimum.

\sldef{Second derivative test}. If \(f'(c) = 0 \land f''(c) < 0\), then \(f\) has a local maximum at \(x = c\). If \(f'(c) = 0 \land f''(c) > 0\), then \(f\) has a local minimum at \(x = c\).

For functions of two variables, let \(D = f_{xx}(a,b)f_{yy}(a,b) - f_{xy}(a,b)^2\). If \(D > 0 \land f_{xx}(a, b) > 0\), the point is a local minimum. If \(D > 0 \land f_{xx}(a, b) < 0\), it is a local maximum. If \(D < 0\), it is a saddle point. If \(D = 0\), there is no conclusion.

\(f\) is \sldef{increasing} on an interval \(I\) if for any two points \(x_1, x_2\) in \(I\), \(x_2 > x_1 \implies f(x_2) > f(x_1)\). If \(x_2 > x_1 \implies f(x_2) < f(x_1)\), then \(f\) is \sldef{decreasing} on \(I\).

\sldef{Tests}. \(f\) is increasing on \(I\) when \(f'(x) > 0\) for all \(x \in I\). \(f\) is decreasing on \(I\) when \(f'(x) < 0\) for all \(x \in I\).

\chapter{Differential equations}
\sldef{Separable equations} are those of the form \(M(x)\sld x = N(y)\sld y\). They can be integrated directly.

Equations of the form \(y' = g(y/x)\) can be made separable by \(u = y/x\) and substituting \(y = ux\), \(y' = u+xu'\) to get \((g(u) - u)^{-1}\sld u = x^{-1}\sld x\).

Equations of the form \(y' = f(ax + by + c)\) where \(f\) is continuous and \(b \neq 0\) can be solved by substituting \(u = ax + by + c\).

Equations of the form \(\sldd{y}{x} + P(x)y = Q(x)\) have general solution \(y = R^{-1}\int RQ\,\sld x\) where \(R = \exp \int P\,\sld x\).

Equations of the form \(y' + P(x)y = Q(x)y^n\) can be reduced to the previous form by substituting \(z = y^{1-n}\) to get \(z' + (1-n)P(x)z = (1-n)Q(x)\).

\section{Second order DEs}
An equation of the form \(y'' + ay' + by = 0\) has a characteristic equation \(\lambda^2 + a\lambda + b = 0\) with roots \(\lambda_1, \lambda_2\).

If \(\lambda_1, \lambda_2\) are distinct and real, then the solution is\\\(y = c_1\exp(\lambda_1x)+c_2\exp(\lambda_2x)\).

If the roots are repeated, then the solution is\\\(y = (c_1+c_2x)\exp(-\frac{1}{2}ax)\).

If the roots are complex, then if \(\lambda_1 = \alpha+\beta i\), \(\lambda_2 = \alpha-\beta i\), the solution is\\\(y = c_1\exp(\alpha x)\cos(\beta x)+c_2\exp(\alpha x)\sin(\beta x)\).

\section{Modelling}
The Malthus model states that \(N' = (B-D)N\) i.e. \(N = N_0\exp((B-D)t)\) where \(N\) is the population, \(B\) is the birth rate and \(D\) is the death rate.

The logistic model states that \(D = sN\), so we have \(N' = BN - sN^2\) i.e. \[\frac{1}{N} = \frac{s}{B} + (\frac{1}{N_0} - \frac{s}{B})e^{-Bt}\] and the long-term equilibrium population is \(B/s\).

The harvesting model states that \(N' = (B - sN)N - E\). To analyse this, consider \(N'' = (B-2sN)(BN-sN^2-E) = -s(B-2sN)(N-\beta_1)(N-\beta_2)\) where \(\beta_1 \le \beta_2\). If \(E > B^2/4s\), \(N\) will go towards zero. If \(0 < E < B^2/4s\), if \(N_0 < \beta_1\), \(N\) will go towards zero in \(T = \int^{N_0}_0 (sN^2-BN+E)^{-1}\,\sld N\), or if \(N_0 > \beta_1\), \(N\) will go towards \(\beta_2\). If \(E = B^2/4s\), if \(N_0 > B/2s\), \(N\) will tend towards the latter, else it will go towards 0.

\chapter{Series}
An expression of the form \(a_1 + a_2 + \cdots + a_n + \cdots\) is an \sldef{infinite series}. \(a_n\) is the \(n\)th term. The sequence \(s_1 = a_1\), \(s_2 = a_1 + a_2\), \(s_n = \sum^n_{k=1}a_k\) is the sequence of \sldef{partial sums} of the series, and \(s_n\) is the \(n\)th partial sum. If the sequence of partial sums converges to a limit \(L\), the series is convergent and its sum is \(L\).

A \sldef{geometric series} is one of the form \(a + ar + ar^2 + \cdots = \sum^\infty_{n=1}ar^{n-1}\) and its \(n\)th partial sum \(s_n = a(1-r^n)/(1-r)\). If \(\lvert r \rvert < 1\) then the series converges and its sum is \(a/(1-r)\).

A \sldef{power series} is one of the form \(c_0 + c_1x + c_2x^2 + \cdots = \sum^\infty_{n=0}c_nx^n\). Such a series can be said to converge for all \(x \in (a - h, a + h)\) and diverges elsewhere except possibly at \(x = a - h\) or \(x = a + h\); \(h\) could be zero, in which case the series converges only at \(x = a\), or \(h\) could be \(\infty\), in which case the series converges everywhere. \(h\) is known as the radius of convergence.

A power series is a function with the domain being the values of \(x\) for which the series converes. This function can be differentiated or integrated term-by-term.

\sldef{Ratio test}. Let \(\rho = \lim_{n\to\infty}\lvert a_{n+1}/a_n \rvert\). If \(\rho < 1\), the series converges; \(\rho > 1\), the series diverges; \(\rho = 1\), there is no conclusion.

\section{Taylor series}
At \(x = a\), for all \(x\) u.o.s.
\begin{itemize}
\item \(f(x) = f(a) + f'(a)(x-a) + \cdots = \sum^\infty_{k=0}\frac{f^{(k)}(a)}{k!}(x-a)^k\)
\item \(1/x = \frac{1}{a} - \frac{x - a}{a^2} + \frac{(x-a)^2}{a^3} - \cdots\)\\\(= \sum^\infty_{n=0}((-1)^n/a^{n+1})(x-a)^n\) for \(\lvert 1-x/a \rvert < 1\)
\item \(\frac{1}{1-cx} = \frac{1}{1-ac} + \frac{c(x-a)}{(1-ac)^2} + \frac{c^2(x-a)^2}{(1-ac)^3} + \cdots\)\\\(= \sum^\infty_{n=0}(c^n/(1-ac)^{n+1})(x-a)^n\)\\for \(\lvert c(a-x)/(ac-1)\rvert < 1\)
\item \(\frac{1}{1+cx} = \frac{1}{1+ac} - \frac{c(x-a)}{(1+ac)^2} + \frac{c^2(x-a)^2}{(1+ac)^3} - \cdots\)\\\(= \sum^\infty_{n=0}((-c)^n/(1+ac)^{n+1})(x-a)^n\)\\for \(\lvert c(a-x)/(ac+1)\rvert < 1\)
\item \(e^{cx} = e^{ac}(1 + c(x-a) + c^2(x-a)^2/2! + \cdots)\)\\\(= \sum^\infty_{n=0}(e^{ac}c^n/n!)(x-a)^n\)
\item \(\sin cx = \sin ac + c(\cos ac)(x-a) - c^2(\sin ac)(x-a)^2/2! - c^3(\cos ac)(x-a)^3/3! - \cdots = \sum^\infty_{n=0} (c^n\sin(ac+n\pi/2)/n!)(x-a)^n\)
\item \(\cos cx = \cos ac - c(\sin ac)(x-a) - c^2(\cos ac)(x-a)^2/2! + c^3(\sin ac)(x-a)^3/3! + \cdots = \sum^\infty_{n=0} (c^n\cos(ac+n\pi/2)/n!)(x-a)^n\)
\item \(\ln x = \ln a + (x-a)/a - (x-a)^2/2a^2 + (x-a)^3/3a^3 - \cdots = \ln a + \sum^\infty_{n=1} ((-1)^{n+1}/(na^n))(x-a)^n\)\\for \(\lvert 1-x/a \rvert < 1\)
\item \(\tan^{-1}x = x - x^3/3 + x^5/5 - \cdots\)\\\(= \sum^\infty_{n=0} (-1)^n x^{2n+1} / (2n+1)\) for \(\lvert x\rvert < 1\) at \(x = 0\)
\end{itemize}

\chapter{Identities}
\begin{itemize}
\item \(\sin^2 x+\cos^2 x = \sec^2 x - \tan^2 x = \csc^2 x - \cot^2 x = 1\)
\item \(\mathop{{}^{\sin}_{\tan}{}}(-x) = -\mathop{{}^{\sin}_{\tan}} x\); \(\cos(-x) = \cos x\)
\item \(\sin(\alpha\pm\beta) = \sin\alpha\cos\beta\pm\cos\alpha\sin\beta\)
\item \(\cos(\alpha\pm\beta) = \cos\alpha\cos\beta\mp\sin\alpha\sin\beta\)
\item \(\tan(\alpha\pm\beta) = (\tan\alpha\pm\tan\beta)/(1\mp\tan\alpha\tan\beta)\)
\item \(\cot(\alpha\pm\beta) = (\cot\alpha\cot\beta\mp 1)/(\cot\beta\pm\cot\alpha)\)
\item \(\sin(2x) = 2\sin x\cos x = (2\tan x)/(1+\tan^2 x)\)
\item \(\cos(2x) = \cos^2 x-\sin^2 x = 2\cos^2 x - 1 = 1 - 2\sin^2 x = (1-\tan^2 x)/(1+\tan^2 x)\)
\item \(\tan(2x) = (2\tan x)/(1-\tan^2 x)\)
\item \(\cot(2x) = (\cot^2 x - 1)/(2\cot x)\)
\item \(\sec(2x) = (\sec^2 x)/(2 - \sec^2 x)\)
\item \(\csc(2x) = (\sec x\csc x)/2\)
\item \(\sin(3x) = -4\sin^3x+3\sin x\)
\item \(\cos(3x) = 4\cos^3x-3\cos x\)
\item \(\tan(3x) = (3\tan x-\tan^3x)/(1-3\tan^2x)\)
\item \(\cot(3x) = (3\cot x-\cot^3x)/(1-3\cot^2x)\)
\item \(\sin^2x = (1-\cos(2x))/2\); \(\cos^2x = (1+\cos(2x))/2\)
\item \(\sin x\pm\sin y = 2\sin((x\pm y)/2)\cos((x\mp y)/2)\)
\item \(\cos x\pm\cos y = \pm2\mathop{{}^{\cos}_{\sin}{}}((x+y)/2)\mathop{{}^{\cos}_{\sin}{}}((x-y)/2)\)
\item \(\mathop{{}^{\cos}_{\sin}{}}x\mathop{{}^{\cos}_{\sin}{}}y = (\cos(x-y)\pm\cos(x+y))/2\)
\item \(\mathop{{}^{\sin}_{\cos}{}}x\mathop{{}^{\cos}_{\sin}{}}y = (\sin(x+y)\pm\sin(x-y))/2\)
\item \(\tan x\tan y = (\cos(x-y)-\cos(x+y))/(\cos(x-y)+\cos(x+y))\)
\end{itemize}
\end{document}
