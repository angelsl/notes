\chapter{Quantum Physics}
\restartlist{enumerate}
\section{Black body radiation}
\begin{enumerate}
    \item A black body is an ideal physical body that absorbs all incident electromagnetic radiation and emits electromagnetic radiation called black body radiation.
    \item Planck's law describes the electromagnetic radiation emitted by a black body in thermal equilibrium at a definite temperature.
    \(I(\lambda)=\frac{2\pi hc^2}{\lambda^5\left(e^{hc/\lambda k_BT}-1\right)}\).
    \(I(f)=\frac{2\pi hf^3}{c^2\left(e^{hf/k_BT}-1\right)}\).
    \item Wein's approximation approximates spectral emittance for small wavelengths. \(I(\lambda) = 2hc^2\lambda^{-5}e^{-hc/\lambda k_BT}\).
    
    Wein's approximation can be derived from Planck's law by assuming \(hf \gg k_BT\), so \((e^{hf/k_BT} - 1)^{-1} \approx e^{-hf/k_BT}\). Substituting this into Planck's law results in Wein's approximation.
    \item Rayleigh-Jeans law approximates spectral emittance for large wavelengths. \(I(\lambda) = 2\pi ck_BT\lambda^{-4}\).
    
    Rayleigh-Jeans law can be derived from Planck's law by Taylor expansion. \(e^{hc/\lambda k_BT} \approx 1 + \frac{hc}{\lambda k_BT}\); substituting this into Planck's law results in Rayleigh-Jeans law.
    \item The Stefan-Boltzmann law gives the total intensity radiated by a black body at temperature \(T\).
    
    \(I = \int^{\infty}_{0}I(\lambda)\mathrm{d}\lambda = \sigma T^4\), \(\sigma = \frac{2\pi^5{k_B}^4}{15c^2h^3} = \SI{5.670400(40)E-8}{\watt\per\square\meter\per\kelvin\tothe{4}}\).
    \item Wein's displacement law \(\lambda_mT = \SI{2.90E-3}{\metre\kelvin}\) gives the peak wavelength of the radiation from a black body. It can be derived from Planck's law by taking the derivative w.r.t. \(\lambda\) and setting it to 0.
\end{enumerate}
\section{Particle model of light}
\begin{enumerate}
    \item The energy of a photon is given by \(E = hf\) and momentum \(p = \frac{h}{\lambda}\).
    \item The photoelectric effect occurs when a material emits electrons from its surface when illuminated. Electrons absorb energy from incident photons to overcome the attraction of positive ions in the material. Electrons emitted as a result of this effect are photoelectrons, and they create a photocurrent.
    \item The work function \(\phi\) is the minimum energy needed to remove an electron from the surface. Since electrons can only either absorb all or none of a photon's energy, electrons are only emitted when \(hf > \phi\).
    \item The stopping potential \(V_0\) is the potential that is just negative enough to stop the photocurrent.
    
    \(eV_0 = \frac{1}{2}m_ev^2_{max} = (\gamma-1)m_ec^2 = hf - \phi\); this is just an energy conservation equation.
    \item Light can be scattered when it hits an electron. When a photon is scattered, its wavelength changes by \(\Delta\lambda = \frac{h}{m_ec}(1-\cos\phi)\) -- this is derived from conservation of energy and momentum.
    \item Pair production occurs when a photon of sufficiently high energy of at least \(2m_ec^2\) hits a target and disappears completely, creating an electron and a positron or other particle-antiparticle pair.
    
    Pair production cannot occur in free space, as otherwise conservation of momentum cannot be satisfied -- consider the case where the photon has exactly \(2m_ec^2\) of energy.
    \item When particle and anti-particle collide, they annihilate and produce photons.
    \item Photons can be scattered when they collide with electrons, being re-emitted with a longer wavelength \(\lambda\prime\) and at an angle \(\theta\). Through conservation of momentum and energy, we have \(\lambda\prime - \lambda_0 = \frac{h}{mc}(1-\cos\theta)\).
    
    If the intensity-wavelength is plotted for a Compton scattering experiment, two peaks at \(\lambda\prime\) and \(\lambda_0\) will be seen. The first peak occurs when photons strike loosely bound electrons and have most of their energy transferred to the electron; the second occurs when photons strike tightly bound electrons and have almost no energy transferred.
\end{enumerate}
\section{Electron energy levels}
\begin{enumerate}
    \item Atoms have a set of infinite possible discrete energy levels in which electrons can reside.
    \item When an electron emits or absorbs a photon with energy equal to the energy difference between the two levels, it can make a transition between the two levels.
    \item If all the possible photons emitted or absorbed by an atom are dispersed to form a spectrum, each line in the spectrum -- emission and absorption, respectively -- corresponds to one particular transition.
    \item When all the electrons of an atom are at the lowest energy levels, the atom is in the ground state. When at least 1 electron is at a higher energy level, the atom is excited.
    \item The Bohr model is based on the postulate that each energy level of a hydrogen atom corresponds to a specific stable circular orbit of the electron around the nucleus i.e. the magnitude of the electron's angular momentum is quantised. Bohr's quantisation condition \(L_n =m_ev_nr_n = n\hbar\).
    \item Using Bohr's quantisation condition, the Coulomb force and Newton's 2nd law, the radius of the hydrogen energy levels and the energy of each level can be determined.

    When deriving, the electron mass \(m_e\) can be replaced by the reduced mass \(\mu_e = \frac{m_eM}{m_e+M}\) to account for the fact that the electron and proton in the nucleus revolve about their mutual centre of mass instead of the proton being stationary.
\end{enumerate}
\section{Lasers}
\begin{enumerate}
    \item Light can interact with atoms through absorption, spontaneous emission, and stimulated emission.
    \item Stimulated emission is when an electron in a higher energy level encounters a passing photon and transits to a lower energy level by emitting a photon with the same frequency and phase.
    \item Stimulated emission can be exploited to create strong coherent beams of light i.e. lasers.
    \begin{enumerate}
        \item Population inversion in a gas is achieved i.e. more atoms are in an excited state than the ground state, by using certain atoms with a metastable state with longer excited lifetimes.
        \item A short while later, some electrons will fall to lower energy states by spontaneous emission. Emitted photons cause stimulated emission of other gas atoms.
        \item An intense, monochromatic and coherent beam of light is created.
    \end{enumerate}
\end{enumerate}
\section{X-ray production}
\begin{enumerate}
    \item X-rays can be produced when electrons hit a metal target and decelerate abruptly. \item Accelerating charges emit photons -- bremstrahhlung -- with energies equal to the loss in kinetic energy. Since the magnitude of deceleration can vary, this gives rise to a continuous spectrum of photons.
    \item Incident electrons can eject electrons from the target metal atoms, leaving a vacancy in one of the electron shells. An electron can fall from a higher energy level, emitting a photon in the process.
    
    If the incident electron kicks out an electron in the \(n = 1\) shell i.e. the K shell, and an electron in the \(n=2\) shell falls to fill the vacancy, the line created in the spectrum is termed the K\textsubscript{\upalpha} line.
    
    This gives rise to the characteristic X-ray spectrum of the target metal.
    \item Mosley's law gives the frequency of the K\textsubscript{\upalpha} lines of an element: \(f = (\SI{2.48E15}{\hertz})(Z-1)^2\).
\end{enumerate}
\section{Wave nature of matter}
\begin{enumerate}
    \item For a particle with mass \(m\) moving at speed \(v\), its de Broglie wavelength \(\lambda = \frac{h}{p} = \frac{h}{\gamma mv}\).
    \item Electrons scattered by suitable surface (i.e. line spacing approximately same order as de Broglie wavelength or smaller) will form diffraction patterns.
\end{enumerate}
