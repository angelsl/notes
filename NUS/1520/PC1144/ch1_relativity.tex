\chapter{Relativity}
\restartlist{enumerate}
\begin{enumerate}
    \item Einstein postulated that
    \begin{slinenum}
        \item the laws of physics are the same in every inertial reference frame, independent of the motion of the reference frame -- covariance; and
        \item the speed of light in vacuum \(c = \SI{299792458}{\metre\per\second}\) is the same in every inertial reference frame -- invariance.
    \end{slinenum}
    These postulates give rise to special relativity.
    \item \(\gamma = \left(1-\frac{v^2}{c^2}\right)^{\frac{-1}{2}}\). \(v = c\sqrt{1 - \frac{1}{\gamma^2}}\).
    \item If an object is moving in a moving reference frame, to an observer in a stationary reference frame, the object will travel a greater distance -- the combination of the movement of the object and of the moving frame itself.
    
    Since light can be this object, and \(c\) must stay the same across both frames, time must be different between the frames. This is time dilation.
    
    The time measured by the observer stationary relative to the event is known as the proper time \(\Delta t_0\). The time measured by an observer moving relative to the observer is \(\Delta t = \gamma\Delta t_0\).
    \item Length contraction can be derived from time dilation. \(\frac{l}{l_0} = \frac{t_0v}{tv} = \frac{1}{\gamma} \Rightarrow l = \frac{l_0}{\gamma}\).
    
    The proper length (or rest length) is the length as seen by an observer stationary relative to the length. Length contraction occurs only in the direction of motion; lengths in the other orthogonal axes remain uncontracted.
    \item Events that occurs simultaneously in one frame may not occur simultaneously in another frame. However, physical events e.g. light being emitted by a light source, or light reaching some object or point, are always simultaneous -- it does not make sense to say that to one observer, light from two sources reaches a detector simultaneously but to another observer it does not.
    \item The Doppler effect occurs as a result of special relativity. Consider a source moving towards a stationary observer with speed \(u\), emitting wave crests every \(T\) in the observer's frame.
    
    The observer measures \(f = \frac{c}{\lambda}\ = \frac{c}{(c-u)T}\). Since the observer's time is dilated \(T = \gamma T_0\), substituting, we get \(f = f_0\sqrt{\frac{c + u}{c - u}}\), where positive \(u\) is when the source is moving towards the observer.
    \item For a primed frame moving in the \(x\)-direction at speed \(v\),
    \begin{itemize}
        \item the Lorentz transformation for coordinates is \(x\prime = \gamma(x-vt)\), \(t\prime = \gamma\left(t - \frac{v}{c^2}x\right)\);
        \item and the inverse Lorentz transformation is \(x = \gamma(x\prime+vt\prime)\), \(t = \gamma\left(t\prime + \frac{v}{c^2}x\prime\right)\)
    \end{itemize}
    \item The spacetime interval \(s^2\) is the 4-dimensional spacetime equivalent of the 3-dimensional distance. \(s^2 = c^2\Delta t^2 - \Delta x^2 - \Delta y^2 - \Delta z^2\). From the spacetime interval,
    \begin{itemize}
        \item Proper time \(\Delta\tau = \sqrt{\Delta t^2 - c^{-2}(\Delta x^2 + \Delta y^2 + \Delta z^2)}\).
        \item Proper distance \(\Delta\sigma = \sqrt{\Delta x^2 + \Delta y^2 + \Delta z^2 - c^2\Delta t^2}\).
    \end{itemize}
    The spacetime quantity is invariant across inertial frames of reference.
    \item The relativistic velocity transformation, where \(v\) is the velocity of the moving frame, for velocity parallel to the motion of the moving frame is \(u_x\prime = \frac{u_x - v}{1 - u_xvc^{-2}}\) and for velocity perpendicular to the moving frame is \(u_{y,z}\prime = \frac{u_{y,z}}{\gamma(1-u_xvc^{-2})}\).
    \item Relativistic momentum is \(\mathbf{p} = \gamma m\mathbf{v}\). Newton's 2nd law stays the same -- \(\mathbf{F} = \sldd{\mathbf{p}}{t}\).
    \item The rest energy of a particle is \(E = mc^2\), and the energy of a moving particle is \(E = \gamma mc^2\), which leads to \(KE = (\gamma-1)mc^2\). This further leads to the energy dispersion relation \(E^2 = \mathbf{p}^2c^2 + m^2c^4\).
    
    Since mass is invariant across all frames, the quantity \((E^2 - \mathbf{p}^2c^2)\) is also invariant.
\end{enumerate}
