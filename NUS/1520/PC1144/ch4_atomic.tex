\chapter{Atomic Structure}
\restartlist{enumerate}
\section{Hydrogen atom}
Solving the three-dimensional Schrodinger equation gives us the energy levels of the hydrogen atom \[E_n=-\frac{1}{(4\pi\varepsilon_0)^2}\frac{m_re^4}{2n^2\hbar^2}=-\frac{\SI{13.60}{\eV}}{n^2} \qquad \left(m_r = \frac{m_em_p}{m_e+m_p}\right)\] where \(n\) is the principal quantum number and \(m_r\) is the reduced mass.

We also get two other quantum numbers, the orbital angular momentum quantum number \(l\) and the magnetic quantum number \(m_l\), where \(0 \leq l < n\) and \(-l \leq m_l \leq l\). The combination of these three numbers gives rise to the familiar orbitals 1s, 2s, 2p\textsubscript{x}, 2p\textsubscript{y}, 2p\textsubscript{z}, and so on, where s represents \(l = 0\) and p represents \(l = 1\) and so on, and \(m_l\) represents the various orbitals in subshells e.g. the 2p\textsubscript{x}, 2p\textsubscript{y} and 2p\textsubscript{z} orbitals.
\section{Zeeman effect}
When an atom is placed in a magnetic field of density \(B\), the energy degeneracy of orbitals within each subshell is removed, and they split into different energy levels with \(\Delta E = \mu_BBm_l\) where \(\mu_B = e\hbar/2m\) is the Bohr magneton, a natural unit of magnetic moment.

If the emission spectrum of a gas is obtained with a magnetic field applied through the gas, then the above effect causes the normal spectral lines of each subshell to split further, and this is known as the normal Zeeman effect.
\section{Spin}
Electrons also have an intrinsic spin quantum number \(s = \frac{1}{2}\) that results in the magnetic spin quantum number \(m_s = \pm \frac{1}{2}\). This results in further splittings in the spectrum known as the anomalous Zeeman effect.

If we combine orbital angular momentum and spin with \(\mathbf{J} = \mathbf{L} + \mathbf{S}\), we have the total angular momentum quantum number \(-j \leq m_j \leq j\). For electron orbitals, since electrons are spin half, \(j = l \pm \frac{1}{2}\). The total number of splittings is thus \(2j + 1\), which agrees with experimental observations.
\section{Selection rules for electron transitions}
From wavefunctions obtained from the Schrodinger equation, we have selection rules for electron transitions that show that some transitions happen with much higher probabilities than others: \begin{align*}
    \Delta n &= \text{anything}\\
    \Delta l &= \pm 1\\
    \Delta m_l &= 0, \pm 1\\
\end{align*}