\documentclass[Economics.tex]{subfiles}
\begin{document}
\chapter{Macroeconomic Policies}
In general, macroeconomic policies can be divided into demand management policies, and supply-side policies.

Demand management policies in general consist of monetary policies and fiscal policies. Monetary policies involve the manipulation of monetary variables like exchange rate, interest rate and the supply of money to influence \AD{}. Fiscal policies involve the use of government spending and taxation to influence \AD{}.
\section{Exchange rate policy}
Decreasing exchange rate increases net exports, as devaluing the currency reduces the cost of the country's exports in foreign currency while increasing the cost of imports in the country's currency, so assuming \(\PEDx{} > 0\) and \(\PEDm{} > 1\), the quantity demanded of the country's exports in foreign currency rises so export revenue rises, and the quantity demanded of imports in the country falls more than proportionately so import expenditure falls, which reinforce an increase in net export revenue, increasing \AD{}. The reverse argument applies.

The condition that \(\PEDx{} > 0\) and \(\PEDm{} > 1\) can be extended and mathematically proven to simply having the sum of the two be greater than one. This is the Marshall-Lerner condition, which states that for a devaluation of a country's currency to increase net exports, \(\PEDx{} + \PEDm{} > 1\).
\subsection{Merits of exchange rate policy}
For countries with relatively large external demand, like Singapore, exchange rate policy has the greatest effect on \AD{}, since net exports will make up the greatest proportion of \AD{}. Net exports takes up about 75\% of Singapore's \AD{}, for example.
\subsection{Problems of exchange rate policy}
\subsubsection{Imported inflation}
For countries that import a large proportion of the raw materials used in production of goods, a devaluation of the currency may lead to imported inflation, with prices of goods in domestic currency increasing, because the price of imported raw materials in domestic currency increases, leading to an increased unit cost of production and thus a fall in supply of goods which involve imported raw materials. The price of exports in foreign currency thus may not fall as much, reducing the increase in net exports.

If the increase in net exports leads to demand-pull inflation, causing price levels to rise, then the price of export goods in domestic currency may increase, offsetting the fall in price of exports in foreign currency due to the devaluation, similarly reducing the increase in net exports.
\subsubsection{J-curve effect}
A devaluation, in the short run, causes net exports and trade balance to fall due to the J-curve effect. The main reason for this is time lags, as producers and consumers take a while to adjust their purchases to the changed prices caused by the change in exchange rate. There may also be contractual agreements or planned advance orders, preventing the quantity of exports or imports from changing until the contracts lapse.  This causes export revenue to remain constant, assuming quantity of exports is constant, while import expenditure rises as a constant quantity of imports is combined with increasing prices of imports in domestic currency, so net exports falls. This is in line with the Marshall-Lerner condition, as quantities not changing implies the demands are both price inelastic.
\section{Interest rate policy}
Lowering interest rate increases consumption and investment as interest rate is the cost of borrowing, which when lowered will make consumers more willing and able to borrow to buy expensive items like cars, increasing consumption, and firms invest more as investments with lower expected returns will appear more profitable since the cost of borrowing to undertake those investments have fallen, increasing investment. Thus \AD{} increases. The reverse argument applies.

A reduced interest rate will also cause an outflow of hot money as investors seek higher returns from countries with higher interest rates, so the supply of the country's currency increases while the demand falls and so the currency depreciates. Prices of the country's exports in foreign currency falls while prices of imports in the country's currency rises, and assuming the Marshall-Lerner condition holds, net exports increases, also increasing \AD{}. This is a small point, however.
\subsection{Merits of interest rate policy}
Since investment increases, capital accumulation increases and this will benefit potential growth in the long run as \AS{} will increase in future with increased capital accumulation.
\subsection{Problems with interest rate policy}
If the economy is in recession, consumer and business confidence may be low, and increasing interest rate may not increase consumption or investment by much as consumers want to save more in case they e.g.\ become unemployed, and firms are pessimistic about investments. This reduces the effectiveness of interest rate policy.

Some firms may not need to borrow in order to invest as they may have their own reserves. Changing interest rate will not really affect how much such firms invest, which diminishes the effectiveness of interest rate policy.

Singapore does not use interest rate policy due to its choice of controlling its exchange rate and having free capital flow. If Singapore were to lower interest rates, for example, hot money would flow out of Singapore, foiling the policy.
\section{Fiscal policy}
Fiscal policies involve the use of government spending and taxation to influence \AD{}.

Reducing personal and corporate income taxes will increase the disposable income of households and the post-tax profits of firms, increasing their purchasing power and expected yields from investments respectively. For households, they increase consumption of normal goods and so consumption increases. For firms, the marginal efficiency of investment curve shifts right and at every interest rate, investment increases. Thus \AD{} increases.

Increasing government spending directly increases \AD{}, it being a component of \AD{}.
\subsection{Merits of fiscal policy}
A fall in personal income taxes will increase the opportunity cost of leisure, which disincentivises leisure and encourages work. People who are currently not seeking jobs may decide to do so, which will boost \AS{} in future.

Since investment increases, capital accumulation increases and this will benefit potential growth in the long run as \AS{} will increase in future with increased capital accumulation.
\subsection{Problems with fiscal policy}
Increasing government spending may lead to the crowding-out effect if government spending is financed by borrowing, as the government will be competing with private firms for loans, which increases the demand for loans and thus the interest rate, reducing investment and diminishing the increase in \AD{} from increased government spending.

The government may also risk running into a budget deficit if it increases government spending. To finance these debts, the government may have to borrow, and then taxes may rise in future in order to repay loans, which will counter an increase in \AD{} in the present. Too huge a government debt may weaken investor confidence which reduces foreign investment in the country, leading to capital flight, which will destabilise the exchange rate, and a reduced credit rating, which will make it more expensive for the country to borrow.

Fiscal policy may also involve time lags, depending on whether legislation is required to change in government spending or taxes, and how fast the changes cause an effect on consumption and investment.

If the economy is in recession, consumer and business confidence may be low, and direct taxes may not increase consumption or investment by much as consumers want to save more in case they e.g.\ become unemployed, and firms are pessimistic about investments. This reduces the effectiveness of this form of fiscal policy.

In Singapore, fiscal policy has limited effect due to its small fiscal multiplier, which in turn is due to having a high marginal propensity to import and marginal propensity to save. The effect of tax cuts is also likely to be small, given that domestic demand is a relatively small component of \AD{}, and that our tax base is small -- only one-third of the working population. Singapore's tax system is also not very counter-cyclical as taxes are based on the previous year's income: if the current year is a recession year while the previous is a boom year, individuals and firms pay tax in the current year based on higher incomes earned the previous year, when ideally the system should leave them with more money due to the recession.
\section{Other policies}
To solve a current account deficit, a government may use export subsidies and tax rebates for export industries to boost the country's exports by making their unit cost of production lower, in turn reducing export prices and thus increasing export revenue, assuming \(\PEDx{} > 1\), boosting the current account and thus the balance of payments.

The government can also use protectionism to do so -- the above policy is actually a form of protectionism anyway.

Of course, this comes with all the demerits of protectionism.
\section{Supply-side policy}
Supply-side policies are a basket of policies used to influence \AS{}. They are generally categorised into market-oriented policies and interventionist policies. Market-oriented policies rely on market forces and competition to achieve greater efficiency, while interventionist policies rely on market intervention and the correction of supposed market failures.
\subsection{Market-oriented policies}
Cutting direct tax rates, while primarily a fiscal policy, affects \AS{} as well. As explained earlier, cutting corporate income taxes results in an increase in investments, increasing capital accumulation, innovation and the development of new technology, leading to an increase in productive capacity and thus \AS{}. Lower personal income taxes increases the opportunity cost of leisure, increasing the incentive to work for longer or more efficiently and also enticing those previously not in the labour market to work. Supply in individual markets increases and so \AS{} increases.

\paragraph{Problems} However, reducing personal income taxes may also encourage people to work fewer hours since they can get the same amount of disposable income while working less. Reducing corporate income taxes may result in firms paying higher dividends to shareholders instead of investing.

Cutting unemployment benefits will also increase the opportunity cost of leisure, which incentivises work, eventually leading to an increase in \AS{}.

\paragraph{Problems} However, this may increase income inequality especially for those who are structurally unemployed, and if there is low unemployment, this will not help much at all.

The government can also encourage competition by introducing pro-competition policies like antitrust laws, removing barriers to entry to regulated markets, privatisation, and reducing trade barriers. The increase in competition encourages firms to become more efficient, producing a greater output from a given amount of resources. Unit cost of production falls, so supply curves shift right, and \AS{} increases.

\paragraph{Problems} In practice, implementing such laws alienates the business community, and it also reduces supernormal profits, making it more difficult for firms to innovate and do research and development as they have less resources to do so.

The government can reform trade unions through laws that restrict the extent to which unions can push wages above equilibrium and enforce restrictive practives, increasing employment, labour market flexibility, and efficiency. The reduced labour costs in turn increase firms' profits, enabling more investment, helping \AS{} increase in future.

\paragraph{Problems} In practice, implementing such laws will alienate workers. Flexible labour markets will also increase income inequality as workers are forced to accept jobs at lower wages. It also leads to lower job security as trade unions have less power with regards to retrenchment and the like, which may cause greater stress in the workplace and thus lower efficiency, diminishing the increase in \AS{}.
\subsection{Interventionist policies}
The government can provide subsides for education and training, which will reduce the price of doing so. Employers and workers will be more willing and able to send employees to and attend, respectively, training. If the training helps to improve workers' productivity, each worker can produce more output per man hour, resulting in a lower unit cost of production, increasing supply and thus \AS{}.

\paragraph{Merits} This also increases \AD{} through an increase in government spending.

\paragraph{Problems} However, it requires time to take effect, and it may not be very effective at all, depending on how receptive workers are. It also requires government spending, and so has all the associated issues.

Subsidising research and development will make firms more willing and able to conduct it as the price of doing so decreases. If the research results in new methods of production that are more efficient, the unit cost of production falls; firms may also be able to produce more from the same amount of resources, so supply and \AS{} increases. 

\paragraph{Merits} This also increases \AD{} through an increase in government spending.

\paragraph{Problems} Research and development may not result in any better method of production -- it is uncertain. If this happens, funds used for the research and development by firms would have been wasted since an opportunity cost has been incurred and they could have been used for other purposes like upgrading equipment or giving employees bonuses, but instead they were channeled into research and development which ultimately did not result in an increase in profits or other benefit for the firm.

To combat search unemployment, the government can provide information through mass media, job agencies and job fairs, to help unemployed workers be matched to suitable jobs more quickly.

\paragraph{Problems} The success of such a policy depends really on how keen the unemployed are about getting unemployed. If they do not really want a job, they may not seek for jobs as actively, and so stay unemployed for longer.
\subsection{Merits of supply-side policy}
Supply-side policy can act on both \AD{} and \AS{}, bringing about non-inflationary growth; the increase in \AD{} helps to drive the increase in national income, while the increase in \AS{} prevents inflation, allowing the actual growth to be sustained.

If prices are lowered due to supply-side policy, it may also improve the price competitiveness of exports, which will improve the country's current account and thus balance of payments.
\subsection{Problems with supply-side policy}
Sup\-ply-side policies often involve high costs, which will incur a high opportunity cost, as resources that could be channeled elsewhere are now channeled to these policies, and if the benefits of the policy are less than the benefits that the resources could have brought if used elsewhere, then there is a misallocation of resources.

If the costs of supply-side policy are funded through taxes, the government may increase taxes, which may have disincentive effects on work as the opportunity cost of leisure is decreased. Workers now have less incentive to work, which may cause labour productivity to fall, offsetting the benefits from supply-side policy.

If the costs are otherwise funded through government borrowing, and taxes may rise in future in order to repay loans, which will lead to disincentive effects as above. Too huge a government debt may weaken investor confidence which reduces foreign investment in the country, leading to capital flight, which will destabilise the exchange rate, and a reduced credit rating, which will make it more expensive for the country to borrow.

Supply-side policies are also uncertain, as mentioned earlier.

On their own, supply-side policies will not cause actual growth if the economy is not near or at full employment. An increase in \AD{} is needed for the economy to have actual growth.
\section{Macroeconomic conflict}
Macroeconomic goals can sometimes conflict with each other. Most often, inflation conflicts with everything else as combatting inflation usually involves cooling the economy and reducing pressures while the other goals involve boosting the economy.
\end{document}
