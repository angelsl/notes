\documentclass[Economics.tex]{subfiles}
\begin{document}
\section{Income-AE model}
Aggregate expenditure \AgE{} is the total planned expenditure on all final goods and services in an economy over a period of time. It is the sum of consumption \(C\), investment \(I\), government expenditure \(G\) and net exports \(X-M\).

The difference between \AgE{} and \AD{} is that \AgE{} deals with nominal values whereas \AD{} deals with real values (which is why \AD{} deals with price levels). Just like \AD{}, \AgE{} varies with the level of national income: when people have more income, they will tend to spend more.

The economy is at equilibrium when the aggregate expenditure is equal to the total income or output i.e. \[Y = \AgE\] On a graph of planned expenditure against national income, this equilibrium is represented by the point of intersection between the \(Y = \AgE\) line and the line representing \(\AgE\) at various income levels.

If there is a change in national income, then there is a movement along the \(\AgE\) curve. If there is a shift of \(AE\) due to \begin{slinenumor}
\item a change in autonomous consumption e.g.\ due to a change in preferences
\item a change in investment expenditure e.g.\ due to a change in interest rate
\item a change in government expenditure e.g.\ due to changes in policy
\item a change in net export expenditure e.g.\ due to changes in exchange rate
\end{slinenumor}, then the multiplier process takes place.
\subsection{Adjustment and multiplier process}
(This is in addition to the multiplier process in section~\ref{s:6.mult}.)

Suppose a firm invests in building a new office, costing \dol[m]{10}. This is an injection of \dol[m]{10} into the economy, causing a shift of \AgE{} from \(\AgE_1\) to \(\AgE_2\). At the current planned output \(Y_1\), \(\AgE > Y\), and there will be unplanned rundown of inventories by \slmf{AB}. Firms respond by hiring more factors of production of value equal to the injection i.e.\ \dol[m]{10}, so planned output rises by the same amount to \(Y_2\), and the economy moves from \(B\) to \(C\).

Households supplying factors of production to these firms receive an additional income of \dol[m]{10}, and assuming an \slmf{MPW} of \num{0.1}, they spend \dol[m]{9} on domestic goods and services, and so at \(Y_2\) planned expenditure exceeds planned output by \slmf{CD} i.e. \dol[m]{9} and there is further unplanned rundown of stocks by that amount. Firms hire more factors of production of value equal to \dol[m]{9} and income increases to \(Y_3\), seen as a movement from \(D\) to \(E\). At \(Y_3\), planned expenditure still exceeds planned output by \slmf{EF} and there is unplanned rundown of stocks by \dol[m]{8.1}. Firms again hire more factors of production, and the cycle repeats until a new equilibrium is reached at \(Y_5\) where planned expenditure equals planned output. At this new equilibrium, the change in income is a multiplied value of the injection that caused the adjustment, with the multiplier as discussed in section~\ref{s:6.mult}.
\end{document}