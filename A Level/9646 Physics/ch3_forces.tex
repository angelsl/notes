\documentclass[Physics.tex]{subfiles}
\begin{document}
\chapter{Forces}
\section{Types of forces}
\sldef{Mass} is the quantity of matter in an object, a measure of inertia, and the property of a body to resist changes in motion. As a result of mass, an object experiences \sldef{weight} \(m\mathbf{g}\), which is the force acting on an object due to gravitational attraction. The \sldef{apparent weight} of an object is the normal contact force exerted by the surface on which the object is. The \sldef{centre of gravity} of an object is the point at which the entire weight of the object may be taken to act.

The \sldef{normal contact force} is the force exerted by the surface of one object on that of another when they are in physical contact; it acts perpendicularly to the surface of contact and prevents the objects from passing through each other. \sldef{Friction} is the resistive force acting between objects that opposes motion, preventing objects from slipping on each other. The \sldef{contact} or \sldef{reaction force} is the resultant of the friction and normal forces.

\sldef{Drag} is a resistive force experienced by an object moving in a fluid, which acts in the direction opposite to the object's motion.

\sldef{Tension} is the pulling force exerted by a rope, string, cable or other flexible object on another object.
\section{Effects of forces}
\sldef{Hooke's law} states that within the limit of proportionality, the extension produced in a material is directly proportional to the load applied. The tension created in a spring or material, within the constant of proportionality, is \begin{equation}\mathbf{F} = -k\mathbf{x}\end{equation}

The \sldef{moment} of a force is the product of the force and the perpendicular distance between the axis of rotation and the line of action of the force, given by \begin{equation}\mathbf{\tau} = \mathbf{F} \times \mathbf{d} = \mathbf{F}_{\mathrel\bot} d\end{equation}

A \sldef{couple} is a pair of forces that are equal in magnitude but opposite in direction whose lines of action do not coincide. Their combined moment is the product of one of the forces and the perpendicular distance between the lines of action of the forces. A couple produces no resultant force as the forces balance. They produce only a resultant moment.

The \sldef{principle of moments} states that for an object to be in rotational equilibrium, the sum of clockwise moments about any pivot must equal the sum of anti-clockwise moments about that pivot.

An object is stable if it returns to its original position after having been displaced slightly (by rotation). A low centre of gravity and a wide base helps to make an object more stable.
\section{Pressure}
\sldef{Pressure} is the force acting per unit area, where the force acts perpendicularly to the plane of the area A. The SI unit of pressure is pascal (\si{\pascal}) i.e. \SI{1}{\newton\per\square\metre}.

In a fluid with density \(\rho\), the pressure acting on an object at depth \(h\) is \begin{equation}p = h\rho g\end{equation} Take into account atmospheric pressure, or any other force acting on the liquid.

\sldef{Upthrust} is an upward force experienced by a body immersed in a fluid due to the pressure difference in the fluid between the lower and upper surface of the object. \begin{equation}\mathbf{U} = -V\rho \mathbf{g}\end{equation} \sldef{Archimedes' principle} states that the upthrust experienced by an object partially or fully immersed in a fluid is equal to the weight of the fluid displaced by the object. For an object to float in a fluid, the maximum upthrust on the object if it is fully immersed in the fluid must be equal to or greater than the object's weight.
\end{document}