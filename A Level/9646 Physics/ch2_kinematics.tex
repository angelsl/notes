\documentclass[Physics.tex]{subfiles}
\begin{document}
\chapter{Kinematics}
\sldef{Distance} \(x\) is the total length covered by a moving object irrespective of the direction of motion.

\sldef{Displacement} or position \(\mathbf{x}\) is the shortest linear distance of a moving object from a given reference point.

\sldef{Speed} is the rate of change of distance travelled with respect to time. \begin{equation}v = \frac{\mathrm{d}x}{\mathrm{d}t}\end{equation} 

\sldef{Velocity} is the rate of change of displacement with respect to time. \begin{equation}\mathbf{v} = \frac{\mathrm{d}\mathbf{x}}{\mathrm{d}t}\end{equation}

\sldef{Acceleration} is the rate of change of velocity with respect to time. \begin{equation}\mathbf{a} = \frac{\mathrm{d}\mathbf{v}}{\mathrm{d}t}\end{equation}

\sldef{Average speed} is the total distance travelled over the total time taken. \begin{equation}\langle v \rangle = \frac{\Delta x}{\Delta t}\end{equation}

\sldef{Average velocity} is the total change in displacement over total time taken. \begin{equation}\langle\mathbf{v}\rangle = \frac{\Delta \mathbf{x}}{\Delta t}\end{equation}

\sldef{Average acceleration} is the total change in velocity over total time taken. \begin{equation}\langle\mathbf{a}\rangle = \frac{\Delta \mathbf{v}}{\Delta t}\end{equation}
\section{Freefall}
An object that experiences no force other than its weight and possibly drag is in freefall. Neglecting air resistance, an object in freefall near the Earth's surface experiences a constant downward acceleration \(\mathbf{g} = \SI{9.81}{\metre\per\square\second}\) towards the Earth's centre of mass.

In reality, an object will experience a drag force \(\mathbf{F}_D \propto \mathbf{v}\) for small \(\mathbf{v}\), and \(\mathbf{F}_D \propto \mathbf{v}^2\) for larger \(\mathbf{v}\). The drag force opposes the object's motion. Terminal velocity is the velocity an object reaches when drag exactly balances the object's weight.
\section{Projectile motion}
In \sldef{projectile motion}, it is assumed that
\begin{slinenum}
\item the acceleration due to gravity is constant throughout the motion, i.e. \(\mathbf{g}\) is constant
\item there is no horizontal acceleration, i.e. \(\mathbf{a}_x = 0\)
\item air resistance is negligible, i.e. \(\mathbf{F}_D = 0\).
\end{slinenum}
Projectile motion with the above assumptions generally creates a parabolic trajectory that is symmetric; the \sldef{trajectory} is the path described by a projectile. \sldef{Range} is the distance on the plane between the point of projection and point of impact.

With air resistance, projectile motion describes an asymmetric trajectory. Air resistance also decreases the object's time of flight, horizontal range and maximum height reached.
\section{Kinematics equations}
The following equations apply only when acceleration is constant.
\begin{align}
\mathbf{a} &= \frac{\Delta\mathbf{v}}{\Delta t} = \frac{\mathbf{v} - \mathbf{u}}{t} \mathrel{\therefore} \mathbf{v} - \mathbf{u} = \mathbf{a}t \mathrel{\therefore} \mathbf{v} = \mathbf{u} + \mathbf{a}t\\
\mathbf{s} &= \int_{}^{}{\mathbf{v}\mathrm{d}t} = \int_{}^{}{(\mathbf{u} + \mathbf{a}t)\mathrm{d}t} = \mathbf{u}t + \frac{1}{2}\mathbf{a}t^{2}\\
\begin{split}\mathbf{s} &= \mathbf{u}t + \frac{1}{2}\mathbf{a}t^{2} = \mathbf{u}t + \frac{1}{2}\frac{( \mathbf{v} - \mathbf{u})}{t}t^{2}%= \mathbf{u}t + \frac{1}{2}t(\mathbf{v} - \mathbf{u})
\\ &= \mathbf{u}t + \frac{1}{2}\mathbf{v}t - \frac{1}{2}\mathbf{u}t = \frac{1}{2}\mathbf{u}t + \frac{1}{2}\mathbf{v}t\\&= \frac{1}{2}(\mathbf{v} + \mathbf{u})t\end{split}\\
\begin{split}\mathbf{v}^{2} &= ( \mathbf{u} + \mathbf{a}t )^{2} = \mathbf{u}^{2} + 2\mathbf{ua}t + \mathbf{a}^{2}t^{2}\\&= \mathbf{u}^{2} + 2\mathbf{a}( \mathbf{s} - \frac{1}{2}\mathbf{a}t^{2} ) + \mathbf{a}^{2}t^{2}\\&= \mathbf{u}^{2} + 2\mathbf{as} - \mathbf{a}^{2}\mathbf{t}^{2} + \mathbf{a}^{2}\mathbf{t}^{2}\\&= \mathbf{u}^{2} + 2\mathbf{as}\end{split}
\end{align}
\end{document}