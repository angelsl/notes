\documentclass[Physics.tex]{subfiles}
\begin{document}
\chapter{Current of Electricity}
\section{Charge, current and potential difference}
\sldef{Electric charge} is the physical property of matter that causes it to experience a force when placed in an electromagnetic field. For charge to flow in a circuit, there must generally be a closed conducting path extending from the positive to negative terminals of an energy source.

The \sldef{potential difference} (p.d.) between two points in an electrical circuit is the electrical energy converted into other forms of energy per unit charge passing from one point to the other. One \sldef{volt} (\SI{1}{\volt}) is the p.d. between two points in an electrical circuit when one joule of electrical energy is converted to other forms of energy as one coulomb of charge passes from one point to the other.

\sldef{Current} is the rate of flow of charge through a particular cross-sectional area with respect to time. \begin{equation}I = \frac{\mathrm{d}Q}{\mathrm{d}t}\end{equation} The \sldef{conventional current} follows positive charges flowing from the positive to negative terminal.

The \sldef{charge} that passes through a given point is the product of the steady current flowing through that point and the time for which the current flows. \begin{equation}Q = It\end{equation} One \sldef{coulomb} (\SI{1}{\coulomb}) is the quantity of charge that flows through a point when a steady current of one ampere flows through that point for one second.
\section{Resistance and resistivity}
The \sldef{resistance} of a device is defined as the ratio of the p.d. across it to the current flowing through it. \begin{equation}R = \frac{V}{I}\end{equation} A resistor has a resistance of one \sldef{ohm} (\SI{1}{\ohm}) if a current of one ampere flows through it when a p.d. of one volt is applied across it.

The power expended by a resistor is the product of the current through it and the p.d. across it. \begin{equation}P = IV = I^2R = \frac{V^2}{R}\end{equation}

\sldef{Ohm's law} states that the ratio of p.d. across a conductor to the current flowing through it is a constant if physical conditions like temperature remain constant.

The resistance of a metal increases with temperature. As temperature increase, vibration of the lattice ions increases; free electrons collide more frequently with the ions, experiencing more obstructions and a lower drift velocity, so resistance increases.

The resistance of a semiconductor generally decreases with temperature, as more electrons are likely to have sufficient thermal energy to escape from the ions, so there are more electrons to act as charge carriers.

A \sldef{diode} is a device that allows current to pass only in one direction. A diode with p.d. in the direction with low resistance is forward biased, and vice versa. An excessive reverse biased p.d. can cause the diode to break down (at the breakdown voltage), leading to a sudden large current i.e. a short circuit.

The resistance of a material is generally proportional to its resistivity and length, and inversely proportional to its cross-sectional area. \begin{equation}R = \frac{\rho l}{A}\end{equation}
\section{Electromotive force and sources}
The \sldef{electromotive force} of a source is the energy converted from other forms to electrical energy per unit charge delivered round a complete circuit. \[\varepsilon = \frac{W}{Q}\]

The e.m.f. of a source is the ability of the source to convert other forms of energy to electrical energy, while p.d. across a resistor is the ability of the resistor to convert electrical energy to other forms of energy.

Sources of e.m.f. generally have an internal resistance. This resistance can simply be treated as a resistor in series with an ideal source.

For a battery with e.m.f. \(\varepsilon\), the terminal p.d. across it is \begin{equation}V = \varepsilon - Ir\end{equation}

The \sldef{output efficiency} of a circuit is generally given by \begin{equation}\frac{R}{R+r}\end{equation} where \(R\) is the resistance of all the devices in the circuit, and \(r\) is the internal resistance of the battery. This is at a maximum when \(R = r\).
\end{document}