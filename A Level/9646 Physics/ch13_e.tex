\documentclass[Physics.tex]{subfiles}
\begin{document}
\chapter{Electric Field}
\sldef{Coulomb's law} states that the magnitude of the electric force between two point charges is directly proportional to the product of the charges and inversely proportional to the square of the distance between them i.e. \begin{equation}\left|\mathbf{F}\right| = \frac{1}{4\pi\varepsilon_{0}}\frac{\left|Q_{1}Q_{2}\right|}{r^{2}}\end{equation} Like charges repel while unlike charges attract.

\(\varepsilon_{0}\) is the permittivity of free space. The permittivity of a medium \(\varepsilon\) is usually specified relative to \(\varepsilon_{0}\).

An \sldef{electric field} is a region of space in which an electric force acts on a charged particle due to the presence of some other source charge creating the field. The electric field can be represented by field lines, similar to those of gravitational fields.

The \sldef{electric field strength} \(\mathbf{E}\) at a point in an electric field is the electrostatic force per unit positive charge exerted on a small test charge placed at the point. \begin{equation}\left|\mathbf{E}\right| = \left|\frac{\mathbf{F}}{q}\right| = \frac{1}{4\pi\varepsilon_{0}}\frac{\left|Q\right|}{r^{2}}\end{equation}

The \sldef{electric potential energy} \(U\) of a charge at a point in an electric field is the work done by an external agent in bringing the charge from infinity to that point without acceleration. For point charges, \begin{equation}U = \frac{1}{4\pi\varepsilon_{0}}\frac{Qq}{r}\end{equation} A charge moving in the direction of the electric force on it loses electric potential energy, and vice versa.

The \sldef{electric potential} \(V\) at a point in an electric field is the work done per unit positive charge by an external agent in bringing a positive test charge from infinity to that point without acceleration. \begin{equation}V = \frac{U}{q}\end{equation} For a field due to a point charge, \begin{equation}V = \frac{1}{4\pi\varepsilon_{0}}\frac{Q}{r}\end{equation}

Equipotentials are as in gravitational fields; they are lines or surfaces of constant potential.

The electric field between two parallel charged plates with p.d. \(\Delta V\) and separation \(d\) is uniform, with electric field strength at any point \begin{equation}\left|\mathbf{E}\right| = \frac{\Delta V}{d}\end{equation} The work done by such a field in moving a positive charge \(q\) a distance \(r\) parallel to the field is \[W = q\Delta V\frac{r}{d}\] Equipotentials are all lines or planes parallel to the plates, while field lines are perpendicular to the plates. Charged particles in such a field move with constant acceleration and thus in a parabolic path (neglecting gravity).
\end{document}