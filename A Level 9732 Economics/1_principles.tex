\documentclass[Economics.tex]{subfiles}
\begin{document}
\chapter{Central Economic Problem}
Economics is the social science that studies the behaviour of individuals, groups, and organisations, when they manage or use scarce resources, which have alternative uses, to achieve desired ends.

The central economic problem is scarcity. Scarcity is the problem of having seemingly unlimited human wants in a world of limited resources; society has insufficient productive resources to fulfil all wants.

There are four main economic resources.
\begin{itemize}
    \item Land refers to naturally occurring resources like forests, marine life and minerals in the earth. In other words, they are inputs into production provided by nature. Land is exhaustible and finite.
    \item Labour refers to all forms of human input, both physical and mental, into production. It is limited in terms of number and in skill level.
    \item Capital refers to inputs into production that have themselves been produced, like factories, machines, and tools.
    \item Enterprise refers to human resources that take risks and innovate. They bring together land, labour and capital to secure production. In many societies, production is limited by the lack of entrepreneurship, as people are not willing to take risks.
\end{itemize}

Due to scarcity, a society cannot have all that it wants, and it has to make choices about what, how, and for whom to produce. A society needs to decide and choose what goods it wishes to produce from the resources it has. A decision about what to produce implies a decision on how much to produce. It then needs to decide how to produce the goods it has chosen. In poor countries, a labour-intensive method is usually chosen, while in rich countries, a capital-intensive method is chosen. After producing the goods, society needs to decide whom gets the produced goods; there needs to be some means of allocating the output and deciding who gets what.

Any choice taken will incur an opportunity cost, which is simply the next best alternative forgone when picking one choice over an alternative. Individuals have limited income but unlimited desires to fulfil. Their objective is to maximise individual satisfaction and so they need to decide how much to spend and save and how to allocate their spending between their different desires. Any such decision made will incur opportunity cost. Firms have limited resources but unlimited production possibilities. Their objective is to maximise profits and so they must decide what type of and how much to produce in order to do so. Governments also have limited economic resources, but aim to maximise societal welfare. They need to decide how to allocate their resources to provide for their citizens to maximise their welfare. Any such decision will incur an opportunity cost.

Economic agents are assumed to make choices rationally. That is, they weigh up marginal costs and marginal benefits, applying the marginalist principle, to make decisions.

Marginal cost is the additional cost of doing one more unit of an activity, while marginal benefit is the additional benefit of doing one more unit of an activity.

In the case of an individual, if consuming one more unit of a good adds more benefit than cost, then it would be rational for the person to consume one more unit, and if consuming one more unit of a good adds more cost than benefit, then it would be rational for the person to consume one less unit. The individual will rationally choose to purchase at a level where their satisfaction is maximised, where marginal cost is equal to marginal benefit. This applies likewise to firms with profit, and governments with societal welfare.

In reality, the marginal benefit may not always be equal to the marginal cost due to time lag, information failure, and violation of the ceteris paribus assumption.
\section{Production possibility curve}
To illustrate choices, economists can use a production possibility curve. A production possibility curve (PPC) shows all possible combinations of 2 goods that could be produced by an economy assuming the full and efficient utilisation of a fixed quantity of resources and a given state of technology.

A PPC can be used to illustrate four fundamental economic concepts.

Scarcity is shown by the set of points outside the curve. Points outside the curve are unattainable given the productive capacity of the economy represented by the curve. These points exist because the economy does not have enough resources to produce the points outside the curve.

Choice is shown by the set of points on the curve. The economy can choose to produce any combination on the curve depending on its objectives.

Opportunity cost is shown by the negative slope of the PPC. To produce more of one good on the PPC, less of the other good must be produced, which is opportunity cost. The value of the slope of the PPC at a point is a measure of the rate one good can be transformed into the other at that point, that is, it is the marginal rate of transformation of one good into the other.

Increasing opportunity cost is shown by the increasingly negative slope of the PPC, or the PPC being concave as seen from the origin. The opportunity cost of producing successive additional units of one good increases. Increasing opportunity cost occurs because resources are imperfect substitutes: factors of production, or resources, are more suitable for the production of one good than they are for other goods. People have different skills, different products require different raw materials, and so on.

Unemployment is shown by the points inside the PPC. Points inside the PPC occur when the economy is not using all of its resources, or it is not utilising the most efficient methods of production, or both. If resources are used more fully and efficiently, the economy can produce on the curve instead, producing more of both goods.

A movement along the PPC means a reallocation of the available resources; that is, resources originally used for one good are now being used to produce another.

A shift of the PPC means that there has been a change in quantity or quality of productive resources or the state of technology. If there is an increase in the quantity or quality of productive resources, or an improvement in the state of technology, the economy will be able to produce previously unattainable combinations of output, because respectively either the resource base has increased, or the economy can now produce more from the same resource base, or both. There will be an outward shift of the PPC, representing potential economic growth. If the opposite occurs, due to, for example, the destruction of natural resources due to fires or a fall in the productivity of labour, the PPC shifts inwards, showing lower potential economic growth.

If a nation invests in capital goods, their economy is likely to grow at a faster rate in the future. Capital goods, as opposed to consumer goods, are man-made goods that aid in future production. Increasing the production of capital goods, or capital accumulation, will benefit the nation, but producing more capital goods in the current period will mean less consumer goods can be produced in the current period.
\section{Specialisation, division of labour and comparative and absolute advantage}
Specialisation occurs when individuals, firms or nations concentrate on the production of specific goods and services, typically those in which they have some sort of advantage in.

An absolute advantage exists when an agent is able to produce something with less resources than another agent. A comparative advantage exists when an agent is able to produce something while incurring a lower opportunity cost than another agent.

In a household, one person may be tasked to do the ironing while another does the cooking. At the workplace, some people are programmers while others handle the accounts of the company. Specialisation allows individuals to concentrate on what they are best at, so more goods and services will be produced.

At the international level, specialisation can take place where countries focus their production on goods and services in which they have a comparative advantage. This allows countries to specialise on what they are best at and so world output of services and goods will be increased.

The law of comparative advantage states that trade can benefit all countries if they specialise in the goods in which they have a comparative advantage.
Division of labour is when production is broken down into a number of simpler, more specialised tasks, allowing workers to achieve a high degree of efficiency, leading to lower unit costs of production, as
\begin{slinenum}
    \item workers will be more familiar with the aspect of production they perform
    \item workers require less training time as they need only to be taught how to perform the few roles they have, so less cost is incurred in training workers, and they can get to work faster
    \item workers do not need to multitask, so time is saved.
\end{slinenum}

Division of labour may also cause higher costs for firms and disadvantages for workers, as
\begin{slinenum}
    \item workers may become complacent or bored, leading to errors and a decrease in productivity, leading potentially to higher unit cost of production
    \item workers are less flexible, being trained in only a few specific roles, so if that worker is absent or leaves the firm, other workers cannot replace that worker until they are given training
    \item workers, being trained only in specific skills, may find it difficult to find new employment if he is fired.
\end{slinenum}

Specialisation and trade can lead to interdependence between countries.

When countries rely on other countries, those countries may be badly affected if the depended-on countries experience disasters or economic downturns. Singapore, for example, is always affected when there is an international crisis, like the 2003 SARS outbreak or the 2008 global economic crisis, as Singapore relies largely on its exports for revenue --- in February 2008, Singapore's exports fell by 23.7\%, and this is cited to be a reason for Singapore's poor economic growth in 2009.

Economies may be more susceptible to imported inflation. In the first half of 2007, high oil prices caused the Consumer Price Index of many countries to increase. The increase in demand for food and commodities in emerging economies also caused food prices in many countries to increase.
\end{document}
