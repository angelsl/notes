\documentclass[Economics.tex]{subfiles}
\begin{document}
\chapter{Macroeconomic Problems}
\section{Negative economic growth}
Economic growth is one of the indicators used to assess an economy's health. Economic growth is generally split into actual and potential growth.

Actual growth or, informally, simply growth, measures the change in the volume of output produced within the country over a year. It is generally measured by real GDP. On the PPC, actual growth is illustrated as a movement from a point in the curve to a point closer to, or on, the curve. On an \AD{}-\AS{} diagram, it is seen as an increase in national income.

Potential growth measures the change in the economy's ability to produce. It is the maximum speed at which the economy could grow. On the PPC, it is seen as an outward shift of the curve. On an \AD{}-\AS{} diagram, it is seen as a rightward shift in the vertical segment of \AS{}.
\subsection{Causes of negative growth}
\subsubsection{Fall in \AD{}}
A fall in aggregate demand will, assuming the economy is not at full employment, cause a fall in national income, which is negative actual growth.

% Rise in interest rates
A fall in \AD{} can be caused by an increase in interest rates. When interest rate increases, domestic consumption falls as it becomes more expensive to borrow to buy expensive items like cars or houses. Interest is also the returns to savings and so the opportunity cost of spending increases, thus consumption falls. Investment also falls, as the cost of borrrowing increases, so some projects which were profitable are no longer so.

% Higher tax rates
An increase in personal income taxes results in a fall in consumers' disposable incomes and purchasing power, so they spend less and consumption falls. An increase in corporate income taxes decreases the expected yields from investment, thus reducing investment. Both of these are components of \AD{} and so \AD{} falls.

% Reduced government expenditure
Government expenditure is itself a component of \AD{}. If it falls, then \AD{} falls.

% Appreciation of exchange rate
If a country's exchange rate increases, the price of its exports in foreign currency increases. Assuming \(\PEDx{} > 0\), this causes the quantity demanded of the country's exports to fall, and so export revenue falls. \AD{} falls.
\subsubsection{Fall in \AS{}}
A fall in aggregate supply will cause negative potential growth, and if the economy is near or at full employment, negative actual growth.

% Depletion of natural resources
If natural resources used to produce goods in a country are depleted, the country can produce less of those goods, so aggregate supply falls. Natural disasters can sometimes destroy land, causing less land to be available for production purposes.

% Fall in quantity/productivity of labour
If there is a fall in the quantity of labour, perhaps due to a natural disaster or disease, a labour shortage may result. Without enough labour, the economy cannot produce as much, and so \AS{} falls.

% Fall in stock and quality of a capital 
If there is a fall in investment, or otherwise for some reason, the amount of capital created is not enough to replace that which has worn out, the country's stock of machinery and productive capacity will be reduced. \AS{} falls.
\subsection{Consequences of negative growth}
\subsubsection{Lower living standards}
If national income falls, then the income of each citizen falls, and their purchasing power falls. They are less able to purchase goods and services for consumption, so their material standard of living falls, and standard of living falls.
\subsubsection{Increase in unemployment}
If national income falls, it means firms are producing less goods. They thus require less labour and so the aggregate demand for labour falls. Assuming wages are sticky downwards, there will be a surplus of labour, and demand-deficient unemployment results.
\subsubsection{Reduction in government revenue}
Since the income of each citizen falls, each citizen pays less tax to the government, and so government tax revenue falls. Since consumption also falls, revenue from indirect taxes also falls.
\subsubsection{Decrease in ability to redistribute income}
It may become more difficult for the government to distribute income to the poor without the rich losing out, as the taxes collected falls, so there is less revenue to redistribute in terms of subsidies and funding for social services.
\subsubsection{Fall in confidence}
Negative growth usually causes a fall in business confidence. They expect lower yields from investments and so investment falls at all interest rates.
\section{High inflation}
Inflation is defined as a period of sustained and inordinate increase in general price level. Inflation is usually measured by the consumer price index, an index measuring the price of a fixed weighted basket of goods and services consumed by a typical household. The consumer price index is usually specified relative to a base year.

The rate of inflation is the percentage change in the consumer price index. It is usually compared year-on-year.
\subsection{Demand-pull inflation}
Demand-pull inflation is a situation where \AD{} is persistently greater than \AS{} i.e.\ the economy is always close to or at full employment. If \AD{} continues to increase, the general price level increases persistently, which is inflation.

In Singapore, demand-pull inflation is likely to be externally generated, simply due to our large external demand -- export revenue makes up about 75\% of our \AD{}.
\subsection{Cost-push inflation}
Cost-push inflation is a situation where persistent increases in the unit cost of production causes a sustained and inordinate increase in general price level. Cost-push inflation is independent of \AD{}.
\subsubsection{Wage-push inflation}
Wage-push inflation occurs when strong trade unions bargain for increases in wages that exceed increases in labour productivity, which results in an increase in unit cost of production across many markets, shifting individual markets' supply curves to the left. This causes the horizontal segment of \AS{} to shift upwards. General price level rises.

When prices rise, trade unions may bargain for even higher wages to keep up with the increases in price level, and the process repeats. This is known as a wage-price spiral, and wage-push inflation results.
\subsubsection{Imported inflation}
When there is a rise in the price of imported raw materials, the unit cost of production of goods produced using those materials increases. The supply for those goods falls and if the economy is one that produces many such goods, then \AS{} falls significantly, and price levels rise. If the prices of imported raw materials continues to increase, then the process repeats and price levels continue to rise, which is inflation.

The increase in the price of imported raw materials may be caused by a devaluation of a country's currency.
\subsubsection{Profit-push inflation}
Cost-push inflation may be caused because monopolies or firms with large market power try to increase their profits by raising the prices of their products.
\subsubsection{Increase in indirect tax}
A rise in indirect taxes, especially those that apply to all goods, like goods and services tax, may increase firms' costs of production, decreasing \AS{}.
\subsubsection{Structural rigidities}
Structural rigidities may lead to inflation. When wages rise in a market due to supply of labour in the market being wage inelastic, unit labour costs rise, so unit costs of production rise and if this makes up a significant part of production in the country, \AS{} falls.
\subsection{Benefits of low inflation}
If a country has relatively lower inflation compared to competing countries, its export competitiveness is boosted, and demand for the country's exports may increase, increasing export revenue and improving the trade balance and thus the current account and balance of payments; it also boosts economic growth. Of course, this may then lead to inflation.

Low but non-zero demand-pull inflation encourages firms to expand their output as non-zero demand-pull inflation implies rising demand, which leads to higher prices and profits. Firms may undertake investments to expand output, which increases capital accumulation, thus increasing \AS{} (and also \AD{} directly), leading to potential growth.

Price stability enables businesses to create long-term plans with confidence, increasing investment in the country, which again increases \AD{} and \AS{} in the long run.
\subsection{Costs of inflation}
\subsubsection{Shoe-leather and menu cost}
Shoe-leather cost refers to the costs incurred in moving money in and out of financial institutions in search of the highest returns.

Menu cost refers to the cost of constantly having to revise price lists, labels and menus.
\subsubsection{Income redistribution}
People who earn incomes that are fixed based on their nominal value lose out the most when inflation occurs, as their real income falls, and they are less able to purchase goods and services. However, variable income earners may not lose out, depending on how their income is determined: insurance agents or property agents, for example, may earn higher commissions when there is inflation.

Debtors also gain while creditors lose as the real value of debt falls when price increases, so the purchasing power of the money repaid by a debtor will be less than that of the money borrowed.

Governments gain while taxpayers lose out as when nominal income increases to match inflation, taxpayers may advance into the next tax bracket and pay more tax, even if there is no change in real pre-tax income.

Businessmen gain from demand-pull inflation as prices usually rise faster than production costs, so profits rise. However, when there is cost-push inflation, profits may fall as rising cost is the reason for inflation.
\subsubsection{Savings}
When there is inflation, savers lose as the real value of savings falls, so purchasing power of savings falls. Even with inflation, if the interest rate is less than the inflation rate, the real value of savings will still fall. People may be discouraged from saving, increasing consumption, ceteris paribus.

Those who are risk-averse, however, may choose to increase savings in order to maintain the real level of savings, and to prepare for uncertainty, so savings may increase and consumption may fall. When this happens, negative growth may result.
\subsubsection{Resource allocation}
In a market economy, prices provide signals to producers in production decisions. When a producer sees that the price of his product has increased faster than other products, he will infer that demand for his product has increased and so devote more resources to the production of that product.

When there is inflation, it becomes hard to distinguish whether a rise in price is due to increased demand or simply due to inflation. Misallocation of resources may occur.
\subsubsection{Compromising money}
When there is hyperinflation, people may no longer accept money as payment for goods as the value of the money falls rapidly. It also becomes impractical to express the value of goods and services as the numeric prices must be constantly adjusted to reflect the falling value of money.

Money also becomes a poor store of value as the same amount of money can buy less goods and services than before, so people find it unwise to use money to store value since its value is decreasing. Finally, businessmen also become reluctant to credit as the amount they get back will be worth less.
\subsubsection{Export competitiveness}
When a country has higher inflation rates than competitors, its goods and services will become less price competitive, ceteris paribus. The price of exports in foreign currency increases and assuming \(\PEDx{} > 0\), quantity demanded falls and so export revenue falls.

If quantity demanded of exports falls, then demand for the country's currency falls, so the country's exchange rate falls. This is reinforced by imported goods becoming cheaper than domestic goods, so demand for imports increases, and import expenditure increases, so supply of the currency increases.

These effects also lead to a fall in the trade balance, the current account balance, and thus the balance of payments.
\subsubsection{Investment competitiveness}
Inflation also makes a country less attractive for investment, as inflation means the cost of production increases. The depreciation of the country's currency may also lead to capital flight, further worsening the exchange rate.
\section{High unemployment}
Unemployment refers to existence of people who are of working age, are available to work and are actively looking for a job, but cannot find one.
\subsection{Demand-deficient unemployment}
Demand-deficient unemployment occurs when the number of workers who are willing and able to work at the prevailing wage i.e.\ the quantity supplied of labour is greater than the number of job vacancies available i.e. quantity demanded. There is a surplus in the labour market.

It is usually caused by a recession. When national income falls, firms produce less and so require less labour, so the aggregate demand for labour falls. Since wages are sticky downwards due to things like contractual obligations, wage does not fall and there will be a surplus of labour.
\subsection{Equilibrium unemployment}
Equilibrium unemployment occurs when workers are unable or unwilling to take up jobs available at the prevailing wage. When an economy achieves full employment, it does not mean there is no unemployment; equilibrium unemployment still exists -- but demand-deficient unemployment does not. The rate of unemployment at full employment is known as the non-accelerating inflation rate of unemployment, or NAIRU.

There are various causes of equilibrium unemployment.
\subsubsection{Search unemployment}
Search unemployment or frictional unemployment exists due to poor information in the labour market causing a time lag before people can find suitable jobs. Employers are not fully informed about the labour that is available and workers are not fully informed about the jobs that are available. Generally, search unemployment is inevitable, short-term and does not pose a serious problem.
\subsubsection{Structural unemployment}
Structural unemployment occurs when there is labour immobility and a change in demand patterns. If there is a fall in demand for an industry's good, either because the industry is declining or due to a loss of comparative advantage, demand and thus quantity demanded (i.e.\ the number of jobs) for labour in that industry also falls. Some workers will be retrenched and if they are unable to find a new job in another industry, they are structurally unemployed -- specifically, sectoral unemployment.

If workers are unable to take up jobs because they are unwilling or unable to physically relocate to take up those jobs, as opposed to not having the right skills to do so, they are regionally unemployed.

If workers are retrenched because they do not have the skills to handle new technology or otherwise because their job has been replaced by technology, they are technologically unemployed.
\subsubsection{Seasonal unemployment}
Seasonal unemployment is equilibrium unemployment due to some industries being subject to seasonal demand, like tourism. During the peak seasons, there is high demand for labour, and vice versa, so there is high seasonal unemployment in the off-peak season.
\subsection{Costs and benefits of unemployment}
Unemployment implies that output and standard of living is lower than it could be, because it means there is unused labour. Higher unemployment also increases government expenditure on unemployment-related benefits while simultaneously reducing tax revenue from both direct and indirect taxes. Crime levels may increase as the unemployed steal to obtain necessities. 

For firms, unemployment means that the demand for goods is lower than it could be, since some people are without an income -- if they had income, they would consume more normal goods. 

The unemployed experience a fall in income as unemployment benefits are usually lower than incomes they would get if they were working (otherwise there would be no incentive to work). The loss of status and the stressed involved in being unemployed may also have adverse effects on workers.

Unemployment has some questionable benefits too. In theory, having a small level of unemployment can ease demand-pull inflation. It may also make it easier for firms to find employees since there are many unemployed to choose from. Finally, workers get more time to enjoy leisure activities.
\section{Balance of payments deficit}
The balance of payments is a record of inflows and outflows of money in and out of a country due to transactions between the residents of the country and the residents of the rest of a year, typically in a year. The value of the balance of payments is the total value of receipts minus the total value of payments for international transactions.

The balance of accounts is usually split into the current account and the capital and financial account.

The capital account has to deal with trade, income flows and current transfers; it is composed of the visible balance i.e. the balance of trade, which comprises exports and imports of goods, and the invisible balance which comprises trade in services, factor income flows i.e.\ wages, interest, rent and profits and current transfers e.g.\ government aids or private gifts of money.

The capital and financial account reflects the net change in foreign financial assets and liabilities. The capital account records the transfers of capital associated with the purchase and sale of fixed assets, patents and trademarks, to and from abroad. The financial account records the flows of money into and out of the country for investment or deposits into banks and other financial institutions.

The balance of payments also consists of errors and omissions, to account for errors, and the official reserves transactions, where surpluses are deposited and deficits are withdrawn.
\subsection{Current account deficit causes}
A current account deficit can be caused by cyclical or structural factors. Cyclical factors have to do with the business cycle, are short-term and are usually not a big worry for the country. Structural factors, however, are more serious and must be addressed by the government.
\subsubsection{Cyclical factors}
When a country has relatively high inflation, its exports become more expensive in foreign currency and thus less competitive. The quantity demanded of the country's exports falls, assuming \(\PEDx{} > 0\). Export revenue thus falls. At the same time, imports become cheaper in domestic currency compared to domestic goods. Demand for imports increases, so import expenditure increases. The balance of trade falls, and thus the current account falls.

When a country has relatively high economic growth, its national income is increasing at a faster rate than other countries'. Assuming imports are normal goods, demand for imports, quantity demanded of imports, and thus import expenditure rises. Conversely, since other countries have lower economic growth, their national incomes are rising at a slower rate and so their demand for the country's import rises at a lower rate. Thus the country's export revenue does not rise as much as import expenditure, and so the balance of trade and thus the current account falls.
\subsubsection{Structural factors}
If a country has an overvalued currency, its exports are likely to be expensive in foreign currency while imports are cheap in its currency, so its export revenue is likely to be low while its import expenditure is high. Any further appreciation of the currency will cause prices of its exports in foreign currency to increase while prices of imports in its currency decreases. As long as \(\PEDx{} > 0\), export revenue falls, and as long as \(\PEDm{} > 1\), import expenditure rises. Thus the balance of trade worsens.

If a country experiences a loss in comparative advantage, it means other countries' exports are now cheaper than the country's exports. Demand for the country's goods, both external and domestic, falls as consumers switch to goods from more efficient countries. Thus export revenue falls while import expenditure increases, and so the balance of trade worsens.

If a country is pursuing an ambitious development programme, it may import large amounts of resources and capital to aid in its development, causing import expenditure to be high. The balance of trade may worsen.
\subsection{Capital and financial account deficit causes}
If a country has relatively lower interest rates, short-term capital i.e.\ hot money flows out of the country to earn more returns from higher interest rates in other countries, causing the financial account of a country to worsen.

If there is low business confidence in a country, local investors will prefer to invest overseas, while overseas investors do not invest in the country. The capital and financial account of the country will fall as investors invest overseas.
\subsection{Costs of a balance of payment deficit}
A persistent balance of payments deficit means that the country's residents are importing more than the country is exporting. There will be a greater supply of the country's currency in the market, and so the currency depreciates. If the government is pursuing a controlled exchange rate, it will have to intervene and buy domestic currency and sell foreign exchange reserves to prevent the exchange rate from falling too much, However, this cannot go on indefinitely as the country has a finite amount of foreign reserves. The currency will eventually be allowed to depreciate.

A depreciating currency will lead to a worsened terms of trade as there is a decrease in the foreign price of exports and an increase in the domestic price of imports, so more goods are exported to support a given volume of imports.

Sometimes, balance of payment deficits are financed by foreign borrowing. Since the money borrowed has to be repaid in the future with interest, this lowers the standard of living of future generations.

There will also be a fall in the stock of the country's assets if the balance of payment deficit is financed through the sale of assets. The sale of assets also means future investment income is not received by the country, which reduces the standard of living of future generations.
\end{document}