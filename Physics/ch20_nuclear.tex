\documentclass[Physics.tex]{subfiles}
\begin{document}
\chapter{Nuclear Physics}
\section{\upalpha{} particle experiment}
In Rutherford's \upalpha{} particle scattering experiment, \upalpha{} particles were fired at a thin gold foil.

Most \upalpha{} particles made it through the foil undeflected or with a small deflection. This showed that the probability of an \upalpha{} getting close to the positive charge in the nucleus is small, which indicated that the nucleus occupies only a small amount of space.

Some \upalpha{} were deflected through large angles, indicating that the atom must have a positively-charged nucleus of very small dimensions in order to provide the large force required for such a deflection.

Few (but not zero) \upalpha{} were reflected backwards. This further proved that the nucleus was small and very massive.
\section{Nuclear structure}
The nucleus contains positive protons and neutral neutrons, which are collectively known as neutrons. Compared to the atom which has diameter on the order of \SI{1e-10}{\metre}, the nucleus has diameter on the order of \SI{1e-15}{\metre}. Most of the mass of an atom is in the nucleus.

Atoms consist of a positive and small nucleus orbited by negative electrons in defined orbitals and a lot of empty space.

The \sldef{proton number} or \sldef{atomic number} \(Z\) is the number of protons in a nucleus. The \sldef{nucleon number} or \sldef{mass number} \(A = Z + N\) is the number of nucleons in a nucleus. \(N\) is the number of neutrons. A \sldef{nuclide} is a particular species with a unique proton number and mass number represented by \slnuclide{\(Z\)}{\(A\)}{X} e.g. carbon-12: \slnuclide{6}{12}{C}.

\sldef{Isotope}s are atoms that have the same number of protons but different number of neutrons. The abundance of isotopes is generally constant throughout the natural world.
\section{Atomic mass unit}
Atomic masses are usually expressed in terms of the \sldef{atomic mass unit} (\si{\amu}). \SI{1}{\amu} is one-twelfth the mass of the carbon-12 atom.

Masses can also be expressed as relative atomic masses. The relative atomic mass of an atom is the ratio of the mass of the atom to the unified atomic mass unit; it is numerically equal to the atomic mass, but is unitless.
\section{Mass-energy equivalence}
As a result of special relativity, mass and energy are equivalent, and their conservation laws can be unified into the law of conservation of mass-energy. Mass and energy are related by \begin{equation}E = mc^2\end{equation}

In nuclear physics (and quantum physics) the \sldef{electron volt} (\si{\electronvolt}) is commonly used. \SI{1}{\electronvolt} is the energy gained by a charge equal to that on an electron in moving through a potential difference of \SI{1}{\volt}.
\section{Binding energy}
When separate nucleons come together to form a nucleus, the resulting nucleus has less mass than the separate nucleons as some of the mass has been converted into potential energy holding the nucleons together. This energy is known as the binding energy.

Formally, the \sldef{binding energy} of a nucleus is the minimum energy required to completely separate the nucleus into its constituent nucleons and protons.

The difference in mass caused by binding energy is known as the mass defect. Formally, the \sldef{mass defect} of a nucleus is the difference between the mass of the separated nucleons and the combined mass of the nucleus.

The mass defect can easily be calculated as in the definition, and the binding energy from the mass defect. \begin{align}\begin{split}\Delta M &= Zm_p + Nm_n - M_\text{nucleus}\\&= Zm_p + Nm_n + Zm_e - M_\text{atom}\end{split}\\E_B &= \Delta Mc^2\end{align}

The binding energy per nucleon is an indicator of the stability of the nucleus. Naturally, the more binding energy per nucleon, the more stable the nucleus.

In general, the binding energy per nucleon increases sharply from the lighter elements and peaks at iron before falling gradually. This implies that iron is the most stable element -- and it is.
\section{Nuclear processes}
In nuclear reactions, nucleons are rearranged similar to how atoms are rearranged in chemical reactions. In all nuclear processes, nucleon number, proton number i.e. charge, momentum and mass-energy must be conserved.

Nuclear reactions can be represented by equations just like chemical reactions e.g. \mbox{\ch{^{1}0n + ^{14}7N -> ^{14}6C + ^{1}1H}}. The previous reaction can also be represented as \mbox{\ch{^{14}7N ~(n,~p)~ ^{14}6C}}.

To find the energy released in a nuclear process, just start with conservation of energy, which leads you to \begin{align}\nonumber E_\text{reactants} &= E_\text{products} + E_\text{released}\\\nonumber\Rightarrow E_\text{released} &= E_\text{reactants} - E_\text{products}\\&= E_{B\text{ (products)}} - E_{B\text{ (reactants)}}\end{align}

If the products have more energy than the reactants, then energy must be supplied in the form of kinetic energy for the reaction to occur. Otherwise, the reaction can (theoretically) occur at rest as long as all conservations are satisfied.

Energy released in nuclear reactions is carried away either as kinetic energy or by a photon.
\subsection{Nuclear fusion and fission}
\sldef{Nuclear fission} is the disintegration of a heavy nucleus into two lighter nuclei of approximately equal masses. Typically, energy is released as the binding energy of the heavy nucleus is less than that of the fission products.

\sldef{Nuclear fusion} is the combination of two light nuclei to produce a heavier nucleus. Typically, energy is released as the binding energy of the reactants is less than that of the fusion product.

Atoms of elements before iron generally undergo nuclear fusion as they have less binding energy per nucleon than iron. Atoms of elements after iron generally undergo nuclear fission as they have less binding energy per nucleon than iron.
\section{Nuclear decay}
\sldef{Radioactive decay} is the spontaneous disintegration of the nucleus of an atom resulting in the emission of particles or radiation. Radioactive decay is a spontaneous and random process.

A \sldef{spontaneous process} cannot be sped up or slowed down by physical means and is independent of any chemical condition and the decay of other atoms.

Being a \sldef{random process} means that it is impossible to predict which nucleus and when any particular nucleus will disintegrate.
\subsection{Types of radiation}
\upalpha{} particles are simply high-energy helium-4 nuclei -- \slnuclide{2}{4}{He}. When emitted from \upalpha{} decay, they usually possess energies in \si{\mega\electronvolt} range. They have high ionising power and a range of about \SIrange{3}{4}{\centi\metre}, and are easily stopped by a sheet of paper.

\upbeta{} particles are high-energy electrons emitted through \upbeta{}(-minus) decay, when a neutron decays into a proton and an electron. They have speeds up to \(0.5c\). Their ionising power is about a tenth of that of \upalpha{} particles, but they have a range about \num{10} times that of alpha particles. They can only be stopped by a few \si{\milli\metre} of aluminium.

\upgamma{} rays are simply high-energy photons, with frequencies higher than X-ray frequency. \upgamma{} rays have ionising power about one-ten thousandth that of \upalpha{} particles, but they have the longest range, only being stopped by a few \si{\centi\metre} of lead. Gamma decay represents the emission of energy when an excited nucleus returns to its ground state.
\subsection{Decay law}
The rate of radioactive decay of a sample is proportional to the number of radioactive nuclei present. Mathematically, \[-\sldd{N}{t} \propto N \Rightarrow \sldd{N}{t} = -\lambda N\] Solving, this gives us \begin{equation}N = N_0e^{-\lambda t}\label{eq:20.nd.n}\end{equation}

The \sldef{decay constant} \(\lambda\) of a nucleus is the probability of decay per unit time of the nucleus.

The \sldef{activity} of a sample is the average number of atoms disintegrating per unit time. \begin{equation}\begin{split}A &= -\sldd{N}{t} = \lambda N\\&= \lambda N_0e^{-\lambda t} = A_0e^{-\lambda t}\end{split}\end{equation}

The \sldef{half-life} of a radioactive substance is the time taken for half the original number of radioactive nuclei to decay. From \eqref{eq:20.nd.n}, we can derive \begin{gather}t_{1/2} = \frac{\ln2}{\lambda}\\\frac{N}{N_0} = \frac{A}{A_0} = \left(\frac{1}{2}\right)^{t/t_{1/2}} = \left(\frac{1}{2}\right)^n\end{gather} where \(n\) is the number of half-lives.
\subsection{Background radiation}
When dealing with experimental measurements of a radioactive source, we must take into account background radiation.

Background radiation comes from various sources, including \begin{slinenum}
\item the air
\item building materials
\item the soil
\item water
\item human bodies
\item medical sources
\end{slinenum}.
\subsection{Radiation's effects on biological organisms}
Radiation hazards to biological organisms arise from exposure, ingestion and inhalation. Doses of radiation are measured in sieverts (\si{\sievert}).

Severe doses of radiation can lead to immediate effects like radiation burns, and delayed effects like cancer, eye cataracts and hereditary defects due to genetic damage.

Generally, \upalpha{} radiation causes little damage as it can barely penetrate the skin; \upbeta{} radiation causes some damage but can be stopped by surface tissues; and \upgamma{} radiation can penetrate deeply, causing damage to organs deep in the body.
\end{document}