\documentclass[Physics.tex]{subfiles}
\begin{document}
\chapter{Electromagnetism}
A \sldef{magnetic field} is a region of space in which a moving charge or a current-carrying conductor will experience a magnetic force when placed in the field.

The force on a moving charge in a magnetic field is \begin{equation}\mathbf{F} = q\mathbf{v} \times \mathbf{B}\end{equation} From this, the force on a current-carrying conductor with length \(\mathbf{l}\) in the direction of the conventional current \(I\) can be derived to be \begin{equation}\mathbf{F} = I\mathbf{l}\times\mathbf{B}\end{equation}

From this, it can be seen that a charge moving in a magnetic field will experience a force perpendicular to both velocity and the field, which can be predicted using Fleming's left-hand rule (or the direction of the right-handed cross product).

The \sldef{magnetic flux density} is the force acting per unit current per unit length of a conductor placed at right angles to the magnetic field i.e. \begin{equation}\mathbf{B} = \frac{\mathbf{F}}{I\mathbf{l}}\end{equation}

Magnetic flux density has units tesla (\si{\tesla}), which is defined equivalently; one \sldef{tesla} is the magnetic flux density of a uniform magnetic field in which a straight wire carrying a current of \SI{1}{\ampere} placed perpendicular to the field experiences a force per unit length of \SI{1}{\newton\per\meter} in a direction at right angles to both the field and the current.

The force on a current-carrying conductor can be used to measure the magnetic flux density, through a current balance. A wire frame balanced on a pivot with a current through it is placed such that one side is in the magnetic field and experiences a magnetic force; a weight is then placed on the opposite side to balance the frame. With needed lengths, the current, and the mass of the weight, flux density can be found.

A magnetic field will never do any work on a moving charge as the force on the charge is always perpendicular to its movement. Charges moving in a magnetic field will undergo circular motion. If there is an electric field parallel to the magnetic field, charges will move in a spiral.

If an electric field and a magnetic field perpendicular to the electric field are combined, only charges moving with a particular velocity through the fields will move through the field undeflected. This can be used as a ``velocity selector'' of sorts.
\section {Magnetic fields caused by currents}
The magnetic field caused by a long straight wire is circular with its direction predicted by the right-hand grip rule (fingers representing the field). The field lines are concentric circles centred at the wire.

The magnetic field caused by a single circular loop is similar to that of an infinitesimally short bar magnet.

A \sldef{solenoid} is a long cylindrical coil of wire. The magnetic field due to a solenoid is similar to that of a bar magnet, and is also predicted by the right-hand grip rule (fingers representing the current).

When a ferrous core is placed within a current-carrying solenoid, it becomes temporarily magnetised as the magnetic field of the solenoid concentrates within the iron bar. The solenoid and the iron bar together produce a very strong magnetic field, stronger than produced by the solenoid alone, thus forming an electromagnet.

When two parallel conductors carry a current in the same direction, they will experience forces toward the other conductor. When two parallel conductors carry a current in the opposite direction, they will experience forces away from the other conductor. This can be predicted from first principles, using the right-hand grip rule. The force per unit length on each wire \begin{equation}\frac{F}{l} = \frac{\mu_0I_1I_2}{2\pi d}\end{equation}

The \sldef{ampere} is defined based on the above result -- one ampere is the constant current which, if maintained in two straight parallel conductors of infinite length placed \SI{1}{\metre} in vacuum, would produce a force per unit length of \SI{2E-7}{\newton\per\meter} between the conductors.
\end{document}