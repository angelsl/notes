\documentclass[Physics.tex]{subfiles}
\begin{document}
\chapter{Electromagnetic Induction}
\sldef{Magnetic flux} \(\phi\) through a plane surface is defined as the product of the magnetic flux density normal to the surface \(\mathbf{B}\) and the area \(\mathbf{A}\) of the surface i.e. \begin{equation}\phi = \mathbf{B} \cdot \mathbf{A}\end{equation}

\(\phi\) has SI units \si{\weber}. One \sldef{weber} is the magnetic flux through a surface if a magnetic field of flux density \SI{1}{\tesla} exists perpendicularly to an area of \SI{1}{\meter\squared}.

\sldef{Magnetic flux linkage} \(\Phi\) is the product of the number of turns \(N\) of the coil and the magnetic flux linking each turn i.e. \begin{equation}\Phi = N\phi\end{equation}

\sldef{Faraday's law of induction} states that the induced e.m.f. \(\varepsilon\) is directly proportional to the rate of change of magnetic flux linkage i.e. \begin{equation}\varepsilon = -\sldd{\Phi}{t}\end{equation}

\sldef{Lenz's law} states that the induced e.m.f. will be directed such that the current which it causes to flow opposes the change producing it. Lenz's law is a manifestation of energy conservation.
\end{document}