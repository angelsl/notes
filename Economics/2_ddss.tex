\documentclass[Economics.tex]{subfiles}
\begin{document}
\chapter{Competitive Resource Allocation}
A market economy is an economic system in which the decisions regarding investment, production and distribution are based on the interaction between demand and supply forces.

The price mechanism is the process in a market economy where consumers and businesses interact to determine the allocation of scarce resources between competing uses.

Prices serve two functions in the market --- namely, the signalling function and rationing function. In a market economy, prices are determined by demand and supply forces. The decisions of producers determine supply, while the decisions of consumers determine demand.

The market equilibrium occurs at a price where the quantity demanded is equal to the quantity supplied. Ceteris paribus, there is no tendency for this price to change.
\section{Demand}
Demand is the quantities of a product that consumers are willing and able to buy at various prices per period, ceteris paribus. It is determined by the private benefit that consumers get from consuming an additional unit of the good, also known as the marginal private benefit.

The law of demand states that the higher the price of a good, the lower the quantity demanded, c.p., because as prices increase, a given income can purchase fewer units of the good, and consumers may choose to purchase cheaper substitutes, thereby decreasing the demand for this good.

A change in the price of a product causes a movement along the demand curve. A change in non-price factors of demand will cause a shift in demand. An increase in demand means that at every price, the quantity demanded of the good increases, and vice versa.
\subsection{Demand factors}
Income affects demand as when income increases, consumers' purchasing power increases and they are more willing and able to pay for normal goods, so the demand for normal goods goes up.

When the prices of substitutes fall or of complements increase, demand for the good falls.

Tastes and preferences of consumers also affect demand and supply. A change that makes a good less desirable will decrease the demand for a good, and vice versa.

Consumers' expectations regarding future prices can affect demand. E.g.\ if consumers expect that prices will fall in future, consumers may reduce demand in the current period and wait till prices fall so that they can pay the lower price.

Government legislation can influence the decisions of consumers and in turn affect demand. For example, if the government implements a law where all buses must have seatbelts, the demand for seatbelts will increase sharply.

Other factors including interest rate can affect demand. Interest can be seen as the cost of borrowing; if it is cheaper to borrow, for goods like houses, consumers may be more willing and able to purchase, so demand increases.
\section{Supply}
Supply is the quantities of the product that suppliers are willing and able to sell at various prices per period, ceteris paribus.

The law of supply states that the higher the price of a good, the higher the quantity supplied, c.p., because as prices increase, the profits of firms per unit of the good sold increases, inducing firms to supply more, and in the short run, the cost of producing an additional unit generally increases as output increases, so firms require a higher price to produce more.

A change in the price of a product causes a movement along the supply curve. A change in non-price factors of supply will cause a shift in supply. An increase in supply means that at every price, the quantity supplied of the good increases, and vice versa.
\subsection{Supply factors}
Changes in the unit cost of production will affect supply as an increase in the unit COP will result in a lower potential profit per unit, so producers decrease the good's production and supply drops.

If a new method of production that is able to produce a good more efficiently is found, more output will be produced with the same amount of input. C.p., this means a lower unit cost of production.

If the profitability of a good in competitive supply increases or in joint supply decreases, supply of the good decreases.

Government policy, like taxes or subsidies, can also affect the supply of a good.

If the number of sellers decreases, the supply drops, and vice versa.

If an expectation that prices will increase in future exists, suppliers may decrease supply in the current period to take advantage of the higher prices in future.
\section{Market adjustment process}
If demand and\slash{}or supply shifts, the market adjustment process takes place. At the initial equilibrium \(E_1\), equilibrium price is at \(P_1\) and output is at \(Q_1\). After the shift, at the initial equilibrium, quantity supplied is \(Q_s\) while quantity demanded is \(Q_d\). There is a shortage\slash{}surplus of \(Q_s Q_d\) units of the good, causing upward\slash{}downward pressure on prices. As prices increase\slash{}decrease, quantity demanded decreases\slash{}increases while quantity supplied increases\slash{}decreases until both are equal at \(Q_2\). The final equilibrium is at \(E_2\), with equilibrium price increasing/decreasing to \(P_2\) and output increasing\slash{}decreasing to \(Q_2\).

If there is a simultaneous shift in demand or supply, price or quantity may be indeterminate, unless it can be established that the shift in demand is greater than the shift in supply, or vice versa.
\section{Related markets}
Markets can be related to each other. Suppose we have a good A.

A complement to A is a good that must be used with A to satisfy wants and needs. If the price of the complement increases, the demand for A will fall.

A substitute to A is a good that satisfies the same needs or wants as A. If the price of a substitute falls, the demand for A will fall.

The demand for A can be a derived demand if another good uses A in its production, e.g.\ if A is steel, then the demand for A is derived partly from the demand for cars and ships. If the demand for cars and ships increases, then the demand for A increases.

A good in joint supply with A means that it is produced jointly from the same resources as A, e.g.\ if A is beef, then the good could be leather: leather and beef can both be produced from the same cow. The increase in supply of beef will lead to an increase in supply of leather, and vice versa.

A good in competitive supply with A means that it is produced exclusively from the same resources as A, and using the resource to produce A means that it cannot be used to produce the other good. Ceteris paribus, increasing the supply of A will decrease the supply of the other good.
\section{Surplus and welfare}
Consumer surplus is the difference between what consumers are willing to pay for a unit of a good and what they actually pay for that unit of a good. It is a measure of consumer welfare.

Producer surplus is the difference between the revenue producers receive from the sale of a unit of the good and the price at which they are willing to supply that unit. It is a measure of producer welfare.

Societal welfare is the sum of consumer welfare and producer welfare. On a demand and supply diagram, it is represented by the triangle formed by the \(P\)-axis, the demand curve, and the supply curve, up to the market equilibrium.

Total revenue or total expenditure \[\slmf{TR} = \slmf{TE} = \slmf{price per unit} \times \slmf{units sold or purchased}\]
\section{Elasticities of demand and supply}
\subsection{PED and PES}
Price elasticity of demand measures the degree of responsiveness of quantity demanded of a good to a change in the price of the good, ceteris paribus. It involves a movement along the demand curve. \PE[D] is given by percentage change in quantity demanded of the good over percentage change in price of good. \PE[D] is always negative as the price and quantity demanded of a good are inversely related.

Price elasticity of supply measures the degree of responsiveness of quantity supplied of a good to a change in the price of the good, ceteris paribus. It involves a movement along the demand curve. \PE[S] is given by percentage change in quantity supplied of the good over percentage change in price of good. \PE[S] is always positive there is a direct relationship between price and quantity supplied of a good.

Values of \PE[S] and \PE[D] have certain meanings.
\begin{itemize}
\item \(\PE = 0\) means demand or supply is perfectly price inelastic: quantity demanded or supplied is constant as price changes.
\item \(0 < \PE < 1\) means demand or supply is price inelastic: a given change in price leads to a less than proportionate change in quantity demanded or supplied.
\item \(\PE = 1\) means a given change in price leads to an exactly proportionate change in quantity demanded or supplied.
\item \(1 < \PE < \infty\) means demand or supply is price elastic: a given change in price leads to a more than proportionate change in quantity demanded or supplied.
\item \(\PE = \infty\) means demand or supply is perfectly price elastic: an infinite amount will be bought or sold at some price or below, but zero above that price.
\end{itemize}
\subsection{PED and PES factors}
There are numerous factors affecting \PE[D].

The availability of substitute goods affects \PE[D]. The greater the availability of substitutes for a good and the closer these substitutes are, the more price elastic the good, as when the price of the good increases, consumers are likely to switch to these substitutes.

The greater the proportion of income spent on the product, the more price elastic the good, as when the price of the good increases, people are more likely to reduce consumption.

The longer the period of time under consideration, the greater the effect of a change in the price of the good on the quantity demanded (the more price elastic the good), as over time, new substitutes can be developed, habits can be broken, etc.

If a good is addictive or habit forming, like cigarettes or liquor, it will tend to be more price inelastic, as consumers will not easily break their habits when price increases.

If a good is a luxury (\(\YED{}> 1\)), then it is likely to be more price elastic as they generally can be forgone if needed when price increases, compared to a necessity --- people are not likely to reduce their consumption of food significantly when price increases.

There are also numerous factors affecting \PE[S].

The behaviour of cost as output increases affects \PE[S]. If there are severe diseconomies of scale as output increases, supply will tend to be more price inelastic as firms will be discouraged from reducing more if price increases as the higher price may not be enough to offset the increase in unit cost of production, which will cause a decrease in profit per unit.

The shorter the period in consideration, the more price inelastic the supply as firms cannot increase production easily in a short period due to fixed factors of production.

In the short run, markets with spare productive capacity will tend to have a more price elastic supply as producers can increase their production easily by utilising their spare capacity.

If a product can be stored cheaply with minimal loss of quality, supply will tend to be price elastic while stocks last as firms can increase their output by selling from their stocks when price increases.

If there is factor substitutability i.e.\ one factor of production can be easily substituted with another, or factor mobility i.e.\ it is easy to move resources from place to place or industry to industry, then supply will tend to be more price elastic.

If there are barriers to entry then supply will tend to be more price inelastic as it is difficult for new firms to enter the market when price increases, and so supply is limited to the limited suppliers.
\subsection{YED and XED}
Income elasticity of demand measures the degree of responsiveness of demand of a good to a change in income, ceteris paribus. It involves shifts in the demand curve in response to changes in income. \YED{} is given by percentage change in demand of a good over percentage change in income.

If \YED{} is negative, it means demand of the good decreases as income increases i.e.\ the good is an inferior good. If \YED{} is positive it means demand of the good increases as income increases i.e.\ the good is normal.

For \(0<\YED<1\), it means the demand of the good increases less than proportionately for a given increase in income, and the good is a necessity. The demand of the good is income inelastic. For \(1<\YED<\infty\), it means the demand of the good increases more than proportionately for a given increase in income, and the good is a luxury. The demand of the good is income elastic.

The main factor affecting \YED{} is the degree of necessity of the good, or the nature of the good. The higher the degree of necessity for a normal good, the lower the \YED{} of the good.

Cross elasticity of demand of a good A with respect to the price of another good B measures the degree of responsiveness of demand of good A to a change in price of good B, ceteris paribus. It involves shifts in the demand of A in response to changes in price of B. \XED{} is given by percentage change in demand of good A over percentage change in price of B.

If \(\XED < 0\), the goods are complements, and if \(\XED > 0\), the goods are substitutes. If \(\XED = 0\), the two goods are independent i.e.\ they have no relationship to each other. The greater the magnitude of \XED, the closer the relationship of the two goods.

The main factor affecting \XED{} is the closeness of the two goods i.e.\ whether they are substitutes or complements and how close of a substitute or complement they are.
\subsection{Limitations of elasticities}
Elasticities of demand and supply can be useful, but they can be limited.

Elasticity figures studied are usually outdated as data is usually collected prior to them being used, so they are rarely, if ever, representative of current economic conditions.

Also, the ceteris paribus assumption does not apply in the real world. Elasticities only consider two variables at once and all other factors must remain unchanged. This never happens in reality.
\end{document}
