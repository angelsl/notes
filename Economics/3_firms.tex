\documentclass[Economics.tex]{subfiles}
\begin{document}
\chapter{Market Structure}
\section{Objectives of firms}
Profit is the difference between the total revenue and total cost. In economics, firms are assumed profit-motivated and so will price to maximise profits.

Profit is maximised if a firm produces at an output where marginal cost equals marginal revenue when marginal cost is increasing, where marginal cost (\mrg{C}) is the cost of producing the last unit of output, and marginal revenue (\mrg{R}) is the revenue gained by selling the last unit of output. If the firm produces at a level below that output, then producing an additional unit of output adds more to revenue than to cost, increasing profit. While \mrg{R} > \mrg{C}, profits are increased by producing more. If the firm produces at a level above that output, then producing an additional unit of output adds more to cost than revenue, decreasing profit. While \mrg{C} > \mrg{R}, profits are increased by producing less. Therefore, profit is maximised at the point where \(\mrg{C} = \mrg{R}\) when \mrg{C} is increasing.

In reality, firms may not be able to identify the profit maximising price and output, as there is lack of information and the ceteris paribus assumption does not hold, as conditions of demand and supply are continuously changing, so even if firms know the profit maximising output level, they might not be able to determine the correct price to sell their goods. Notwithstanding that, it is difficult for firms to even determine the demand curve they face or predict how competitors may behave in response to their actions.

Many firms thus practice cost-plus pricing, where they estimate their long run average cost and add a profit margin to arrive at the price.

Some firms may choose not to maximise profits to avoid unwanted attention from the government, who may be concerned that these firms are exploiting customers, or from other firms, who are likely to be interested in any profitable assets they have.

Firms can choose to maximise sales revenue, as it can make it easier for them to take out loans. Sales revenue is maximised when \(\mrg{R} = 0\), as total revenue will be at a peak at this point. The level of output is likely to be past the profit maximising point, but the firm can still earn supernormal, normal or subnormal profits, depending on total revenue and total cost at this point.

Firms can choose to maximise sales volume, as employees' status and salary are often more linked to the size of the firm than its profitability. Sales volume is maximised when average cost equals average revenue. A firm can choose to produce above this level, but this level is the highest output level that the firm can earn normal profits. The firm will always earn normal profits at this point.

Firms can choose to profit satisfice, where firms aim for a profit level that will keep shareholders happy.

Firms may be reluctant to accept the risks and pressures associated with fiercely competitive policies, or they may be aiming to satisfy other stakeholders. Stakeholders include consumers, who want low prices and high quality; workers, who want high wages and job satisfaction and security; suppliers, who want high prices; community, who want employment without congestion; and environmentalists, who want clean environments and the conservation of flora and fauna. Profit satisficing may conflict with profit maximisation in the short run, but is compatible in the long run. Showing concern for the environment, for example, by avoiding nature reserves or not selling genetically modified foods, may raise a firm's costs, but it will also provide the firm with good publicity and may increase demand for the firm's products, creating brand loyalty. In the long run, revenue may rise more than costs, increasing profit.
\section{Costs of production}
A plant is a site where production or distribution of a product occurs. A firm is a decision-making unit that hires factors, combines them to create output and then sell the output; firms own plants. An industry is a collection of firms producing similar goods and services.

Factors of production are inputs used to produce output. They can be either fixed factors, which are factors whose quantity cannot be increased or decreased in a given time period e.g.\ land and machines, or variable factors, which are factors whose quantity can be changed within the given time period, e.g.\ labour and raw materials.

In production, a short-run time period is a time period in which there is at least one fixed factor; output can only be increased by using more variable factors. The long-run is when all factors are variable.

Typically, at small outputs, the marginal cost is high; it decreases to a minimum as output expands, and then starts increasing past that minimum. As the firm expands its output initially, it will employ more units of variable factors, causing the combination of fixed and variable factors to improve e.g.\ due to specialisation, raising the marginal productivity of the variable factor. Assuming the price of the variable input is constant, \mrg{C} will fall. Past a certain output, the law of diminishing marginal returns takes effect, and the proportion of factors becomes less efficient, reducing the marginal productivity of the variable factor and so \mrg{C} rises.
\section{Economies of scale}
The long run costs of firms are generally affected by economies and diseconomies of scale.

Dis\slash{}economies of scale are rises\slash{}falls in unit cost enjoyed by the firm or firms from growth of the firm (internal) or expansion of the entire industry (external). Internal economies or diseconomies of scale are represented by movements along the long run average cost (\slmf{LRAC}) curves of the firm, while external economies or diseconomies of scale are represented by \slmf{LRAC} shifts.
\subsection{Internal economies}
Technical economies can lead to internal economies of scale. They occur when a firm is able to have functional specialisation of labour, allowing the efficiency of labour to increase, lowering the unit COP. Also, some inputs e.g.\ machines can only be employed in large indivisible units that can produce a large output, which is unsuitable for small firms, but greatly economical for larger firms. Some inputs like machines can result in greater output with a less than proportionate increase in cost e.g.\ a double decker bus can carry double the passengers a single decker bus can, but the cost of purchasing and operating it may not necessarily double.

Internal economies of scale can arise due to marketing economies. As a firm becomes bigger, it purchases its inputs in bulk and so may secure discounts on purchases of inputs, as suppliers are eager to get the firm's orders. A larger firm is able to spread its advertising costs over a larger output and so unit cost is reduced.

Internal economies of scale can arise due to financial economies, where they can obtain loans at lower interest rates due to greater credit worthiness; it can also issue shares to the public to raise funds in the capital market.

Internal economies of scale can arise due to risk-bearing economies, where larger firms can deal with risks better, through diversification: if one product is not selling well, it can depend on its other products for revenue, so it is less likely to shut down than a smaller firm selling only one or two products.

Managerial economies of scale arise when firms can hire professionals in various fields to specialise and lead different departments that can help to increase a firm's output, lowering its unit cost.
\subsection{Internal diseconomies}
Internal diseconomies of scale can occur due to managerial diseconomies, which are the most common reason, as other types of diseconomies, like technical diseconomies, can simply be avoided by creating smaller plants. A firm can grow so large that it becomes difficult to manage, and more bureaucracy is involved in making decisions, slowing operations down. Paperwork can reduce work efficiency, resulting in lower productivity and thus higher unit cost. Management can find it hard to coordinate the operations of large firms, resulting in inefficiency.

Employees in a large organisation can experience feelings of alienation as the firm may become insensitive to the needs of its workers, affecting worker morale, resulting in lower productivity. An employee in a large organisation that receives a fixed salary may have little motivation to be efficient, as the quality of his work rarely translates into greater salaries.

Financial diseconomies can occur when big firs become too big and borrow too much without repaying debts, affecting the firm's credit worthiness. Banks would be reluctant to offer loans to the firm, and may begin to charge higher interest, increasing the cost of borrowing.

Risk-bearing diseconomies can occur if one branch of a firm has poor performance, which leads to negative spill over effects on other branches, increasing cost of production.

The minimum efficient scale of production is the output at the lowest point of the \slmf{LRAC}. It represents the output after which internal diseconomies will start taking effect.
\subsection{External economies}
External economies of scale can occur as when an industry expands, amenities will be developed. As firms set up in an area, the government will develop amenities for the industries, reducing costs for individual firms and facilitating production. The government will also develop a better transport network so raw materials can be transported to, and outputs away from, the firms more efficiently, reducing transport costs.

External diseconomies of scale can occur when an industry expands excessively.

The increased demand for factors of production can lead to a shortage of the factors, leading to higher prices and so higher unit cost at all levels of output.

If too many firms concentrate in one area, this can result in traffic congestion, leading to loss of man-hours as time is wasted waiting for traffic, etc. Noise, water and air pollution may also result, forcing the government to impose taxes and fines. This all leads to increased unit costs at all outputs.
\section{Size of a firm}
Various demand factors can affect the size of a firm.

If the demand for a good is small, a firm that produces that good only or similar goods with small demand will remain small simply because the market for their produces is small.

Some consumers prefer firms that are more personal with their customers, something large firms cannot as easily provide. Thus, smaller firms can usually coexist with larger ones.

Supply factors also affect the size of a firm. Some goods are naturally suited for small firms e.g.\ dentists, where a single dentist can only work for so many hours before his efficiency drops.

A firm will enter short run shutdown when its \(\slmf{AVC} > \slmf{AR}\) or equivalently \(\slmf{TVC} > \slmf{TR}\), as when this is the case, the firm would incur the least loss by producing no output. There is no such thing as long run subnormal profit, as a firm would simply exit the market.
\section{Features of market structures}
Market structure is the way in which goods and services are supplied by firms in an industry. The market structure a firm operates in will determine its behaviour, or pricing and output decision and competitive strategies, and performance, or profitability and efficiency level.
\subsection{Barriers to entry}
Some markets have barriers to entry (BTE). A barrier to entry is something that prevents the entry of new firms into an industry, thereby limiting the amount of competition faced by existing firms. There are various types of barriers to entry.

Natural barriers to entry include economies of scale and natural monopolies, among others.

Economies of scale can be a barrier to entry when the minimum efficient scale of production is large compared to market demand; firms incur huge outlays in terms of infrastructural investment and a large output is needed to produce a good at its lowest unit cost. Therefore, new small firms entering such a market would find it difficult to compete.

In the extreme case where economies of scale persist past market demand, a natural monopoly arises where the MES is so large to the extent where one firm alone can satisfy the entire market demand, and if the demand is split equally with another firm, both firms' average costs increase to the point where they earn subnormal profits.

If a firm owns the resources needed to produce a particular good, then that firm can prevent other firms from entering the industry. E.g.\ Debeers owns most of the world's diamond mines and so monopolises the world's production of diamonds.

Some manufacturing industries use capital-intensive production techniques, so a large capital outlay is required to start production that may hinder entrants to the market.

State-created barriers to entry are, as in the name, created by the state.

Licenses are exclusive permits to produce owned by a firm. If a firm does not own a license to produce a controlled good, it cannot produce that good. Thus, licenses act as a barrier to entry.

Patents, copyrights and trademarks are granted by the government; they grant a firm the exclusive license to produce a good or use a specific technique for a period, in order to promote innovation. Other firms that wish to produce the same good or use the same method must pay royalties to the firm.

Firms can create barriers to entry to try to enjoy monopoly power and long-run supernormal profit.
\begin{itemize}
\item Advertising and brand name image creation helps to create brand loyalty in a market, making it more difficult for a newcomer to enter.
\item Firms can produce many varieties of a product (product proliferation), making entry difficult.
\item Firms may maintain excess productive capacity, discouraging potential entrants, as they know that incumbent firms will easily increase output to depress price when they enter.
\item Firms can practice predatory pricing, which is the setting of price to levels so low that entrants are discouraged from entering the market.
\item Firms can also practice limit pricing, which is when firms set prices low and restrict their profits to avoid attracting potential entrants.
\item Firms can practice restrictive practices e.g.\ exclusive dealing arrangements with merchants that stock only that firm's products, like discounts or other favourable trading terms.
\end{itemize}
\subsection{Price discrimation}
Price discrimination occurs when a firm sells the same product to different groups of consumers at different prices for reasons not due to cost, to increase profits by reducing consumer surplus.

For price discrimination to be possible, the firm must be a price setter, markets must be separated and no resale between segmented markets should be possible, and the \PE[D] in the segmented markets must be different to enable the monopolist to charge different prices.

Firms need to be price setters as they need to be able to set different prices without losing market share completely, with their market power derived from product differentiation. The market needs to be separated with no arbitrage possible, otherwise a consumer can easily defeat price discrimination by buying from someone who has purchased the good at a lower price.

1st degree price discrimination is when the seller charges the maximum price that a consumer is willing to pay for that unit of output e.g.\ via an auction. This removes the entire consumer surplus.

2nd degree price discrimination is when the seller charges the same consumer different prices for different quantities sold e.g.\ car park charges. When a natural monopoly is regulated, it may practice two-tier pricing, where it charges some users a higher price and others a lower price, so that the monopoly can survive while also being more equitable.

3rd degree price discrimination is when the seller divides his consumers into different groups and charges a different price to each group e.g.\ in movie tickets or buffets. In this case, the firm produces at the point where the combined \(\mrg{C} = \mrg{R}\); the \mrg{C} is then equated to the \mrg{R} of separate markets to find the output distribution between the markets.
\subsection{Collusion}
Firms can choose to collude or compete. When they collude, they can form a cartel, in which firms coordinate their activities and act as a single firm. They can also initiate price leadership, where firms follow the largest firm's prices (dominant price leadership) or use prices that best reflect market conditions (barometric).

Cartels are generally illegal in countries, and they may not last very long because of disputes and the incentive to cheat e.g.\ selling below the agreed price or selling more than their assigned quota.
\subsection{Competition}
When competing, firms can do so using price or non-price strategies. Price strategies are those directly related to decreasing their price, while non-price strategies are things like advertising or product proliferation.

Price competition occurs when a firm lowers its price in order to attract customers away from rivals. This can become full-fledged price wars, where firms continually lower prices until other firms are driven out of the market. This may be good, but it may also mean the firm that wins now gets to enjoy monopoly power.

Forms of non-price competition include product development and advertising. Product development differentiates a firm's products from others, and advertising informs consumers of and persuades consumers to purchase the firm's goods, in order to increase demand for their product as well as make their product less substitutable by other firms' i.e.\ decrease \PE[D] for their product.

However, non-price competition tends to involve extra costs that may not be worth the benefit received from undertaking them.
\section{Spectrum of market competition}
\begin{itemize}
\item In the perfectly competitive market, there are a large number of firms selling a homogenous product. There are no BTE to the market, and there is perfect knowledge in the market i.e.\ producers know all costs and there are no industry secrets.
\item In the monopolistic competitive market, there are a large number of firms selling differentiated products. There are little BTE to the market, but there may not be perfect knowledge.
\item In the oligopolistic market, there are a few large firms, selling differentiated or homogenous products.
\item Finally, in a monopoly, there is one firm selling a unique product, and there are very strong BTE.
\end{itemize}

The characteristics of market structures lead to their behaviours.
\subsection{Market power}
In markets where there are a large number of firms selling homogenous products, each firm produces a small proportion of the market output i.e.\ has a small market share, and so a change in any firm's output will not significantly change the market price. Thus firms in such a market, i.e.\ perfectly competitive market, are price takers, and have no market power, and face a perfectly price elastic demand.

Conversely, firms that sell a differentiated product or have a large market share can raise their price without losing all their customers and so they are said to have a degree of market power as they can restrict output to increase price above marginal cost. These firms face a downward-sloping demand.
\subsection{Types of profit}
In the short run, all firms can earn all kinds of profits (supernormal, normal and subnormal), but the presence of barriers to entry affects the long run profits of a firm. Firms in markets with no barriers to entry i.e.\ perfectly competitive or monopolistic competitive markets cannot earn supernormal or subnormal profits in the long run.

If firms are earning supernormal profits in the short run, new firms will be attracted by the supernormal profits and enter the industry. Firms earning supernormal profits will also expand and so industry output increases at all prices, therefore supply increases, leading to a fall in price, ceteris paribus. Supernormal profits will be competed away until all firms earn only normal profits.

If firms are earning subnormal profits in the short run, some firms will shut down so industry output decreases at all prices, therefore supply decreases, leading to an increase in price, ceteris paribus. Total revenue and thus profits will increase until all firms earn normal profits.

When all firms are earning normal profits, there will be no incentive for new firms to enter; existing firms stay in the industry as their revenue is sufficient to cover all their costs, and the industry is now in long run equilibrium as there is no tendency for firms to move in or out of the industry.

Firms in markets with barriers to entry i.e.\ oligopoly or monopoly can earn supernormal or normal profits in the long run, due to the significant barriers to entry preventing firms from entering to compete away supernormal profits. No firm will have subnormal profits in the long run as such firms will exit.
Firms in markets with market power i.e.\ not perfectly competitive are allocative inefficient.
\subsection{Efficiency}
Assuming the firm is profit maximising, it will produce at a level where \(P > \mrg{C}\), so consumers value the last unit of the good more than it costs to produce it; the good is thus underproduced: increasing output can increase consumer's surplus and thus welfare. This is allocative inefficiency. There is a deadweight loss caused to society that increases with the market power a firm has.

Firms in markets with no market power are allocative efficient. In a perfectly competitive industry, the price consumers pay, which reflects the value consumers place on extra units of the good, is equal to the cost of producing the last unit of output. When \(P > \mrg{C}\), the value consumers place on the last unit of output is greater than the cost of producing it, so more should be consumed, and vice versa. Thus allocative efficiency occurs when \(P = \mrg{C}\).

Productive efficiency occurs when firms are producing a given output at the lowest possible cost i.e.\ they are producing on their \slmf{LRAC}. Since firms are assumed to be profit maximising, they would want to minimise costs, so they will choose the lowest cost method of production possible; therefore all firms are productive efficient.

Dynamic efficiency occurs when product and process innovation occurs. Firms that can earn supernormal profits in the long run are generally dynamic efficient, as supernormal profits provide funds for a firm to channel to funding research and development, so firms with supernormal profits are able to innovate. Such firms are usually forced to innovate to maintain their dominant position in the market. Generally, the more competition a firm faces, the more they need to innovate to maintain their dominant position, so oligopolies are more dynamic efficient than monopolies.

X-inefficiency is the difference between efficient behaviour of firms assumed by theory and their observed behaviour. It occurs when efficiency is not achieved due to a lack of competitive pressure. Generally, firms experiencing more competition are less X-inefficient as they have competitive pressure to force them to keep \slmf{AC} low, and innovate to maintain their position in the market. The perfectly competitive market is completely X-efficient because if it is less cost efficient than other firms, it will make subnormal profit and be driven out of the market; firms in such a market also have no incentive to advertise as all firms make homogenous products.

Advertising is usually considered wasteful as it is an extra cost on top of what is needed to operate. In this light, the perfectly competitive and monopoly markets are not as wasteful as there is no need for them to advertise; the monopolistic competitive and oligopolistic markets tend to advertise more, and so waste more.
\subsection{Equity}
Equity is the fairness and justness of the allocation of resources.

Generally, a firm structure in which firms can earn supernormal profits in the long run is inequitable as supernormal profits go to shareholders, who generally consist of higher-income earners, worsening the income distribution as the rich get more.

However, even if a firm cannot earn supernormal profit in the long run, it does not guarantee equity as there is no guarantee that goods produced are distributed to individuals fairly, especially if the good is a necessity; those who earn higher income have the ability to purchase more, while the poor might not get enough because they do not have enough money to purchase what they need.
\subsection{Variety}
Generally, firms that sell differentiated products produce more variety and so allow consumer choice, which is a part of consumer welfare. Only the perfectly competitive market cannot have variety; in other markets, there can be a large variety of goods.
\subsection{Mutual interdependence}
Firms in market structures with few firms tend to exhibit mutual interdependence. Collusion is an extreme case of this, of course, but its effects are also seen when these firms compete, in the form of price rigidity, which is modelled using a kinked demand curve.

Two key assumptions are made in the kinked demand model, namely that if a firm lowers price, rival firms will follow suit and lower their price to avoid losing customers to the former firm; and if a firm raises price, rival firms will stay at the original price to gain customers from the former firm. These assumptions result in a demand curve that is price elastic above the equilibrium price, and price inelastic below the equilibrium price, and so there is a kink at the equilibrium price.

The corresponding \mrg{R} curve is made from the composition of the two separate \mrg{R} curves that the two segments of demand would separately produce, and there is a discontinuous range in the \mrg{R} at equilibrium output; if \mrg{C} at the equilibrium varies but stays within the discontinuous range, price will not change and so prices are rigid.
\end{document}
